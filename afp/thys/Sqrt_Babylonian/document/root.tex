\documentclass[11pt,a4paper]{article}
\usepackage[T1]{fontenc}
\usepackage{isabelle,isabellesym}
\usepackage{amssymb}

% this should be the last package used
\usepackage{pdfsetup}

% urls in roman style, theory text in math-similar italics
\urlstyle{rm}
\isabellestyle{it}

\newcommand\isafor{\textsf{IsaFoR}}
\newcommand\ceta{\textsf{Ce\kern-.18emT\kern-.18emA}}
\newcommand\nats{\mathbb{N}}
\newcommand\reals{\mathbb{R}}
\newcommand\rats{\mathbb{Q}}
\newcommand\fieldext[2]{#1[#2]}
\newcommand\ratsb{\fieldext\rats{\sqrt b}}

\begin{document}

\title{Computing N-th Roots using the Babylonian Method\thanks{This research is supported by FWF (Austrian Science Fund) project P22767-N13.}}
\author{Ren\'e Thiemann}
\maketitle

\begin{abstract}
  We implement the Babylonian method \cite{Babylon} to compute n-th roots of numbers.
  We provide precise algorithms for naturals, integers and rationals, and
  offer an approximation algorithm for square roots within linear ordered fields. Moreover, there
  are precise algorithms to compute the floor and the ceiling of n-th roots. 
\end{abstract}

\tableofcontents


% include generated text of all theories
\input{session}

\section*{Acknowledgements}
We thank Bertram Felgenhauer for 
for mentioning Cauchy's mean theorem during the formalization of
the algorithms for computing n-th roots.

\bibliographystyle{abbrv}
\bibliography{root}

\end{document}
