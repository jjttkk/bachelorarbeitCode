\documentclass[11pt,a4paper]{article}
\usepackage[T1]{fontenc}
\usepackage{isabelle,isabellesym}
\usepackage{amssymb}
% this should be the last package used
\usepackage{pdfsetup}

% urls in roman style, theory text in math-similar italics
\urlstyle{rm}
\isabellestyle{it}


\begin{document}

\title{Region Quadtrees}
\author{Tobias Nipkow\\Technical University of Munich}
\maketitle

\begin{abstract}
These theories formalize \emph{region quadtrees}, which are traditionally used
to represent two-dimensional images of (black and white) pixels.
Building on these quadtrees, addition and multiplication of recursive
block matrices are verified.
The generalization of region quadtrees to $k$ dimensions is also formalized.
\end{abstract}

\section{Introduction}

These theories formalize so-called \emph{region quadtrees}, as opposed to \emph{point quadtrees}
\cite{Samet84,Samet90,Aluru04}.
The following variants are covered:
\begin{itemize}
\item Ordinary region quadtrees.
\item Block matrices based on region quadtrees. Operations: matrix addition and multiplication.
  Based on the work of Wise \cite{Wise84,Wise85,Wise86,Wise87,Wise92}.
\item A $k$-dimenstional generalization of region quadtrees.
This is inspired by the $k$-dimensional point trees by Bentley \cite{Bentley75,FriedmanBF77}
which have already been formalized by Rau \cite{KD_Tree-AFP}.
\end{itemize}
For the details of the operations covered see the individual theories.

\tableofcontents

\newpage

% include generated text of all theories
\input{session}

\bibliographystyle{abbrv}
\bibliography{root}

\end{document}
