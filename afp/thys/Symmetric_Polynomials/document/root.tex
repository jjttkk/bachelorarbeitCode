\documentclass[11pt,a4paper]{article}
\usepackage[T1]{fontenc}
\usepackage{isabelle,isabellesym}
\usepackage{amsfonts,amsmath,amssymb}

% this should be the last package used
\usepackage{pdfsetup}

% urls in roman style, theory text in math-similar italics
\urlstyle{rm}
\isabellestyle{it}


\begin{document}

\title{Symmetric Polynomials}
\author{Manuel Eberl}
\maketitle

\begin{abstract}
A symmetric polynomial is a polynomial in variables $X_1, \ldots, X_n$ that does not discriminate between its variables, i.\,e.\ it is invariant under any permutation of them. These polynomials are important in the study of the relationship between the coefficients of a univariate polynomial and its roots in its algebraic closure.

This article provides a definition of symmetric polynomials and the elementary symmetric polynomials $e_1,\ldots, e_n$ and proofs of their basic properties, including three notable ones:
\begin{itemize}
\item Vieta's formula, which gives an explicit expression for the $k$-th coefficient of a univariate monic polynomial in terms of its roots $x_1,\ldots,x_n$, namely $c_k = (-1)^{n-k}e_{n-k}(x_1,\ldots,x_n)$.
\item Second, the Fundamental Theorem of Symmetric Polynomials, which states that any symmetric polynomial is itself a uniquely determined polynomial combination of the elementary symmetric polynomials.
\item Third, as a corollary of the previous two, that given a polynomial over some ring $R$, any symmetric polynomial combination of its roots is also in $R$ even when the roots are not.
\end{itemize}
Both the symmetry property itself and the witness for the Fundamental Theorem are executable.
\end{abstract}

\newpage
\tableofcontents
\newpage
\parindent 0pt\parskip 0.5ex

\input{session}

\nocite{blum_smith_coskey13}
\bibliographystyle{abbrv}
\bibliography{root}

\end{document}

%%% Local Variables:
%%% mode: latex
%%% TeX-master: t
%%% End:
