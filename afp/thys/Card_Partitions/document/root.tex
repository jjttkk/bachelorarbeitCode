\documentclass[11pt,a4paper]{article}
\usepackage[T1]{fontenc}
\usepackage{isabelle,isabellesym}

% this should be the last package used
\usepackage{pdfsetup}

% urls in roman style, theory text in math-similar italics
\urlstyle{rm}
\isabellestyle{it}


\begin{document}

\title{Cardinality of Set Partitions}
\author{Lukas Bulwahn}
\maketitle

\begin{abstract}

The theory's main theorem states that the cardinality of set partitions of
size $k$ on a carrier set of size $n$ is expressed by Stirling numbers of the
second kind.
In Isabelle, Stirling numbers of the second kind are defined in the AFP entry
`Discrete Summation'~\cite{Discrete_Summation-AFP} through their well-known
recurrence relation. The main theorem relates them to the alternative
definition as cardinality of set partitions.
The proof follows the simple and short explanation in Richard P. Stanley's
`Enumerative Combinatorics: Volume 1'~\cite{Stanley-2012} and
Wikipedia~\cite{Wikipedia-Stirling-Numbers-2015}, and unravels the full
details and implicit reasoning steps of these explanations.

\end{abstract}

\tableofcontents

% sane default for proof documents
\parindent 0pt\parskip 0.5ex

% generated text of all theories
\input{session}

\bibliographystyle{abbrv}
\bibliography{root}

\end{document}

%%% Local Variables:
%%% mode: latex
%%% TeX-master: t
%%% End:
