\documentclass[11pt,a4paper]{article}
\usepackage{isabelle,isabellesym}

% further packages required for unusual symbols (see also
% isabellesym.sty), use only when needed

%\usepackage{amssymb}
  %for \<leadsto>, \<box>, \<diamond>, \<sqsupset>, \<mho>, \<Join>,
  %\<lhd>, \<lesssim>, \<greatersim>, \<lessapprox>, \<greaterapprox>,
  %\<triangleq>, \<yen>, \<lozenge>

%\usepackage{eurosym}
  %for \<euro>

%\usepackage[only,bigsqcap]{stmaryrd}
  %for \<Sqinter>

%\usepackage{eufrak}
  %for \<AA> ... \<ZZ>, \<aa> ... \<zz> (also included in amssymb)

%\usepackage{textcomp}
  %for \<onequarter>, \<onehalf>, \<threequarters>, \<degree>, \<cent>,
  %\<currency>

% this should be the last package used
\usepackage{pdfsetup}

% urls in roman style, theory text in math-similar italics
\urlstyle{rm}
\isabellestyle{it}

% for uniform font size
%\renewcommand{\isastyle}{\isastyleminor}


\begin{document}
\title{The Hales--Jewett Theorem}
\author{Ujkan Sulejmani, Manuel Eberl, Katharina Kreuzer}
\maketitle

\begin{abstract}
    This article is a formalisation of a proof of the Hales--Jewett theorem presented in the textbook \emph{Ramsey Theory} by Graham et al.~\cite{thebook}.
    
    The Hales--Jewett theorem is a result in Ramsey Theory which states that, for any non-negative integers $r$ and $t$, there exists a minimal dimension $N$, such that any $r$-coloured $N'$-dimensional cube over $t$ elements (with $N' \geq N$) contains a monochromatic line. This theorem generalises Van der Waerden's Theorem, which has already been formalised in another AFP entry~\cite{vdw}.
\end{abstract}


\newpage
\tableofcontents

% sane default for proof documents
\newpage
\parindent 0pt\parskip 0.5ex

% generated text of all theories
\input{session}

% optional bibliography
\nocite{vdw}
\bibliographystyle{abbrv}
\bibliography{root}

\end{document}

%%% Local Variables:
%%% mode: latex
%%% TeX-master: t
%%% End:
