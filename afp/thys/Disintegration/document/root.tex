\documentclass[11pt,a4paper]{article}
\usepackage[T1]{fontenc}
\usepackage{isabelle,isabellesym}

% further packages required for unusual symbols (see also
% isabellesym.sty), use only when needed

\usepackage{amssymb}
  %for \<leadsto>, \<box>, \<diamond>, \<sqsupset>, \<mho>, \<Join>,
  %\<lhd>, \<lesssim>, \<greatersim>, \<lessapprox>, \<greaterapprox>,
  %\<triangleq>, \<yen>, \<lozenge>
\usepackage{amsmath}

%\usepackage{eurosym}
  %for \<euro>

\usepackage[only,bigsqcap]{stmaryrd}
%for \<Sqinter>

%\usepackage{eufrak}
  %for \<AA> ... \<ZZ>, \<aa> ... \<zz> (also included in amssymb)

%\usepackage{textcomp}
  %for \<onequarter>, \<onehalf>, \<threequarters>, \<degree>, \<cent>,
  %\<currency>

% this should be the last package used
\usepackage{pdfsetup}

% urls in roman style, theory text in math-similar italics
\urlstyle{rm}
\isabellestyle{it}


% for uniform font size
%\renewcommand{\isastyle}{\isastyleminor}


\begin{document}

\title{Disintegration Theorem}
\author{Michikazu Hirata}
\maketitle
\begin{abstract}
  We formalize mixture and disintegraion of measures.
  This entry is a formalization of Chapter 14.D of the book by Baccelli et.al.~\cite{baccelli:hal-02460214}.
  The main result is the disintegration theorem:
  let $(X,\Sigma_X)$ be a measurable space, $(Y,\Sigma_Y)$ be a standard Borel space, $\nu$ be a $\sigma$-finite measure on $X \times Y$,
  and $\nu_X$ be the marginal measure on $X$ defined by $\nu_X(A) = \nu(A\times Y)$.
  Assume that $\nu_X$ is $\sigma$-finite, then there exists a probability kernel $\kappa$
  from $X$ to $Y$ such that
  \[ \nu(A\times B) = \int_A \kappa_x (B) \nu_X(\mathrm{d}x), \: A\in\Sigma_X, B\in\Sigma_Y.\]
  Such a probability kernel is unique $\nu_X$-almost everywhere.
\end{abstract}

\tableofcontents

% sane default for proof documents
\parindent 0pt\parskip 0.5ex

% generated text of all theories
\input{session}

% optional bibliography
\bibliographystyle{abbrv}
\bibliography{root}

\end{document}

%%% Local Variables:
%%% mode: latex
%%% TeX-master: t
%%% End:
