\documentclass[11pt,a4paper]{article}
\usepackage[T1]{fontenc}
\usepackage{isabelle,isabellesym}

% further packages required for unusual symbols (see also
% isabellesym.sty), use only when needed

%\usepackage{amssymb}
  %for \<leadsto>, \<box>, \<diamond>, \<sqsupset>, \<mho>, \<Join>,
  %\<lhd>, \<lesssim>, \<greatersim>, \<lessapprox>, \<greaterapprox>,
  %\<triangleq>, \<yen>, \<lozenge>

%\usepackage{eurosym}
  %for \<euro>

%\usepackage[only,bigsqcap]{stmaryrd}
  %for \<Sqinter>

%\usepackage{eufrak}
  %for \<AA> ... \<ZZ>, \<aa> ... \<zz> (also included in amssymb)

%\usepackage{textcomp}
  %for \<onequarter>, \<onehalf>, \<threequarters>, \<degree>, \<cent>,
  %\<currency>

% this should be the last package used
\usepackage{pdfsetup}

% urls in roman style, theory text in math-similar italics
\urlstyle{rm}
\isabellestyle{it}

% for uniform font size
%\renewcommand{\isastyle}{\isastyleminor}


\begin{document}

\title{A formal model for the SPARCv8 ISA and a proof of non-interference for the LEON3 processor}
\author{Zh\'e H\'ou, David San\'an, Alwen Tiu and Yang Liu}
\maketitle

\begin{abstract}
We formalise the SPARCv8 instruction set architecture (ISA) which is
used in processors such as LEON3. Our formalisation can be specialised
to any SPARCv8 CPU, here we use LEON3 as a running example. Our model
covers the operational semantics for all the instructions in the
integer unit of the SPARCv8 architecture and it supports Isabelle code
export, which effectively turns the Isabelle model into a SPARCv8 CPU
simulator. We prove the language-based non-interference property for
the LEON3 processor.

Our model is based on deterministic monad, which is a modified version
of the non-deterministic monad from NICTA/l4v. We also use the Word
library developed by Jeremy Dawson and Gerwin Klein.
\end{abstract}

\tableofcontents

% sane default for proof documents
\parindent 0pt\parskip 0.5ex

% generated text of all theories
\input{session}

% optional bibliography
%\bibliographystyle{abbrv}
%\bibliography{root}

\end{document}

%%% Local Variables:
%%% mode: latex
%%% TeX-master: t
%%% End:
