\documentclass[11pt,DIV10,a4paper,twoside=semi,openright,titlepage]{scrreprt}
\usepackage[T1]{fontenc}
\usepackage{fixltx2e}
%%%%%%%%%%%%%%%%%%%%%%%%%%%%%%%%%%%%%%%%%%%%%%%%%%%%%%%%%%%%%%%%%%%%%%%
%%% Overrides the (rightfully issued) warning by Koma Script that \rm
%%% etc. should not be used (they are deprecated since more than a
%%% decade)
  \DeclareOldFontCommand{\rm}{\normalfont\rmfamily}{\mathrm}
  \DeclareOldFontCommand{\sf}{\normalfont\sffamily}{\mathsf}
  \DeclareOldFontCommand{\tt}{\normalfont\ttfamily}{\mathtt}
  \DeclareOldFontCommand{\bf}{\normalfont\bfseries}{\mathbf}
  \DeclareOldFontCommand{\it}{\normalfont\itshape}{\mathit}
%%%%%%%%%%%%%%%%%%%%%%%%%%%%%%%%%%%%%%%%%%%%%%%%%%%%%%%%%%%%%%%%%%%%%%%

\usepackage{isabelle,isabellesym}
\usepackage{stmaryrd}
\usepackage{paralist}
\usepackage{xspace}
\newcommand{\testgen}{HOL-TestGen\xspace}
\newcommand{\testgenFW}{HOL-TestGen/FW\xspace}
\usepackage[numbers, sort&compress, sectionbib]{natbib}
\usepackage{graphicx}
\usepackage{color}
\sloppy

\usepackage{amssymb}



\newcommand{\isasymmodels}{\isamath{\models}}
\newcommand{\HOL}{HOL}

\newcommand{\ie}{i.\,e.}
\newcommand{\eg}{e.\,g.}

\usepackage{pdfsetup}

\urlstyle{rm}
\isabellestyle{it}
\renewcommand{\isastyle}{\isastyleminor}

\pagestyle{empty} 
\begin{document}
\renewcommand{\subsubsectionautorefname}{Section}
\renewcommand{\subsectionautorefname}{Section}
\renewcommand{\sectionautorefname}{Section}
\renewcommand{\chapterautorefname}{Chapter}
\newcommand{\subtableautorefname}{\tableautorefname}
\newcommand{\subfigureautorefname}{\figureautorefname}

\title{The Unified Policy Framework\\
  (UPF)}
\author{Achim D. Brucker\footnotemark[1] \quad
        Lukas Br\"ugger\footnotemark[2]  \quad
        Burkhart Wolff\footnotemark[3]\\[1.5em]
  \normalsize
  \normalsize\footnotemark[1]~SAP SE, Vincenz-Priessnitz-Str. 1, 76131 Karlsruhe,
  Germany \texorpdfstring{\\}{}
  \normalsize\href{mailto:"Achim D. Brucker"
    <achim.brucker@sap.com>}{achim.brucker@sap.com}\\[1em]
  %
  \normalsize\footnotemark[2]Information Security, ETH Zurich, 8092 Zurich, Switzerland
  \texorpdfstring{\\}{}
  \normalsize\href{mailto:"Lukas Bruegger"
    <lukas.a.bruegger@gmail.com>}{Lukas.A.Bruegger@gmail.com}\\[1em]
  %
  \normalsize\footnotemark[3]~Univ. Paris-Sud, Laboratoire LRI,
  UMR8623, 91405 Orsay, France
  France\texorpdfstring{\\}{}
  \normalsize\href{mailto:"Burkhart Wolff" <burkhart.wolff@lri.fr>}{burkhart.wolff@lri.fr}
}

\pagestyle{empty}
\publishers{%
  \normalfont\normalsize%
    \centerline{\textsf{\textbf{\large Abstract}}}
    \vspace{1ex}%
    \parbox{0.8\linewidth}{%
      We present the \emph{Unified Policy Framework}
      (UPF), a generic framework for modelling security
      (access-control) policies; in Isabelle/\HOL. 
      %\cite{}.
      UPF emphasizes the view that a policy is a policy decision
      function that grants or denies access to resources, permissions,
      etc. In other words, instead of modelling the
      relations of permitted or prohibited requests directly, we model
      the concrete function that implements the policy decision point
      in a system, seen as an ``aspect'' of ``wrapper'' around the 
      business logic  % Fachlogik 
      of a system.
      In more detail, UPF is based on the following four  principles:
      \begin{inparaenum}
       \item Functional representation of policies,
       \item No conflicts are possible,
       \item Three-valued decision type (allow, deny, undefined),
       \item Output type not containing the decision only.
      \end{inparaenum}
   }
}

\maketitle
\cleardoublepage
\pagestyle{plain}
\tableofcontents
\cleardoublepage

\section{Motivation}
Because of its connected life, the modern world is increasingly
depending on secure implementations and configurations of network
infrastructures. As building blocks of the latter, firewalls are
playing a central role in ensuring the overall \emph{security} of
networked applications.

Firewalls, routers applying network-address-translation (NAT) and
similar networking systems suffer from the same quality problems as
other complex software. Jennifer Rexford mentioned in her keynote at
POPL 2012 that high-end firewalls consist of more than 20
million lines of code comprising components written in Ada as well as
LISP. However, the testing techniques discussed here are of
wider interest to all network infrastructure operators that need to
ensure the security and reliability of their infrastructures across
system changes such as system upgrades or hardware replacements. This
is because firewalls and routers are active network elements that can
filter and rewrite network traffic based on configurable rules. The
\emph{configuration} by appropriate rule sets implements a security
policy or links networks together.

Thus, it is, firstly, important to test both the implementation of a
firewall and, secondly, the correct configuration for each use. To
address this problem, we model firewall policies formally in
Isabelle/HOL. This formalization is based on the Unified Policy
Framework (UPF)~\cite{brucker.ea:upf:2014}. This formalization allows
to express access control policies on the network level using a
combinator-based language that is close to textbook-style
specifications of firewall rules. To actually test the implementation
as well as the configuration of a firewall, we use
HOL-TestGen~\cite{brucker.ea:interactive:2005,brucker.ea:hol-testgen-fw:2013,brucker.ea:theorem-prover:2012}
to generate test cases that can be used to validate the compliance of
real network middleboxes (e.g., firewalls, routers). In this document,
we focus on the Isabelle formalization of network access control
policies. For details of the overall approach, we refer the reader
elsewhere~\cite{brucker.ea:formal-fw-testing:2014}

\section{The Unified Policy Framework (UPF)}
Our formalization of firewall policies is based on the Unified Policy
Framework (UPF. In this section, we briefly introduce UPF, for all
details we refer the reader to)~\cite{brucker.ea:upf:2014}.

UPF is a generic framework for policy modeling with the primary goal
of being used for test case generation. The interested reader is
referred to~\cite{brucker.ea:model-based:2011} for an application of
UPF to large scale access control policies in the health care domain;
a comprehensive treatment is also contained in the reference manual
coming with the distribution on the \testgen website
(\url{http://www.brucker.ch/projects/hol-testgen/}).  UPF is based on
the following four principles:
\begin{enumerate}
\item policies are represented as \emph{functions} (rather than relations),
\item policy combination avoids conflicts by construction,
\item the decision type is three-valued (allow, deny, undefined),
\item the output type does not only contain the decision but also a
  `slot' for arbitrary result data.
\end{enumerate}


Formally, the concept of a policy is specified as a partial function
from some input to a decision value and additional some
output. \emph{Partial} functions are used because elementary policies
are described by partial system behavior, which are glued together by
operators such as function override and functional composition.
\begin{gather*}
  \types\enspace \alpha \mapsto \beta =
  \alpha \rightharpoonup \beta \operatorname{decision}
\end{gather*}
where the enumeration type $\operatorname{decision}$ is
\begin{gather*}
  \datatype \qquad \alpha\ap \operatorname{decision}
  = \operatorname{allow}\ap \alpha \mid
  \operatorname{deny}\ap \alpha
\end{gather*}

As policies are partial functions or `maps', the notions of a
\emph{domain} $\dom \ap p \ofType \alpha \rightharpoonup \beta
\Rightarrow \HolSet{\alpha}$ and a \emph{range} $\ran\ap p \ofType
[\alpha \rightharpoonup \beta] \Rightarrow \HolSet{\beta}$ can be
inherited from the Isabelle library.

Inspired by the Z notation~\cite{spivey:z-notation:1992}, there is the
concept of \emph{domain restriction} $\_ \triangleleft \_$ and
\emph{range restriction} $\_ \triangleright \_$, defined as:
\begin{gather*}
  \begin{array}{lrl}
    \isadefinition &\_ \triangleleft \_&\ofType \HolSet{\alpha}
           \Rightarrow \alpha \mapsto \beta \Rightarrow \alpha \mapsto \beta\\
    \where & S \triangleleft p &= \lambda x\spot \HolIf x \in S \HolThen p\ap x
    \HolElse \bottom \\[.5ex]
    \isadefinition & \_ \triangleright \_ &\ofType \alpha \mapsto
    \beta \Rightarrow \HolSet{\beta\ap\operatorname{decision}}
    \Rightarrow \alpha \mapsto \beta \\ 
    \where & p \triangleright S &= \lambda x\spot \HolIf
    \bigl(\operatorname{the}\ap
    (p\ap x)\bigr) \in S \HolThen p\ap x \HolElse \bottom \\
  \end{array}
\end{gather*}
The operator `$\operatorname{the}$' strips off the $\HolSome$, if it
exists. Otherwise the range restriction is underspecified.

There are many operators that change the result of applying the
policy to a particular element. The essential one is the
\emph{update}:
\begin{gather*}
  p(x \mapsto t) = \lambda y\spot \HolIf y = x \HolThen \lfloor
  t\rfloor \HolElse p\ap y
\end{gather*}

Next, there are three categories of elementary policies in UPF,
relating to the three possible decision values:
\begin{itemize}
\item The empty policy; undefined for all elements: $\emptyset =
  \lambda x\spot \bottom$
\item A policy allowing everything, written as $A_f\ap f$, or $A_U$ if
  the additional output is unit (defined as $\lambda x\spot \lfloor
  \operatorname{allow} ()\rfloor$).
\item A policy denying everything, written as $D_f\ap f$, or $D_U$ if
  the additional output is unit.
\end{itemize}

The most often used approach to define individual rules is to define a
rule as a refinement of one of the elementary policies, by using a
domain restriction. As an example,
\begin{gather*}
\bigl\{(\operatorname{Alice}, \operatorname{obj1},
\operatorname{read})\bigr\} \triangleleft A_U
\end{gather*}

Finally, rules can be combined to policies in three different ways:
\begin{itemize}
\item Override operators: used for policies of the same type, written
  as $\_ \oplus_i \_$.
\item Parallel combination operators: used for the parallel
  composition of policies of potentially different type, written as
  $\_ \otimes_i \_$.
\item Sequential combination operators: used for the sequential
  composition of policies of potentially different type, written as
  $\_ \circ_i \_$.
\end{itemize}

All three combinators exist in four variants, depending on how the
decisions of the constituent policies are to be combined. For example,
the $\_ \prodTwo \_$ operator is the parallel combination operator where the
decision of the second policy is used.

Several interesting algebraic properties are proved for UPF
operators. For example, distributivity
\begin{gather*}
( P_1 \oplus  P_2)  \otimes P_3  = ( P_1  \otimes P_3) \oplus ( P_2 \otimes P_3)
\end{gather*}
Other UPF concepts are introduced in this paper on-the-fly when
needed.


%%%%%%%%%%%%%%%%%%%%%%%%%%%%%%
% <session>
  % \input{session}
  \chapter{The Unified Policy Framework (UPF)}
  \input{UPFCore}
  \input{ElementaryPolicies}
  \input{SeqComposition}
  \input{ParallelComposition}
  \input{Analysis}
  \input{Normalisation}
  \input{NormalisationTestSpecification}
  \input{UPF}
  \chapter{Example}
  In this chapter, we present a small example application of UPF for
modeling access control for a Web service that might be used in a
hospital. This scenario is motivated by our formalization of the NHS
system~\cite{bruegger:generation:2012,brucker.ea:model-based:2011}. 

UPF was also successfully used for modeling network security policies
such as the ones enforced by
firewalls~\cite{bruegger:generation:2012,brucker.ea:formal-fw-testing:2014}. These
models were also used for generating test cases using
HOL-TestGen~\cite{brucker.ea:theorem-prover:2012}.

  \input{Service}
  \input{ServiceExample}
% </session>
%%%%%%%%%%%%%%%%%%%%%%%%%%%%%%
\section{Conclusion}\label{sec:concl}
  We have presented a verification of two variants of Gabow's algorithm: Computation of the strongly connected components of
  a graph, and emptiness check of a generalized B\"uchi automaton. We have extracted efficient code with a performance comparable to a
  reference implementation in Java.
  
  We have modularized the formalization in two directions: First, we share most of the proofs between the two variants of the algorithm. Second,
  we use a stepwise refinement approach to separate the algorithmic ideas and the correctness proof from implementation details.
  Sharing of the proofs reduced the overall effort of developing both algorithms. Using a stepwise refinement approach allowed us to
  formalize an efficient implementation, without making the correctness proof complex and unmanageable by cluttering it with implementation details.

  Our development approach is independent of Gabow's algorithm, and can be re-used for the verification of other algorithms.

  \paragraph{Current and Future Work} 
  An important direction of future work is to fine-tune the implementation of 
  the emptiness check algorithm for speed, as speed of the checking algorithm
  directely influences the performance of the modelchecker.

  
\chapter{Appendix}
\input{Monads}

%%% Local Variables:
%%% mode: latex
%%% TeX-master: "root"
%%% End:


\nocite{brucker.ea:formal-fw-testing:2014,brucker.ea:hol-testgen-fw:2013,brucker.ea:theorem-prover:2012,brucker.ea:model-based:2011}
\nocite{bruegger:generation:2012}
\bibliographystyle{abbrvnat}
\bibliography{root}


\end{document}

%%% Local Variables:
%%% mode: latex
%%% TeX-master: t
%%% End:
