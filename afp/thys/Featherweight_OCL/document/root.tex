\documentclass[fontsize=10pt,DIV12,paper=a4,open=right,twoside,abstract=true]{scrreprt}
\usepackage{fixltx2e}
\usepackage[T1]{fontenc}
\usepackage{textcomp}
\usepackage[english]{babel}
\usepackage{isabelle}
\isatagannexa
  \usepackage{omg}
  \usepackage{draftwatermark}
  \SetWatermarkAngle{55}
  \SetWatermarkLightness{.9}
  \SetWatermarkFontSize{3cm}
  \SetWatermarkScale{1.4}
  \SetWatermarkText{\textbf{\textsf{Draft Proposal}}}
\endisatagannexa
\usepackage[nocolortable, noaclist,isasymonly,nocolor]{hol-ocl-isar}
\renewcommand{\lfloor}{\isasymHolOclLiftLeft}
\renewcommand{\rfloor}{\isasymHolOclLiftRight}
\renewcommand{\lceil}{\isasymHolOclDropLeft}
\renewcommand{\rceil}{\isasymHolOclDropRight}
\renewcommand{\oclkeywordstyle}{\bfseries}
\renewcommand{\javakeywordstyle}{\bfseries}
\renewcommand{\smlkeywordstyle}{\bfseries}
\renewcommand{\holoclthykeywordstyle}{}

\usepackage{lstisar}
\usepackage{railsetup}
\usepackage[]{mathtools}
\usepackage{%
  multirow,
  paralist,
  booktabs,     %       "        "      "
  threeparttable,
  longtable,    % Mehrseitige Tabellen 
}



\usepackage{graphicx}
\usepackage[numbers, sort&compress, sectionbib]{natbib}
\usepackage{chapterbib}
\usepackage[caption=false]{subfig}
\usepackage{tabu}
\usepackage{prooftree}
%\usepackage[draft]{fixme}
\usepackage[pdfpagelabels, pageanchor=false, bookmarksnumbered, plainpages=false]{hyperref}
\graphicspath{{data/},{figures/}}
\makeatletter
\renewcommand*\l@section{\bprot@dottedtocline{1}{1.5em}{2.8em}}
\renewcommand*\l@subsection{\bprot@dottedtocline{2}{3.8em}{3.7em}}
\renewcommand*\l@subsubsection{\bprot@dottedtocline{3}{7.0em}{5em}}
\renewcommand*\l@paragraph{\bprot@dottedtocline{4}{10em}{6.2em}}
%\renewcommand*\l@paragraph{\bprot@dottedtocline{4}{10em}{5.5em}}
\renewcommand*\l@subparagraph{\bprot@dottedtocline{5}{12em}{7.7em}}
%\renewcommand*\l@subparagraph{\bprot@dottedtocline{5}{12em}{6.5em}}
\makeatother
%%%%%%%%%%%%%%%%%%%%%%%%%%%%%%%%%%%%%%%%%%%%%%%%%%%%%%%%%%%%%%%%%%%%%
%%% Overall the (rightfully issued) warning by Koma Script that \rm
%%% etc. should not be used (they are deprecated since more than a
%%% decade)
  \DeclareOldFontCommand{\rm}{\normalfont\rmfamily}{\mathrm}
  \DeclareOldFontCommand{\sf}{\normalfont\sffamily}{\mathsf}
  \DeclareOldFontCommand{\tt}{\normalfont\ttfamily}{\mathtt}
  \DeclareOldFontCommand{\bf}{\normalfont\bfseries}{\mathbf}
  \DeclareOldFontCommand{\it}{\normalfont\itshape}{\mathit}
%%%%%%%%%%%%%%%%%%%%%%%%%%%%%%%%%%%%%%%%%%%%%%%%%%%%%%%%%%%%%%%%%%%%%

\setcounter{tocdepth}{3} % printed TOC not too detailed
\hypersetup{bookmarksdepth=3} % more detailed digital TOC (aka bookmarks)
\sloppy
\allowdisplaybreaks[4]
\raggedbottom

\newcommand{\HOL}{HOL\xspace}
\newcommand{\OCL}{OCL\xspace}
\newcommand{\UML}{UML\xspace}
\newcommand{\HOLOCL}{HOL-OCL\xspace}
\newcommand{\FOCL}{Featherweight OCL\xspace}
\renewcommand{\HolTrue}{\mathrm{true}}
\renewcommand{\HolFalse}{\mathrm{false}}
\newcommand{\ptmi}[1]{\using{\mi{#1}}}
\newcommand{\Lemma}[1]{{\color{BrickRed}%
    \mathbf{\operatorname{lemma}}}~\text{#1:}\quad}
\newcommand{\done}{{\color{OliveGreen}\operatorname{done}}}
\newcommand{\apply}[1]{{\holoclthykeywordstyle%
    \operatorname{apply}}(\text{#1})}
\newcommand{\fun} {{\holoclthykeywordstyle\operatorname{fun}}}
\newcommand{\isardef} {{\holoclthykeywordstyle\operatorname{definition}}}
\newcommand{\where} {{\holoclthykeywordstyle\operatorname{where}}}
\newcommand{\datatype} {{\holoclthykeywordstyle\operatorname{datatype}}}
\newcommand{\types} {{\holoclthykeywordstyle\operatorname{types}}}
\newcommand{\pglabel}[1]{\text{#1}}
\renewcommand{\isasymOclUndefined}{\ensuremath{\mathtt{invalid}}}
\newcommand{\isasymOclNull}{\ensuremath{\mathtt{null}}}
\newcommand{\isasymOclInvalid}{\isasymOclUndefined}
\DeclareMathOperator{\inv}{inv}
\newcommand{\Null}[1]{{\ensuremath{\mathtt{null}_\text{{#1}}}}}
\newcommand{\testgen}{HOL-TestGen\xspace}
\newcommand{\HolOption}{\mathrm{option}}
\newcommand{\ran}{\mathrm{ran}}
\newcommand{\dom}{\mathrm{dom}}
\newcommand{\typedef}{\mathrm{typedef}}
\newcommand{\typesynonym}{\mathrm{type\_synonym}}
\newcommand{\mi}[1]{\,\text{#1}}
\newcommand{\state}[1]{\ifthenelse{\equal{}{#1}}%
  {\operatorname{state}}%
  {\operatorname{\mathit{state}}(#1)}%
}
\newcommand{\mocl}[1]{\text{\inlineocl|#1|}}
\DeclareMathOperator{\TCnull}{null}
\DeclareMathOperator{\HolNull}{null}
\DeclareMathOperator{\HolBot}{bot}
\newcommand{\isaAA}{\mathfrak{A}}

% urls in roman style, theory text in math-similar italics
\urlstyle{rm}
\isabellestyle{it}
\newcommand{\ie}{i.\,e.\xspace}
\newcommand{\eg}{e.\,g.\xspace}

\newenvironment{isamarkuplazy_text}{\par \isacommand{lazy{\isacharunderscore}text}\isamarkupfalse\isacartoucheopen\isastyletext\begin{isapar}}{\end{isapar}\isacartoucheclose}
\renewcommand{\isasymguillemotleft}{\isatext{\textquotedblleft}}
\renewcommand{\isasymguillemotright}{\isatext{\textquotedblright}}
\begin{document}
\renewcommand{\subsubsectionautorefname}{Section}
\renewcommand{\subsectionautorefname}{Section}
\renewcommand{\sectionautorefname}{Section}
\renewcommand{\chapterautorefname}{Chapter}
\newcommand{\subtableautorefname}{\tableautorefname}
\newcommand{\subfigureautorefname}{\figureautorefname}
\isatagannexa
\renewcommand\thepart{\Alph{part}}
\renewcommand\partname{Annex}
\endisatagannexa

\newenvironment{matharray}[1]{\[\begin{array}{#1}}{\end{array}\]} % from 'iman.sty'
\newcommand{\indexdef}[3]%
{\ifthenelse{\equal{}{#1}}{\index{#3 (#2)|bold}}{\index{#3 (#1\ #2)|bold}}} % from 'isar.sty'



\isatagafp
  \title{Featherweight OCL}
  \subtitle{A Proposal for a Machine-Checked Formal Semantics for OCL 2.5 %\\
    %\includegraphics[scale=.5]{figures/logo_focl}
  }
\endisatagafp
\isatagannexa
  \title{A Formal Machine-Checked Semantics for OCL 2.5}
  \subtitle{A Proposal for the "Annex A" of the OCL Standard}
\endisatagannexa
\author{%
  \href{http://www.brucker.ch/}{Achim D. Brucker}\footnotemark[1]
  \and
  \href{https://www.lri.fr/~tuong/}{Fr\'ed\'eric Tuong}\footnotemark[2]~\footnotemark[3]
  \and
  \href{https://www.lri.fr/~wolff/}{Burkhart Wolff}\footnotemark[2]~\footnotemark[3]}
\publishers{%
  \footnotemark[1]~SAP SE\\
  Vincenz-Priessnitz-Str. 1, 76131 Karlsruhe,
  Germany \texorpdfstring{\\}{} \href{mailto:"Achim D. Brucker"
    <achim.brucker@sap.com>}{achim.brucker@sap.com}\\[2em]
  %
  \footnotemark[2]~LRI, Univ. Paris-Sud, CNRS, CentraleSup\'elec, Universit\'e Paris-Saclay \\
  b\^at. 650 Ada Lovelace, 91405 Orsay, France \texorpdfstring{\\}{}
    \href{mailto:"Frederic Tuong"
    <frederic.tuong@lri.fr>}{frederic.tuong@lri.fr} \hspace{4.5em}
    \href{mailto:"Burkhart Wolff"
    <burkhart.wolff@lri.fr>}{burkhart.wolff@lri.fr} \\[2em]
  %
  \footnotemark[3]~IRT SystemX\\
  8 av.~de la Vauve, 91120 Palaiseau, France \texorpdfstring{\\}{}
    \href{mailto:"Frederic Tuong"
    <frederic.tuong@irt-systemx.fr>}{frederic.tuong@irt-systemx.fr} \quad
    \href{mailto:"Burkhart Wolff"
    <burkhart.wolff@irt-systemx.fr>}{burkhart.wolff@irt-systemx.fr}
}


\maketitle
\isatagannexa
\cleardoublepage
\endisatagannexa

\isatagafp
  \begin{abstract}
    The Unified Modeling Language (UML) is one of the few modeling
    languages that is widely used in industry. While UML is mostly known
    as diagrammatic modeling language (\eg, visualizing class models),
    it is complemented by a textual language, called Object Constraint
    Language (OCL). OCL is a textual annotation language, originally based on a
    three-valued logic, that turns UML into a formal language.
    Unfortunately the semantics of this specification language, captured
    in the ``Annex A'' of the OCL standard, leads to different
    interpretations of corner cases.  Many of these corner cases had
    been subject to formal analysis since more than ten years.

    The situation complicated with the arrival of version 2.3 of the OCL 
    standard. OCL was aligned with the latest version of UML: this led to the 
    extension of the three-valued logic by a second exception element, called
    \inlineocl{null}.  While the first exception element
    \inlineocl{invalid} has a strict semantics, \inlineocl{null} has a
    non strict interpretation. The combination of these semantic features lead
    to remarkable confusion for implementors of OCL compilers and
    interpreters.

    In this paper, we provide a formalization of the core of OCL in
    HOL\@. It provides denotational definitions, a logical calculus and
    operational rules that allow for the execution of OCL expressions by
    a mixture of term rewriting and code compilation. Moreover, we describe
    a coding-scheme for UML class models that were annotated by 
    code-invariants and code contracts. An implementation of this coding-scheme
    has been undertaken: it consists of a kind of compiler that takes a UML class
    model and translates it into a family of definitions and derived
    theorems over them capturing the properties of constructors and selectors,
    tests and casts resulting from the class model. However, this compiler
    is \emph{not} included in this document.

    Our formalization reveals several inconsistencies and contradictions
    in the current version of the OCL standard.  They reflect a challenge
    to define and implement OCL tools in a uniform manner.  Overall, this
    document is intended to provide the basis for a machine-checked text
    ``Annex A'' of the OCL standard targeting at tool implementors.
  \end{abstract}
  \tableofcontents
\endisatagafp

\part{Formal Semantics of OCL}
\section{Motivation}
Because of its connected life, the modern world is increasingly
depending on secure implementations and configurations of network
infrastructures. As building blocks of the latter, firewalls are
playing a central role in ensuring the overall \emph{security} of
networked applications.

Firewalls, routers applying network-address-translation (NAT) and
similar networking systems suffer from the same quality problems as
other complex software. Jennifer Rexford mentioned in her keynote at
POPL 2012 that high-end firewalls consist of more than 20
million lines of code comprising components written in Ada as well as
LISP. However, the testing techniques discussed here are of
wider interest to all network infrastructure operators that need to
ensure the security and reliability of their infrastructures across
system changes such as system upgrades or hardware replacements. This
is because firewalls and routers are active network elements that can
filter and rewrite network traffic based on configurable rules. The
\emph{configuration} by appropriate rule sets implements a security
policy or links networks together.

Thus, it is, firstly, important to test both the implementation of a
firewall and, secondly, the correct configuration for each use. To
address this problem, we model firewall policies formally in
Isabelle/HOL. This formalization is based on the Unified Policy
Framework (UPF)~\cite{brucker.ea:upf:2014}. This formalization allows
to express access control policies on the network level using a
combinator-based language that is close to textbook-style
specifications of firewall rules. To actually test the implementation
as well as the configuration of a firewall, we use
HOL-TestGen~\cite{brucker.ea:interactive:2005,brucker.ea:hol-testgen-fw:2013,brucker.ea:theorem-prover:2012}
to generate test cases that can be used to validate the compliance of
real network middleboxes (e.g., firewalls, routers). In this document,
we focus on the Isabelle formalization of network access control
policies. For details of the overall approach, we refer the reader
elsewhere~\cite{brucker.ea:formal-fw-testing:2014}

\section{The Unified Policy Framework (UPF)}
Our formalization of firewall policies is based on the Unified Policy
Framework (UPF. In this section, we briefly introduce UPF, for all
details we refer the reader to)~\cite{brucker.ea:upf:2014}.

UPF is a generic framework for policy modeling with the primary goal
of being used for test case generation. The interested reader is
referred to~\cite{brucker.ea:model-based:2011} for an application of
UPF to large scale access control policies in the health care domain;
a comprehensive treatment is also contained in the reference manual
coming with the distribution on the \testgen website
(\url{http://www.brucker.ch/projects/hol-testgen/}).  UPF is based on
the following four principles:
\begin{enumerate}
\item policies are represented as \emph{functions} (rather than relations),
\item policy combination avoids conflicts by construction,
\item the decision type is three-valued (allow, deny, undefined),
\item the output type does not only contain the decision but also a
  `slot' for arbitrary result data.
\end{enumerate}


Formally, the concept of a policy is specified as a partial function
from some input to a decision value and additional some
output. \emph{Partial} functions are used because elementary policies
are described by partial system behavior, which are glued together by
operators such as function override and functional composition.
\begin{gather*}
  \types\enspace \alpha \mapsto \beta =
  \alpha \rightharpoonup \beta \operatorname{decision}
\end{gather*}
where the enumeration type $\operatorname{decision}$ is
\begin{gather*}
  \datatype \qquad \alpha\ap \operatorname{decision}
  = \operatorname{allow}\ap \alpha \mid
  \operatorname{deny}\ap \alpha
\end{gather*}

As policies are partial functions or `maps', the notions of a
\emph{domain} $\dom \ap p \ofType \alpha \rightharpoonup \beta
\Rightarrow \HolSet{\alpha}$ and a \emph{range} $\ran\ap p \ofType
[\alpha \rightharpoonup \beta] \Rightarrow \HolSet{\beta}$ can be
inherited from the Isabelle library.

Inspired by the Z notation~\cite{spivey:z-notation:1992}, there is the
concept of \emph{domain restriction} $\_ \triangleleft \_$ and
\emph{range restriction} $\_ \triangleright \_$, defined as:
\begin{gather*}
  \begin{array}{lrl}
    \isadefinition &\_ \triangleleft \_&\ofType \HolSet{\alpha}
           \Rightarrow \alpha \mapsto \beta \Rightarrow \alpha \mapsto \beta\\
    \where & S \triangleleft p &= \lambda x\spot \HolIf x \in S \HolThen p\ap x
    \HolElse \bottom \\[.5ex]
    \isadefinition & \_ \triangleright \_ &\ofType \alpha \mapsto
    \beta \Rightarrow \HolSet{\beta\ap\operatorname{decision}}
    \Rightarrow \alpha \mapsto \beta \\ 
    \where & p \triangleright S &= \lambda x\spot \HolIf
    \bigl(\operatorname{the}\ap
    (p\ap x)\bigr) \in S \HolThen p\ap x \HolElse \bottom \\
  \end{array}
\end{gather*}
The operator `$\operatorname{the}$' strips off the $\HolSome$, if it
exists. Otherwise the range restriction is underspecified.

There are many operators that change the result of applying the
policy to a particular element. The essential one is the
\emph{update}:
\begin{gather*}
  p(x \mapsto t) = \lambda y\spot \HolIf y = x \HolThen \lfloor
  t\rfloor \HolElse p\ap y
\end{gather*}

Next, there are three categories of elementary policies in UPF,
relating to the three possible decision values:
\begin{itemize}
\item The empty policy; undefined for all elements: $\emptyset =
  \lambda x\spot \bottom$
\item A policy allowing everything, written as $A_f\ap f$, or $A_U$ if
  the additional output is unit (defined as $\lambda x\spot \lfloor
  \operatorname{allow} ()\rfloor$).
\item A policy denying everything, written as $D_f\ap f$, or $D_U$ if
  the additional output is unit.
\end{itemize}

The most often used approach to define individual rules is to define a
rule as a refinement of one of the elementary policies, by using a
domain restriction. As an example,
\begin{gather*}
\bigl\{(\operatorname{Alice}, \operatorname{obj1},
\operatorname{read})\bigr\} \triangleleft A_U
\end{gather*}

Finally, rules can be combined to policies in three different ways:
\begin{itemize}
\item Override operators: used for policies of the same type, written
  as $\_ \oplus_i \_$.
\item Parallel combination operators: used for the parallel
  composition of policies of potentially different type, written as
  $\_ \otimes_i \_$.
\item Sequential combination operators: used for the sequential
  composition of policies of potentially different type, written as
  $\_ \circ_i \_$.
\end{itemize}

All three combinators exist in four variants, depending on how the
decisions of the constituent policies are to be combined. For example,
the $\_ \prodTwo \_$ operator is the parallel combination operator where the
decision of the second policy is used.

Several interesting algebraic properties are proved for UPF
operators. For example, distributivity
\begin{gather*}
( P_1 \oplus  P_2)  \otimes P_3  = ( P_1  \otimes P_3) \oplus ( P_2 \otimes P_3)
\end{gather*}
Other UPF concepts are introduced in this paper on-the-fly when
needed.


%\clearpage
\isatagafp
\input{session}
\endisatagafp
\isatagannexa
\input{UML_Types.tex}
\input{UML_Logic.tex}
\input{UML_PropertyProfiles.tex}
\input{UML_Boolean.tex}
\input{UML_Void.tex}
\input{UML_Integer.tex}
\input{UML_Real.tex}
\input{UML_String.tex}
\input{UML_Pair.tex}
\input{UML_Bag.tex}
\input{UML_Set.tex}
\input{UML_Sequence.tex}
\input{UML_Library.tex}
\input{UML_State.tex}
\input{UML_Contracts.tex}
%\input{UML_Tools.tex}
%\input{UML_Main.tex}
% \input{Design_UML.tex}
% \input{Design_OCL.tex}
\input{Analysis_UML.tex}
\input{Analysis_OCL.tex}
\part{Bibliography}
\endisatagannexa
\isatagafp
\section{Conclusion}\label{sec:concl}
  We have presented a verification of two variants of Gabow's algorithm: Computation of the strongly connected components of
  a graph, and emptiness check of a generalized B\"uchi automaton. We have extracted efficient code with a performance comparable to a
  reference implementation in Java.
  
  We have modularized the formalization in two directions: First, we share most of the proofs between the two variants of the algorithm. Second,
  we use a stepwise refinement approach to separate the algorithmic ideas and the correctness proof from implementation details.
  Sharing of the proofs reduced the overall effort of developing both algorithms. Using a stepwise refinement approach allowed us to
  formalize an efficient implementation, without making the correctness proof complex and unmanageable by cluttering it with implementation details.

  Our development approach is independent of Gabow's algorithm, and can be re-used for the verification of other algorithms.

  \paragraph{Current and Future Work} 
  An important direction of future work is to fine-tune the implementation of 
  the emptiness check algorithm for speed, as speed of the checking algorithm
  directely influences the performance of the modelchecker.
 %no conclusion for standard document
\endisatagafp
\bibliographystyle{abbrvnat}
\bibliography{root}

\isatagafp
\appendix
\part{Appendix}
\endisatagafp
\isatagannexa
\part{The OCL And Featherweight OCL Syntax}
\endisatagannexa
\isatagafp
\chapter{The OCL And Featherweight OCL Syntax}
\endisatagafp
\newcommand{\simpleArgs}[1]{\_}
\newcommand{\hide}[1]{}
\newcommand{\hideT}[1]{}
\newcommand{\foclcolorbox}[2]{#2}
\newcommand{\isaFS}[1]{\isa{\footnotesize #1}}

{
\begin{longtable}[C]
{@{}%
c%
l%
l%
l% >{$}l<{$}%
@{}}
  \caption{Comparison of different concrete syntax variants for OCL \label{tab:comp-diff-syntax}}\\
  \toprule
&  OCL & Featherweight OCL  & Logical Constant \\
  \midrule
\endfirsthead
  \toprule
&  OCL & Featherweight OCL & Logical Constant \\
  \midrule
\endhead
  \midrule \multicolumn{3}{r}{\emph{Continued on next page}}
\endfoot
  \bottomrule
  \endlastfoot
  %%%%%%%%%%%%%%%%%%%%%%%%%%%%%%%%%%%%%%%%%%%%%%%%%%%%%%%%%%%%%%%%%%%%%%%
  %%%% 11.3.1 OclAny
  %%%%%%%%%%%%%%%%%%%%%%%%%%%%%%%%%%%%%%%%%%%%%%%%%%%%%%%%%%%%%%%%%%%%%%%
  \multirow{11}{*}{\rotatebox{90}{OclAny}}
  &\footnotesize\inlineocl"_ = _"
  & \hide{\color{Gray}($\text{\isaFS{logic}}^{\text{\color{GreenYellow}1000}}$)} \foclcolorbox{Apricot}{\isaFS{op}} \foclcolorbox{Apricot}{\isaFS{{\isasymtriangleq}}} & {{\isaFS{UML{\isacharunderscore}Logic{\isachardot}StrongEq}}\hideT{\text{\space\color{Black}\isaFS{const}}}}%
  \\
& \footnotesize\inlineocl"_ <> _"
& \hide{\color{Gray}($\text{\isaFS{logic}}^{\text{\color{GreenYellow}1000}}$)} \foclcolorbox{Apricot}{\isaFS{op}} \foclcolorbox{Apricot}{\isaFS{{\isacharless}{\isachargreater}}} & {{\color{Gray} \isaFS{notequal}}}%
  \\
&\footnotesize\inlineocl"_ ->oclAsSet( _ )"&&\\
&\footnotesize\inlineocl"_ .oclIsNew()"
& \hide{\color{Gray}($\text{\isaFS{logic}}^{\text{\color{GreenYellow}1000}}$)}\simpleArgs{$\text{\isaFS{logic}}^{\text{\color{GreenYellow}0}}$} \foclcolorbox{Apricot}{\isaFS{{\isachardot}oclIsNew{\isacharparenleft}{\isacharparenright}}} & {{ \isaFS{UML{\isacharunderscore}State{\isachardot}OclIsNew}}\hideT{\text{\space\color{Black}\isaFS{const}}}}%
\\
  %
&\footnotesize\inlineocl"not ( _ ->oclIsUndefined() )"
& \hide{\color{Gray}($\text{\isaFS{logic}}^{\text{\color{GreenYellow}100}}$)} \foclcolorbox{Apricot}{\isaFS{{\isasymdelta}}}\simpleArgs{$\text{\isaFS{logic}}^{\text{\color{GreenYellow}100}}$} & {{ \isaFS{UML{\isacharunderscore}Logic{\isachardot}defined}}\hideT{\text{\space\color{Black}\isaFS{const}}}}%
\\

%
  
&\footnotesize\inlineocl"not ( _ ->oclIsInvalid() )"
& \hide{\color{Gray}($\text{\isaFS{logic}}^{\text{\color{GreenYellow}100}}$)} \foclcolorbox{Apricot}{\isaFS{{\isasymupsilon}}}\simpleArgs{$\text{\isaFS{logic}}^{\text{\color{GreenYellow}100}}$} & {{ \isaFS{UML{\isacharunderscore}Logic{\isachardot}valid}}\hideT{\text{\space\color{Black}\isaFS{const}}}}%
\\
&\footnotesize\inlineocl"_ ->oclAsType( _ )"&&\\
&\footnotesize\inlineocl"_ ->oclIsTypeOf( _ )"&&\\
&\footnotesize\inlineocl"_ ->oclIsKindOf( _ )"&&\\
&\footnotesize\inlineocl"_ ->oclIsInState( _ )"&&\\
&\footnotesize\inlineocl"_ ->oclType()"&&\\
&\footnotesize\inlineocl"_ ->oclLocale()"&&\\

  \cmidrule{1-4}
  %%%%%%%%%%%%%%%%%%%%%%%%%%%%%%%%%%%%%%%%%%%%%%%%%%%%%%%%%%%%%%%%%%%%%%%
  %%%% 11.3.2 OclVoid
  %%%%%%%%%%%%%%%%%%%%%%%%%%%%%%%%%%%%%%%%%%%%%%%%%%%%%%%%%%%%%%%%%%%%%%%
  \multirow{11}{*}{\rotatebox{90}{OclVoid}}
  &\footnotesize\inlineocl"_ = _"
  & \hide{\color{Gray}($\text{\isaFS{logic}}^{\text{\color{GreenYellow}1000}}$)} \foclcolorbox{Apricot}{\isaFS{op}} \foclcolorbox{Apricot}{\isaFS{{\isasymtriangleq}}} & {{\isaFS{UML{\isacharunderscore}Logic{\isachardot}StrongEq}}\hideT{\text{\space\color{Black}\isaFS{const}}}}%
  \\
& \footnotesize\inlineocl"_ <> _"
& \hide{\color{Gray}($\text{\isaFS{logic}}^{\text{\color{GreenYellow}1000}}$)} \foclcolorbox{Apricot}{\isaFS{op}} \foclcolorbox{Apricot}{\isaFS{{\isacharless}{\isachargreater}}} & {{\color{Gray} \isaFS{notequal}}}%
  \\
&\footnotesize\inlineocl"_ ->oclAsSet( _ )"&&\\
&\footnotesize\inlineocl"_ .oclIsNew()"
& \hide{\color{Gray}($\text{\isaFS{logic}}^{\text{\color{GreenYellow}1000}}$)}\simpleArgs{$\text{\isaFS{logic}}^{\text{\color{GreenYellow}0}}$} \foclcolorbox{Apricot}{\isaFS{{\isachardot}oclIsNew{\isacharparenleft}{\isacharparenright}}} & {{ \isaFS{UML{\isacharunderscore}State{\isachardot}OclIsNew}}\hideT{\text{\space\color{Black}\isaFS{const}}}}%
\\
  %
&\footnotesize\inlineocl"not ( _ ->oclIsUndefined() )"
& \hide{\color{Gray}($\text{\isaFS{logic}}^{\text{\color{GreenYellow}100}}$)} \foclcolorbox{Apricot}{\isaFS{{\isasymdelta}}}\simpleArgs{$\text{\isaFS{logic}}^{\text{\color{GreenYellow}100}}$} & {{ \isaFS{UML{\isacharunderscore}Logic{\isachardot}defined}}\hideT{\text{\space\color{Black}\isaFS{const}}}}%
\\

%
  
&\footnotesize\inlineocl"not ( _ ->oclIsInvalid() )"
& \hide{\color{Gray}($\text{\isaFS{logic}}^{\text{\color{GreenYellow}100}}$)} \foclcolorbox{Apricot}{\isaFS{{\isasymupsilon}}}\simpleArgs{$\text{\isaFS{logic}}^{\text{\color{GreenYellow}100}}$} & {{ \isaFS{UML{\isacharunderscore}Logic{\isachardot}valid}}\hideT{\text{\space\color{Black}\isaFS{const}}}}%
\\
&\footnotesize\inlineocl"_ ->oclAsType( _ )"&&\\
&\footnotesize\inlineocl"_ ->oclIsTypeOf( _ )"&&\\
&\footnotesize\inlineocl"_ ->oclIsKindOf( _ )"&&\\
&\footnotesize\inlineocl"_ ->oclIsInState( _ )"&&\\
&\footnotesize\inlineocl"_ ->oclType()"&&\\
&\footnotesize\inlineocl"_ ->oclLocale()"&&\\
  \cmidrule{1-4}
  %%%%%%%%%%%%%%%%%%%%%%%%%%%%%%%%%%%%%%%%%%%%%%%%%%%%%%%%%%%%%%%%%%%%%%%
  %%%% 11.3.3 OclInvalid
  %%%%%%%%%%%%%%%%%%%%%%%%%%%%%%%%%%%%%%%%%%%%%%%%%%%%%%%%%%%%%%%%%%%%%%%
  \multirow{11}{*}{\rotatebox{90}{OclInvalid}}
  &\footnotesize\inlineocl"_ = _"
  & \hide{\color{Gray}($\text{\isaFS{logic}}^{\text{\color{GreenYellow}1000}}$)} \foclcolorbox{Apricot}{\isaFS{op}} \foclcolorbox{Apricot}{\isaFS{{\isasymtriangleq}}} & {{\isaFS{UML{\isacharunderscore}Logic{\isachardot}StrongEq}}\hideT{\text{\space\color{Black}\isaFS{const}}}}%
  \\
& \footnotesize\inlineocl"_ <> _"
& \hide{\color{Gray}($\text{\isaFS{logic}}^{\text{\color{GreenYellow}1000}}$)} \foclcolorbox{Apricot}{\isaFS{op}} \foclcolorbox{Apricot}{\isaFS{{\isacharless}{\isachargreater}}} & {{\color{Gray} \isaFS{notequal}}}%
  \\
&\footnotesize\inlineocl"_ ->oclAsSet( _ )"&&\\
&\footnotesize\inlineocl"_ .oclIsNew()"
& \hide{\color{Gray}($\text{\isaFS{logic}}^{\text{\color{GreenYellow}1000}}$)}\simpleArgs{$\text{\isaFS{logic}}^{\text{\color{GreenYellow}0}}$} \foclcolorbox{Apricot}{\isaFS{{\isachardot}oclIsNew{\isacharparenleft}{\isacharparenright}}} & {{ \isaFS{UML{\isacharunderscore}State{\isachardot}OclIsNew}}\hideT{\text{\space\color{Black}\isaFS{const}}}}%
\\
  %
&\footnotesize\inlineocl"not ( _ ->oclIsUndefined() )"
& \hide{\color{Gray}($\text{\isaFS{logic}}^{\text{\color{GreenYellow}100}}$)} \foclcolorbox{Apricot}{\isaFS{{\isasymdelta}}}\simpleArgs{$\text{\isaFS{logic}}^{\text{\color{GreenYellow}100}}$} & {{ \isaFS{UML{\isacharunderscore}Logic{\isachardot}defined}}\hideT{\text{\space\color{Black}\isaFS{const}}}}%
\\

%
  
&\footnotesize\inlineocl"not ( _ ->oclIsInvalid() )"
& \hide{\color{Gray}($\text{\isaFS{logic}}^{\text{\color{GreenYellow}100}}$)} \foclcolorbox{Apricot}{\isaFS{{\isasymupsilon}}}\simpleArgs{$\text{\isaFS{logic}}^{\text{\color{GreenYellow}100}}$} & {{ \isaFS{UML{\isacharunderscore}Logic{\isachardot}valid}}\hideT{\text{\space\color{Black}\isaFS{const}}}}%
\\
&\footnotesize\inlineocl"_ ->oclAsType( _ )"&&\\
&\footnotesize\inlineocl"_ ->oclIsTypeOf( _ )"&&\\
&\footnotesize\inlineocl"_ ->oclIsKindOf( _ )"&&\\
&\footnotesize\inlineocl"_ ->oclIsInState( _ )"&&\\
&\footnotesize\inlineocl"_ ->oclType()"&&\\
&\footnotesize\inlineocl"_ ->oclLocale()"&&\\
  \cmidrule{1-4}
  %%%%%%%%%%%%%%%%%%%%%%%%%%%%%%%%%%%%%%%%%%%%%%%%%%%%%%%%%%%%%%%%%%%%%%%
  %%%% 11.3.4 OclMessage
  %%%%%%%%%%%%%%%%%%%%%%%%%%%%%%%%%%%%%%%%%%%%%%%%%%%%%%%%%%%%%%%%%%%%%%%
%  \multirow{4}{*}{\rotatebox{90}{OclMessage}}
%&\footnotesize\inlineocl"_ ->hasReturned()"&&\\
%&\footnotesize\inlineocl"_ ->result()"&&\\
%&\footnotesize\inlineocl"_ ->isSignalSent()"&&\\
%&\footnotesize\inlineocl"_ ->isOperationCall()"&&\\
%  \cmidrule{1-4}
  %%%%%%%%%%%%%%%%%%%%%%%%%%%%%%%%%%%%%%%%%%%%%%%%%%%%%%%%%%%%%%%%%%%%%%%
  %%%% 11.5.1 Real
  %%%%%%%%%%%%%%%%%%%%%%%%%%%%%%%%%%%%%%%%%%%%%%%%%%%%%%%%%%%%%%%%%%%%%%%
\multirow{7}{*}{\rotatebox{90}{Real}}
&\footnotesize\inlineocl"_ + _"
& \hide{\color{Gray}($\text{\isaFS{logic}}^{\text{\color{GreenYellow}1000}}$)} \foclcolorbox{Apricot}{\isaFS{op}} \foclcolorbox{Apricot}{\isaFS{{\isacharplus}\isactrlsub r\isactrlsub e\isactrlsub a\isactrlsub l}} & {{ \isaFS{UML{\isacharunderscore}Real{\isachardot}OclAdd\isactrlsub R\isactrlsub e\isactrlsub a\isactrlsub l}}\hideT{\text{\space\color{Black}\isaFS{const}}}}%
\\
%
&\footnotesize\inlineocl"_ - _"
& \hide{\color{Gray}($\text{\isaFS{logic}}^{\text{\color{GreenYellow}1000}}$)} \foclcolorbox{Apricot}{\isaFS{op}} \foclcolorbox{Apricot}{\isaFS{{\isacharminus}\isactrlsub r\isactrlsub e\isactrlsub a\isactrlsub l}} & {{ \isaFS{UML{\isacharunderscore}Real{\isachardot}OclMinus\isactrlsub R\isactrlsub e\isactrlsub a\isactrlsub l}}\hideT{\text{\space\color{Black}\isaFS{const}}}}%
\\
%
&\footnotesize\inlineocl"_ * _"
& \hide{\color{Gray}($\text{\isaFS{logic}}^{\text{\color{GreenYellow}1000}}$)} \foclcolorbox{Apricot}{\isaFS{op}} \foclcolorbox{Apricot}{\isaFS{{\isacharasterisk}\isactrlsub r\isactrlsub e\isactrlsub a\isactrlsub l}} & {{ \isaFS{UML{\isacharunderscore}Real{\isachardot}OclMult\isactrlsub R\isactrlsub e\isactrlsub a\isactrlsub l}}\hideT{\text{\space\color{Black}\isaFS{const}}}}%
\\
& \footnotesize\inlineocl"- _" &&\\
& \footnotesize\inlineocl"_ / _" &&\\
& \footnotesize\inlineocl"_ .abs()" &&\\
& \footnotesize\inlineocl"_ .floor()" &&\\
& \footnotesize\inlineocl"_ .round()" &&\\
& \footnotesize\inlineocl"_ .max()" &&\\
& \footnotesize\inlineocl"_ .min()" &&\\
%
&\footnotesize\inlineocl"_ < _"
& \hide{\color{Gray}($\text{\isaFS{logic}}^{\text{\color{GreenYellow}1000}}$)} \foclcolorbox{Apricot}{\isaFS{op}} \foclcolorbox{Apricot}{\isaFS{{\isacharless}\isactrlsub r\isactrlsub e\isactrlsub a\isactrlsub l}} & {{ \isaFS{UML{\isacharunderscore}Real{\isachardot}OclLess\isactrlsub R\isactrlsub e\isactrlsub a\isactrlsub l}}\hideT{\text{\space\color{Black}\isaFS{const}}}}%
\\
& \footnotesize\inlineocl"_ > _" & &\\
&\footnotesize\inlineocl"_ <= _"
& \hide{\color{Gray}($\text{\isaFS{logic}}^{\text{\color{GreenYellow}1000}}$)} \foclcolorbox{Apricot}{\isaFS{op}} \foclcolorbox{Apricot}{\isaFS{{\isasymle}\isactrlsub r\isactrlsub e\isactrlsub a\isactrlsub l}} & {{ \isaFS{UML{\isacharunderscore}Real{\isachardot}OclLe\isactrlsub R\isactrlsub e\isactrlsub a\isactrlsub l}}\hideT{\text{\space\color{Black}\isaFS{const}}}}%
  \\
& \footnotesize\inlineocl"_ >= _" & &\\
& \footnotesize\inlineocl"_ .toString()" &&\\
%
&\footnotesize\textcolor{Gray}{\inlineocl"_ .div(_)"}
& \hide{\color{Gray}($\text{\isaFS{logic}}^{\text{\color{GreenYellow}1000}}$)} \foclcolorbox{Apricot}{\isaFS{op}} \foclcolorbox{Apricot}{\isaFS{div\isactrlsub r\isactrlsub e\isactrlsub a\isactrlsub l}} & {{ \isaFS{UML{\isacharunderscore}Real{\isachardot}OclDivision\isactrlsub R\isactrlsub e\isactrlsub a\isactrlsub l}}\hideT{\text{\space\color{Black}\isaFS{const}}}}%
\\
%
&\footnotesize\textcolor{Gray}{\inlineocl"_ .mod(_)"}
& \hide{\color{Gray}($\text{\isaFS{logic}}^{\text{\color{GreenYellow}1000}}$)} \foclcolorbox{Apricot}{\isaFS{op}} \foclcolorbox{Apricot}{\isaFS{mod\isactrlsub r\isactrlsub e\isactrlsub a\isactrlsub l}} & {{ \isaFS{UML{\isacharunderscore}Real{\isachardot}OclModulus\isactrlsub R\isactrlsub e\isactrlsub a\isactrlsub l}}\hideT{\text{\space\color{Black}\isaFS{const}}}}%
\\
%

  %

&\footnotesize\textcolor{Gray}{\footnotesize\inlineocl"_ ->oclAsType(Integer)"}
& \hide{\color{Gray}($\text{\isaFS{logic}}^{\text{\color{GreenYellow}1000}}$)}\simpleArgs{$\text{\isaFS{logic}}^{\text{\color{GreenYellow}0}}$} \foclcolorbox{Apricot}{\isaFS{{\isacharminus}{\isachargreater}oclAsType\isactrlsub R\isactrlsub e\isactrlsub a\isactrlsub l{\isacharparenleft}Integer{\isacharparenright}}} & {{ \isaFS{UML{\isacharunderscore}Library{\isachardot}OclAsInteger\isactrlsub R\isactrlsub e\isactrlsub a\isactrlsub l}}\hideT{\text{\space\color{Black}\isaFS{const}}}}%
\\

%

&\footnotesize\textcolor{Gray}{\footnotesize\inlineocl"_ ->oclAsType(Boolean)"}
& \hide{\color{Gray}($\text{\isaFS{logic}}^{\text{\color{GreenYellow}1000}}$)}\simpleArgs{$\text{\isaFS{logic}}^{\text{\color{GreenYellow}0}}$} \foclcolorbox{Apricot}{\isaFS{{\isacharminus}{\isachargreater}oclAsType\isactrlsub R\isactrlsub e\isactrlsub a\isactrlsub l{\isacharparenleft}Boolean{\isacharparenright}}} & {{ \isaFS{UML{\isacharunderscore}Library{\isachardot}OclAsBoolean\isactrlsub R\isactrlsub e\isactrlsub a\isactrlsub l}}\hideT{\text{\space\color{Black}\isaFS{const}}}}%
\\
\cmidrule{1-4}
%%%%
%%%%
%%%%
%%%%
\multirow{11}{*}{\rotatebox{90}{Real Literals}}
%
&\footnotesize\inlineocl"0.0"
& \hide{\color{Gray}($\text{\isaFS{logic}}^{\text{\color{GreenYellow}1000}}$)} \foclcolorbox{Apricot}{\isaFS{{\isasymzero}{\isachardot}{\isasymzero}}} & {{ \isaFS{UML{\isacharunderscore}Real{\isachardot}OclReal{\isadigit{0}}}}\hideT{\text{\space\color{Black}\isaFS{const}}}}%
\\

%
&\footnotesize\inlineocl"1.0"
& \hide{\color{Gray}($\text{\isaFS{logic}}^{\text{\color{GreenYellow}1000}}$)} \foclcolorbox{Apricot}{\isaFS{{\isasymone}{\isachardot}{\isasymzero}}} & {{ \isaFS{UML{\isacharunderscore}Real{\isachardot}OclReal{\isadigit{1}}}}\hideT{\text{\space\color{Black}\isaFS{const}}}}%
\\

%
&\footnotesize\inlineocl"2.0"
& \hide{\color{Gray}($\text{\isaFS{logic}}^{\text{\color{GreenYellow}1000}}$)} \foclcolorbox{Apricot}{\isaFS{{\isasymtwo}{\isachardot}{\isasymzero}}} & {{ \isaFS{UML{\isacharunderscore}Real{\isachardot}OclReal{\isadigit{2}}}}\hideT{\text{\space\color{Black}\isaFS{const}}}}%
\\

%
&\footnotesize\inlineocl"3.0"
& \hide{\color{Gray}($\text{\isaFS{logic}}^{\text{\color{GreenYellow}1000}}$)} \foclcolorbox{Apricot}{\isaFS{{\isasymthree}{\isachardot}{\isasymzero}}} & {{ \isaFS{UML{\isacharunderscore}Real{\isachardot}OclReal{\isadigit{3}}}}\hideT{\text{\space\color{Black}\isaFS{const}}}}%
\\

%
&\footnotesize\inlineocl"4.0"
& \hide{\color{Gray}($\text{\isaFS{logic}}^{\text{\color{GreenYellow}1000}}$)} \foclcolorbox{Apricot}{\isaFS{{\isasymfour}{\isachardot}{\isasymzero}}} & {{ \isaFS{UML{\isacharunderscore}Real{\isachardot}OclReal{\isadigit{4}}}}\hideT{\text{\space\color{Black}\isaFS{const}}}}%
\\

%
&\footnotesize\inlineocl"5.0"
& \hide{\color{Gray}($\text{\isaFS{logic}}^{\text{\color{GreenYellow}1000}}$)} \foclcolorbox{Apricot}{\isaFS{{\isasymfive}{\isachardot}{\isasymzero}}} & {{ \isaFS{UML{\isacharunderscore}Real{\isachardot}OclReal{\isadigit{5}}}}\hideT{\text{\space\color{Black}\isaFS{const}}}}%
\\

%

&\footnotesize\inlineocl"6.0"
& \hide{\color{Gray}($\text{\isaFS{logic}}^{\text{\color{GreenYellow}1000}}$)} \foclcolorbox{Apricot}{\isaFS{{\isasymsix}{\isachardot}{\isasymzero}}} & {{ \isaFS{UML{\isacharunderscore}Real{\isachardot}OclReal{\isadigit{6}}}}\hideT{\text{\space\color{Black}\isaFS{const}}}}%
\\

%

&\footnotesize\inlineocl"7.0"
& \hide{\color{Gray}($\text{\isaFS{logic}}^{\text{\color{GreenYellow}1000}}$)} \foclcolorbox{Apricot}{\isaFS{{\isasymseven}{\isachardot}{\isasymzero}}} & {{ \isaFS{UML{\isacharunderscore}Real{\isachardot}OclReal{\isadigit{7}}}}\hideT{\text{\space\color{Black}\isaFS{const}}}}%
\\

%

&\footnotesize\inlineocl"8.0"
& \hide{\color{Gray}($\text{\isaFS{logic}}^{\text{\color{GreenYellow}1000}}$)} \foclcolorbox{Apricot}{\isaFS{{\isasymeight}{\isachardot}{\isasymzero}}} & {{ \isaFS{UML{\isacharunderscore}Real{\isachardot}OclReal{\isadigit{8}}}}\hideT{\text{\space\color{Black}\isaFS{const}}}}%
\\

%

&\footnotesize\inlineocl"9.0"
& \hide{\color{Gray}($\text{\isaFS{logic}}^{\text{\color{GreenYellow}1000}}$)} \foclcolorbox{Apricot}{\isaFS{{\isasymnine}{\isachardot}{\isasymzero}}} & {{ \isaFS{UML{\isacharunderscore}Real{\isachardot}OclReal{\isadigit{9}}}}\hideT{\text{\space\color{Black}\isaFS{const}}}}%
\\

%
&\footnotesize\inlineocl"10.0"
& \hide{\color{Gray}($\text{\isaFS{logic}}^{\text{\color{GreenYellow}1000}}$)} \foclcolorbox{Apricot}{\isaFS{{\isasymone}{\isasymzero}{\isachardot}{\isasymzero}}} & {{ \isaFS{UML{\isacharunderscore}Real{\isachardot}OclReal{\isadigit{1}}{\isadigit{0}}}}\hideT{\text{\space\color{Black}\isaFS{const}}}}%
  \\
&
& \hide{\color{Gray}($\text{\isaFS{logic}}^{\text{\color{GreenYellow}1000}}$)} \foclcolorbox{Apricot}{\isaFS{{\isasympi}}} & {{ \isaFS{UML{\isacharunderscore}Real{\isachardot}OclRealpi}}\hideT{\text{\space\color{Black}\isaFS{const}}}}%
\\
\cmidrule{1-4}
  %%%%%%%%%%%%%%%%%%%%%%%%%%%%%%%%%%%%%%%%%%%%%%%%%%%%%%%%%%%%%%%%%%%%%%%
  %%%% 11.5.2 Integer
  %%%%%%%%%%%%%%%%%%%%%%%%%%%%%%%%%%%%%%%%%%%%%%%%%%%%%%%%%%%%%%%%%%%%%%%
\multirow{7}{*}{\rotatebox{90}{Integer}}
&\footnotesize\inlineocl"_ - _"
& \hide{\color{Gray}($\text{\isaFS{logic}}^{\text{\color{GreenYellow}1000}}$)} \foclcolorbox{Apricot}{\isaFS{op}} \foclcolorbox{Apricot}{\isaFS{{\isacharminus}\isactrlsub i\isactrlsub n\isactrlsub t}} & {{ \isaFS{UML{\isacharunderscore}Integer{\isachardot}OclMinus\isactrlsub I\isactrlsub n\isactrlsub t\isactrlsub e\isactrlsub g\isactrlsub e\isactrlsub r}}\hideT{\text{\space\color{Black}\isaFS{const}}}}%
\\
&\footnotesize\inlineocl"_ + _"
& \hide{\color{Gray}($\text{\isaFS{logic}}^{\text{\color{GreenYellow}1000}}$)} \foclcolorbox{Apricot}{\isaFS{op}} \foclcolorbox{Apricot}{\isaFS{{\isacharplus}\isactrlsub i\isactrlsub n\isactrlsub t}} & {{ \isaFS{UML{\isacharunderscore}Integer{\isachardot}OclAdd\isactrlsub I\isactrlsub n\isactrlsub t\isactrlsub e\isactrlsub g\isactrlsub e\isactrlsub r}}\hideT{\text{\space\color{Black}\isaFS{const}}}}%
\\
%
  &\footnotesize\inlineocl"- _" && \\
% 
&\footnotesize\inlineocl"_ * _"
& \hide{\color{Gray}($\text{\isaFS{logic}}^{\text{\color{GreenYellow}1000}}$)} \foclcolorbox{Apricot}{\isaFS{op}} \foclcolorbox{Apricot}{\isaFS{{\isacharasterisk}\isactrlsub i\isactrlsub n\isactrlsub t}} & {{ \isaFS{UML{\isacharunderscore}Integer{\isachardot}OclMult\isactrlsub I\isactrlsub n\isactrlsub t\isactrlsub e\isactrlsub g\isactrlsub e\isactrlsub r}}\hideT{\text{\space\color{Black}\isaFS{const}}}}%
\\
  &\footnotesize\inlineocl"_ / _" && \\
  &\footnotesize\inlineocl"_ .abs()" && \\

  %
  
&\footnotesize\inlineocl"_ div ( _ )"
& \hide{\color{Gray}($\text{\isaFS{logic}}^{\text{\color{GreenYellow}1000}}$)} \foclcolorbox{Apricot}{\isaFS{op}} \foclcolorbox{Apricot}{\isaFS{div\isactrlsub i\isactrlsub n\isactrlsub t}} & {{ \isaFS{UML{\isacharunderscore}Integer{\isachardot}OclDivision\isactrlsub I\isactrlsub n\isactrlsub t\isactrlsub e\isactrlsub g\isactrlsub e\isactrlsub r}}\hideT{\text{\space\color{Black}\isaFS{const}}}}%
\\
%
&\footnotesize\inlineocl"_ mod ( _ )"
& \hide{\color{Gray}($\text{\isaFS{logic}}^{\text{\color{GreenYellow}1000}}$)} \foclcolorbox{Apricot}{\isaFS{op}} \foclcolorbox{Apricot}{\isaFS{mod\isactrlsub i\isactrlsub n\isactrlsub t}} & {{ \isaFS{UML{\isacharunderscore}Integer{\isachardot}OclModulus\isactrlsub I\isactrlsub n\isactrlsub t\isactrlsub e\isactrlsub g\isactrlsub e\isactrlsub r}}\hideT{\text{\space\color{Black}\isaFS{const}}}}%
\\
%
& \footnotesize\inlineocl"_ .max()" &&\\
& \footnotesize\inlineocl"_ .min()" &&\\
& \footnotesize\inlineocl"_ .toString()" &&\\

  
&\textcolor{Gray}{\footnotesize\inlineocl"_ < _"}
& \hide{\color{Gray}($\text{\isaFS{logic}}^{\text{\color{GreenYellow}1000}}$)} \foclcolorbox{Apricot}{\isaFS{op}} \foclcolorbox{Apricot}{\isaFS{{\isacharless}\isactrlsub i\isactrlsub n\isactrlsub t}} & {{ \isaFS{UML{\isacharunderscore}Integer{\isachardot}OclLess\isactrlsub I\isactrlsub n\isactrlsub t\isactrlsub e\isactrlsub g\isactrlsub e\isactrlsub r}}\hideT{\text{\space\color{Black}\isaFS{const}}}}%
\\
%
&\textcolor{Gray}{\footnotesize\inlineocl"_ <= _"}
& \hide{\color{Gray}($\text{\isaFS{logic}}^{\text{\color{GreenYellow}1000}}$)} \foclcolorbox{Apricot}{\isaFS{op}} \foclcolorbox{Apricot}{\isaFS{{\isasymle}\isactrlsub i\isactrlsub n\isactrlsub t}} & {{ \isaFS{UML{\isacharunderscore}Integer{\isachardot}OclLe\isactrlsub I\isactrlsub n\isactrlsub t\isactrlsub e\isactrlsub g\isactrlsub e\isactrlsub r}}\hideT{\text{\space\color{Black}\isaFS{const}}}}%
  \\
  
&\textcolor{Gray}{\footnotesize\inlineocl"_ ->oclAsType(Real)"}
& \hide{\color{Gray}($\text{\isaFS{logic}}^{\text{\color{GreenYellow}1000}}$)}\simpleArgs{$\text{\isaFS{logic}}^{\text{\color{GreenYellow}0}}$} \foclcolorbox{Apricot}{\isaFS{{\isacharminus}{\isachargreater}oclAsType\isactrlsub I\isactrlsub n\isactrlsub t{\isacharparenleft}Real{\isacharparenright}}} & {{ \isaFS{UML{\isacharunderscore}Library{\isachardot}OclAsReal\isactrlsub I\isactrlsub n\isactrlsub t}}\hideT{\text{\space\color{Black}\isaFS{const}}}}%
\\
%
&\textcolor{Gray}{\footnotesize\inlineocl"_ ->oclAsType(Boolean)"}
& \hide{\color{Gray}($\text{\isaFS{logic}}^{\text{\color{GreenYellow}1000}}$)}\simpleArgs{$\text{\isaFS{logic}}^{\text{\color{GreenYellow}0}}$} \foclcolorbox{Apricot}{\isaFS{{\isacharminus}{\isachargreater}oclAsType\isactrlsub I\isactrlsub n\isactrlsub t{\isacharparenleft}Boolean{\isacharparenright}}} & {{ \isaFS{UML{\isacharunderscore}Library{\isachardot}OclAsBoolean\isactrlsub I\isactrlsub n\isactrlsub t}}\hideT{\text{\space\color{Black}\isaFS{const}}}}%
\\
\cmidrule{1-4}
%%%%
%%%%
%%%%
%%%%
\multirow{10}{*}{\rotatebox{90}{Integer Literals}}
&\footnotesize\inlineocl"0"
& \hide{\color{Gray}($\text{\isaFS{logic}}^{\text{\color{GreenYellow}1000}}$)} \foclcolorbox{Apricot}{\isaFS{{\isasymzero}}} & {{ \isaFS{UML{\isacharunderscore}Integer{\isachardot}OclInt{\isadigit{0}}}}\hideT{\text{\space\color{Black}\isaFS{const}}}}%
\\

%
&\footnotesize\inlineocl"1"
& \hide{\color{Gray}($\text{\isaFS{logic}}^{\text{\color{GreenYellow}1000}}$)} \foclcolorbox{Apricot}{\isaFS{{\isasymone}}} & {{ \isaFS{UML{\isacharunderscore}Integer{\isachardot}OclInt{\isadigit{1}}}}\hideT{\text{\space\color{Black}\isaFS{const}}}}%
\\

%
&\footnotesize\inlineocl"2"
& \hide{\color{Gray}($\text{\isaFS{logic}}^{\text{\color{GreenYellow}1000}}$)} \foclcolorbox{Apricot}{\isaFS{{\isasymtwo}}} & {{ \isaFS{UML{\isacharunderscore}Integer{\isachardot}OclInt{\isadigit{2}}}}\hideT{\text{\space\color{Black}\isaFS{const}}}}%
\\

%
&\footnotesize\inlineocl"3"
& \hide{\color{Gray}($\text{\isaFS{logic}}^{\text{\color{GreenYellow}1000}}$)} \foclcolorbox{Apricot}{\isaFS{{\isasymthree}}} & {{ \isaFS{UML{\isacharunderscore}Integer{\isachardot}OclInt{\isadigit{3}}}}\hideT{\text{\space\color{Black}\isaFS{const}}}}%
\\

%
&\footnotesize\inlineocl"4"
& \hide{\color{Gray}($\text{\isaFS{logic}}^{\text{\color{GreenYellow}1000}}$)} \foclcolorbox{Apricot}{\isaFS{{\isasymfour}}} & {{ \isaFS{UML{\isacharunderscore}Integer{\isachardot}OclInt{\isadigit{4}}}}\hideT{\text{\space\color{Black}\isaFS{const}}}}%
\\

%
&\footnotesize\inlineocl"5"
& \hide{\color{Gray}($\text{\isaFS{logic}}^{\text{\color{GreenYellow}1000}}$)} \foclcolorbox{Apricot}{\isaFS{{\isasymfive}}} & {{ \isaFS{UML{\isacharunderscore}Integer{\isachardot}OclInt{\isadigit{5}}}}\hideT{\text{\space\color{Black}\isaFS{const}}}}%
\\

%
&\footnotesize\inlineocl"6"
& \hide{\color{Gray}($\text{\isaFS{logic}}^{\text{\color{GreenYellow}1000}}$)} \foclcolorbox{Apricot}{\isaFS{{\isasymsix}}} & {{ \isaFS{UML{\isacharunderscore}Integer{\isachardot}OclInt{\isadigit{6}}}}\hideT{\text{\space\color{Black}\isaFS{const}}}}%
\\

%
&\footnotesize\inlineocl"7"
& \hide{\color{Gray}($\text{\isaFS{logic}}^{\text{\color{GreenYellow}1000}}$)} \foclcolorbox{Apricot}{\isaFS{{\isasymseven}}} & {{ \isaFS{UML{\isacharunderscore}Integer{\isachardot}OclInt{\isadigit{7}}}}\hideT{\text{\space\color{Black}\isaFS{const}}}}%
\\

%
&\footnotesize\inlineocl"8"
& \hide{\color{Gray}($\text{\isaFS{logic}}^{\text{\color{GreenYellow}1000}}$)} \foclcolorbox{Apricot}{\isaFS{{\isasymeight}}} & {{ \isaFS{UML{\isacharunderscore}Integer{\isachardot}OclInt{\isadigit{8}}}}\hideT{\text{\space\color{Black}\isaFS{const}}}}%
\\

%
&\footnotesize\inlineocl"9"
& \hide{\color{Gray}($\text{\isaFS{logic}}^{\text{\color{GreenYellow}1000}}$)} \foclcolorbox{Apricot}{\isaFS{{\isasymnine}}} & {{ \isaFS{UML{\isacharunderscore}Integer{\isachardot}OclInt{\isadigit{9}}}}\hideT{\text{\space\color{Black}\isaFS{const}}}}%
\\

%
&\footnotesize\inlineocl"10"
& \hide{\color{Gray}($\text{\isaFS{logic}}^{\text{\color{GreenYellow}1000}}$)} \foclcolorbox{Apricot}{\isaFS{{\isasymone}{\isasymzero}}} & {{ \isaFS{UML{\isacharunderscore}Integer{\isachardot}OclInt{\isadigit{1}}{\isadigit{0}}}}\hideT{\text{\space\color{Black}\isaFS{const}}}}%
\\
\cmidrule{1-4}
  %%%%%%%%%%%%%%%%%%%%%%%%%%%%%%%%%%%%%%%%%%%%%%%%%%%%%%%%%%%%%%%%%%%%%%%
  %%%% 11.5.3 String
  %%%%%%%%%%%%%%%%%%%%%%%%%%%%%%%%%%%%%%%%%%%%%%%%%%%%%%%%%%%%%%%%%%%%%%%
\multirow{20}{*}{\rotatebox{90}{String and String Literals}}
&\footnotesize\inlineocl"_ + _"
& \hide{\color{Gray}($\text{\isaFS{logic}}^{\text{\color{GreenYellow}1000}}$)} \foclcolorbox{Apricot}{\isaFS{op}} \foclcolorbox{Apricot}{\isaFS{{\isacharplus}\isactrlsub s\isactrlsub t\isactrlsub r\isactrlsub i\isactrlsub n\isactrlsub g}} & {{ \isaFS{UML{\isacharunderscore}String{\isachardot}OclAdd\isactrlsub S\isactrlsub t\isactrlsub r\isactrlsub i\isactrlsub n\isactrlsub g}}\hideT{\text{\space\color{Black}\isaFS{const}}}}%
\\
&\footnotesize\inlineocl"_ .size()"&&\\
&\footnotesize\inlineocl"_ .concat( _ )"&&\\
&\footnotesize\inlineocl"_ .substring( _ , _ )"&&\\
&\footnotesize\inlineocl"_ .toInteger()"&&\\
&\footnotesize\inlineocl"_ .toReal()"&&\\
&\footnotesize\inlineocl"_ .toUpperCase()"&&\\
&\footnotesize\inlineocl"_ .toLowerCase()"&&\\
&\footnotesize\inlineocl"_ .indexOf()"&&\\
&\footnotesize\inlineocl"_ .equalsIgnoreCase( _ )"&&\\
&\footnotesize\inlineocl"_ .at( _ )"&&\\
&\footnotesize\inlineocl"_ .characters()"&&\\
&\footnotesize\inlineocl"_ .toBoolean()"&&\\
&\footnotesize\inlineocl"_ < _ "&&\\
&\footnotesize\inlineocl"_ > _ "&&\\
&\footnotesize\inlineocl"_ <= _ "&&\\
&\footnotesize\inlineocl"_ >= _ "&&\\
%
&\footnotesize\inlineocl"a"
& \hide{\color{Gray}($\text{\isaFS{logic}}^{\text{\color{GreenYellow}1000}}$)} \foclcolorbox{Apricot}{\isaFS{{\isasyma}}} & {{ \isaFS{UML{\isacharunderscore}String{\isachardot}OclStringa}}\hideT{\text{\space\color{Black}\isaFS{const}}}}%
\\

%
&\footnotesize\inlineocl"b"
& \hide{\color{Gray}($\text{\isaFS{logic}}^{\text{\color{GreenYellow}1000}}$)} \foclcolorbox{Apricot}{\isaFS{{\isasymb}}} & {{ \isaFS{UML{\isacharunderscore}String{\isachardot}OclStringb}}\hideT{\text{\space\color{Black}\isaFS{const}}}}%
\\

%
&\footnotesize\inlineocl"c"
& \hide{\color{Gray}($\text{\isaFS{logic}}^{\text{\color{GreenYellow}1000}}$)} \foclcolorbox{Apricot}{\isaFS{{\isasymc}}} & {{ \isaFS{UML{\isacharunderscore}String{\isachardot}OclStringc}}\hideT{\text{\space\color{Black}\isaFS{const}}}}%
\\


\cmidrule{1-4}
  %%%%%%%%%%%%%%%%%%%%%%%%%%%%%%%%%%%%%%%%%%%%%%%%%%%%%%%%%%%%%%%%%%%%%%%
  %%%% 11.5.4 Boolean
  %%%%%%%%%%%%%%%%%%%%%%%%%%%%%%%%%%%%%%%%%%%%%%%%%%%%%%%%%%%%%%%%%%%%%%%
\multirow{6}{*}{\rotatebox{90}{Boolean and Core Logic}}
%
& \footnotesize\inlineocl"_ or _"
& \hide{\color{Gray}($\text{\isaFS{logic}}^{\text{\color{GreenYellow}1000}}$)} \foclcolorbox{Apricot}{\isaFS{op}} \foclcolorbox{Apricot}{\isaFS{or}} & {{ \isaFS{UML{\isacharunderscore}Logic{\isachardot}OclOr}}\hideT{\text{\space\color{Black}\isaFS{const}}}}%
\\
& \footnotesize\inlineocl"_ xor _"&&\\
& \footnotesize\inlineocl"_ and _"
& \hide{\color{Gray}($\text{\isaFS{logic}}^{\text{\color{GreenYellow}1000}}$)} \foclcolorbox{Apricot}{\isaFS{op}} \foclcolorbox{Apricot}{\isaFS{and}} & {{ \isaFS{UML{\isacharunderscore}Logic{\isachardot}OclAnd}}\hideT{\text{\space\color{Black}\isaFS{const}}}}%
\\
%
&\footnotesize\inlineocl"not _"
& \hide{\color{Gray}($\text{\isaFS{logic}}^{\text{\color{GreenYellow}1000}}$)}
  \foclcolorbox{Apricot}{\isaFS{not}} & {{
                                        \isaFS{UML{\isacharunderscore}Logic{\isachardot}OclNot}}\hideT{\text{\space\color{Black}\isaFS{const}}}}%
  \\
&\footnotesize\inlineocl"_ implies _"
& \hide{\color{Gray}($\text{\isaFS{logic}}^{\text{\color{GreenYellow}1000}}$)} \foclcolorbox{Apricot}{\isaFS{op}} \foclcolorbox{Apricot}{\isaFS{implies}} & {{ \isaFS{UML{\isacharunderscore}Logic{\isachardot}OclImplies}}\hideT{\text{\space\color{Black}\isaFS{const}}}}%
\\
&\footnotesize\inlineocl"_ .toString()"&&\\
  &\footnotesize\inlineocl"if _  then _ else _ endif"
& \hide{\color{Gray}($\text{\isaFS{logic}}^{\text{\color{GreenYellow}50}}$)} \foclcolorbox{Apricot}{\isaFS{if}}\simpleArgs{$\text{\isaFS{logic}}^{\text{\color{GreenYellow}10}}$} \foclcolorbox{Apricot}{\isaFS{then}} \simpleArgs{$\text{\isaFS{logic}}^{\text{\color{GreenYellow}10}}$} \foclcolorbox{Apricot}{\isaFS{else}} \simpleArgs{$\text{\isaFS{logic}}^{\text{\color{GreenYellow}10}}$} \foclcolorbox{Apricot}{\isaFS{endif}} & {{ \isaFS{UML{\isacharunderscore}Logic{\isachardot}OclIf}}\hideT{\text{\space\color{Black}\isaFS{const}}}}%
\\
& \footnotesize\inlineocl"_ = _"
& \hide{\color{Gray}($\text{\isaFS{logic}}^{\text{\color{GreenYellow}1000}}$)} \foclcolorbox{Apricot}{\isaFS{op}} \foclcolorbox{Apricot}{\isaFS{{\isasymdoteq}}} & {{ \isaFS{UML{\isacharunderscore}Logic{\isachardot}StrictRefEq}}\hideT{\text{\space\color{Black}\isaFS{const}}}}%
\\
%
& \footnotesize\inlineocl"_ <> _"
& \hide{\color{Gray}($\text{\isaFS{logic}}^{\text{\color{GreenYellow}1000}}$)} \foclcolorbox{Apricot}{\isaFS{op}} \foclcolorbox{Apricot}{\isaFS{{\isacharless}{\isachargreater}}} & {{\color{Gray} \isaFS{notequal}}}%
  \\
%
  %
&
& \hide{\color{Gray}($\text{\isaFS{logic}}^{\text{\color{GreenYellow}50}}$)}\simpleArgs{$\text{\isaFS{logic}}^{\text{\color{GreenYellow}0}}$} \foclcolorbox{Apricot}{\isaFS{{\isacharbar}{\isasymnoteq}}} \simpleArgs{$\text{\isaFS{logic}}^{\text{\color{GreenYellow}0}}$} & {{\color{Gray} \isaFS{OclNonValid}}}%
\\
%
&
& \hide{\color{Gray}($\text{\isaFS{logic}}^{\text{\color{GreenYellow}50}}$)}\simpleArgs{$\text{\isaFS{logic}}^{\text{\color{GreenYellow}0}}$} \foclcolorbox{Apricot}{\isaFS{{\isasymTurnstile}}} \simpleArgs{$\text{\isaFS{logic}}^{\text{\color{GreenYellow}0}}$} & {{ \isaFS{UML{\isacharunderscore}Logic{\isachardot}OclValid}}\hideT{\text{\space\color{Black}\isaFS{const}}}}%
\\
&\footnotesize\textcolor{Gray}{\inlineocl"_ = _"}
& \hide{\color{Gray}($\text{\isaFS{logic}}^{\text{\color{GreenYellow}1000}}$)} \foclcolorbox{Apricot}{\isaFS{op}} \foclcolorbox{Apricot}{\isaFS{{\isasymtriangleq}}} & {{\isaFS{UML{\isacharunderscore}Logic{\isachardot}StrongEq}}\hideT{\text{\space\color{Black}\isaFS{const}}}}%
\\
%

\cmidrule{1-4}
  %%%%%%%%%%%%%%%%%%%%%%%%%%%%%%%%%%%%%%%%%%%%%%%%%%%%%%%%%%%%%%%%%%%%%%%
  %%%% 11.5.5 UnlimitedNatural
  %%%%%%%%%%%%%%%%%%%%%%%%%%%%%%%%%%%%%%%%%%%%%%%%%%%%%%%%%%%%%%%%%%%%%%%

  %%%%%%%%%%%%%%%%%%%%%%%%%%%%%%%%%%%%%%%%%%%%%%%%%%%%%%%%%%%%%%%%%%%%%%%
  %%%% 11.7.1 Collection
  %%%%%%%%%%%%%%%%%%%%%%%%%%%%%%%%%%%%%%%%%%%%%%%%%%%%%%%%%%%%%%%%%%%%%%%

  %%%%%%%%%%%%%%%%%%%%%%%%%%%%%%%%%%%%%%%%%%%%%%%%%%%%%%%%%%%%%%%%%%%%%%% 
  %%%% 11.7.2 Set
  %%%%%%%%%%%%%%%%%%%%%%%%%%%%%%%%%%%%%%%%%%%%%%%%%%%%%%%%%%%%%%%%%%%%%%%
  \multirow{12}{*}{\rotatebox{90}{Set and Iterators on Set}}
&\footnotesize\inlineocl"Set ( _ )"
& \hide{\color{Gray}($\text{\isaFS{type}}^{\text{\color{GreenYellow}1000}}$)} \foclcolorbox{Apricot}{\isaFS{Set{\isacharparenleft}}} $\text{\isaFS{type}}^{\text{\color{GreenYellow}0}}$ \foclcolorbox{Apricot}{\isaFS{{\isacharparenright}}} & {{ \isaFS{UML{\isacharunderscore}Types{\isachardot}Set\isactrlsub b\isactrlsub a\isactrlsub s\isactrlsub e}}\text{\space\color{Black}\isaFS{type}}}%
\\

%

&\footnotesize\inlineocl"Set{}"
       & \hide{\color{Gray}($\text{\isaFS{logic}}^{\text{\color{GreenYellow}1000}}$)} \foclcolorbox{Apricot}{\isaFS{Set{\isacharbraceleft}{\isacharbraceright}}} & {{ \isaFS{UML{\isacharunderscore}Set{\isachardot}mtSet}}\hideT{\text{\space\color{Black}\isaFS{const}}}}%
\\

%
&\footnotesize\inlineocl"Set{ _ }"
& \hide{\color{Gray}($\text{\isaFS{logic}}^{\text{\color{GreenYellow}1000}}$)} \foclcolorbox{Apricot}{\isaFS{Set{\isacharbraceleft}}} $\text{\isaFS{args}}^{\text{\color{GreenYellow}0}}$ \foclcolorbox{Apricot}{\isaFS{{\isacharbraceright}}} & {{\color{Gray} \isaFS{OclFinset}}}%
\\
       &\footnotesize\inlineocl"_ ->union( _ )"
& \hide{\color{Gray}($\text{\isaFS{logic}}^{\text{\color{GreenYellow}1000}}$)}\simpleArgs{$\text{\isaFS{logic}}^{\text{\color{GreenYellow}0}}$} \foclcolorbox{Apricot}{\isaFS{{\isacharminus}{\isachargreater}union\isactrlsub S\isactrlsub e\isactrlsub t{\isacharparenleft}}} \simpleArgs{$\text{\isaFS{logic}}^{\text{\color{GreenYellow}0}}$} \foclcolorbox{Apricot}{\isaFS{{\isacharparenright}}} & {{ \isaFS{UML{\isacharunderscore}Set{\isachardot}OclUnion}}\hideT{\text{\space\color{Black}\isaFS{const}}}}%
\\
  &\footnotesize\inlineocl"_ = _"
  & \hide{\color{Gray}($\text{\isaFS{logic}}^{\text{\color{GreenYellow}1000}}$)} \foclcolorbox{Apricot}{\isaFS{op}} \foclcolorbox{Apricot}{\isaFS{{\isasymtriangleq}}} & {{\isaFS{UML{\isacharunderscore}Logic{\isachardot}StrongEq}}\hideT{\text{\space\color{Black}\isaFS{const}}}}%
  \\
&\footnotesize\inlineocl"_ ->intersection( _ )"
& \hide{\color{Gray}($\text{\isaFS{logic}}^{\text{\color{GreenYellow}1000}}$)}\simpleArgs{$\text{\isaFS{logic}}^{\text{\color{GreenYellow}0}}$} \foclcolorbox{Apricot}{\isaFS{{\isacharminus}{\isachargreater}intersection\isactrlsub S\isactrlsub e\isactrlsub t{\isacharparenleft}}} \simpleArgs{$\text{\isaFS{logic}}^{\text{\color{GreenYellow}0}}$} \foclcolorbox{Apricot}{\isaFS{{\isacharparenright}}} & {{ \isaFS{UML{\isacharunderscore}Set{\isachardot}OclIntersection}}\hideT{\text{\space\color{Black}\isaFS{const}}}}%
\\
&\footnotesize\inlineocl"_ - _"&&\\

&\footnotesize\inlineocl"_ ->including( _ )"
& \hide{\color{Gray}($\text{\isaFS{logic}}^{\text{\color{GreenYellow}1000}}$)}\simpleArgs{$\text{\isaFS{logic}}^{\text{\color{GreenYellow}0}}$} \foclcolorbox{Apricot}{\isaFS{{\isacharminus}{\isachargreater}including\isactrlsub S\isactrlsub e\isactrlsub t{\isacharparenleft}}} \simpleArgs{$\text{\isaFS{logic}}^{\text{\color{GreenYellow}0}}$} \foclcolorbox{Apricot}{\isaFS{{\isacharparenright}}} & {{ \isaFS{UML{\isacharunderscore}Set{\isachardot}OclIncluding}}\hideT{\text{\space\color{Black}\isaFS{const}}}}%
\\

&\footnotesize\inlineocl"_ ->excluding( _ )"
& \hide{\color{Gray}($\text{\isaFS{logic}}^{\text{\color{GreenYellow}1000}}$)}\simpleArgs{$\text{\isaFS{logic}}^{\text{\color{GreenYellow}0}}$} \foclcolorbox{Apricot}{\isaFS{{\isacharminus}{\isachargreater}excluding\isactrlsub S\isactrlsub e\isactrlsub t{\isacharparenleft}}} \simpleArgs{$\text{\isaFS{logic}}^{\text{\color{GreenYellow}0}}$} \foclcolorbox{Apricot}{\isaFS{{\isacharparenright}}} & {{ \isaFS{UML{\isacharunderscore}Set{\isachardot}OclExcluding}}\hideT{\text{\space\color{Black}\isaFS{const}}}}%
\\

&\footnotesize\inlineocl"_ ->symmetricDifference( _ )"&&\\

&\footnotesize\inlineocl"_ ->count( _ )"
& \hide{\color{Gray}($\text{\isaFS{logic}}^{\text{\color{GreenYellow}1000}}$)}\simpleArgs{$\text{\isaFS{logic}}^{\text{\color{GreenYellow}0}}$} \foclcolorbox{Apricot}{\isaFS{{\isacharminus}{\isachargreater}count\isactrlsub S\isactrlsub e\isactrlsub t{\isacharparenleft}}} \simpleArgs{$\text{\isaFS{logic}}^{\text{\color{GreenYellow}0}}$} \foclcolorbox{Apricot}{\isaFS{{\isacharparenright}}} & {{ \isaFS{UML{\isacharunderscore}Set{\isachardot}OclCount}}\hideT{\text{\space\color{Black}\isaFS{const}}}}%
\\

&\footnotesize\inlineocl"_ ->flatten()"&&\\
&\footnotesize\inlineocl"_ ->selectByKind( _ )"&&\\
&\footnotesize\inlineocl"_ ->selectByType( _ )"&&\\
  
&\footnotesize\inlineocl"_ ->reject( _ | _ )"
& \hide{\color{Gray}($\text{\isaFS{logic}}^{\text{\color{GreenYellow}1000}}$)}\simpleArgs{$\text{\isaFS{logic}}^{\text{\color{GreenYellow}0}}$} \foclcolorbox{Apricot}{\isaFS{{\isacharminus}{\isachargreater}reject\isactrlsub S\isactrlsub e\isactrlsub t{\isacharparenleft}}} \fbox{$\text{\isaFS{id}}$} \foclcolorbox{Apricot}{\isaFS{{\isacharbar}}} \simpleArgs{$\text{\isaFS{logic}}^{\text{\color{GreenYellow}0}}$} \foclcolorbox{Apricot}{\isaFS{{\isacharparenright}}} & {{\color{Gray} \isaFS{OclRejectSet}}}%
\\

%

&\footnotesize\inlineocl"_ ->select( _ | _ )"
& \hide{\color{Gray}($\text{\isaFS{logic}}^{\text{\color{GreenYellow}1000}}$)}\simpleArgs{$\text{\isaFS{logic}}^{\text{\color{GreenYellow}0}}$} \foclcolorbox{Apricot}{\isaFS{{\isacharminus}{\isachargreater}select\isactrlsub S\isactrlsub e\isactrlsub t{\isacharparenleft}}} \fbox{$\text{\isaFS{id}}$} \foclcolorbox{Apricot}{\isaFS{{\isacharbar}}} \simpleArgs{$\text{\isaFS{logic}}^{\text{\color{GreenYellow}0}}$} \foclcolorbox{Apricot}{\isaFS{{\isacharparenright}}} & {{\color{Gray} \isaFS{OclSelectSet}}}%
\\

%

&\footnotesize\inlineocl"_ ->iterate( _ ; _ = _ | _ )"
& \hide{\color{Gray}($\text{\isaFS{logic}}^{\text{\color{GreenYellow}1000}}$)}\simpleArgs{$\text{\isaFS{logic}}^{\text{\color{GreenYellow}0}}$} \foclcolorbox{Apricot}{\isaFS{{\isacharminus}{\isachargreater}iterate\isactrlsub S\isactrlsub e\isactrlsub t{\isacharparenleft}}} $\text{\isaFS{idt}}^{\text{\color{GreenYellow}0}}$ \foclcolorbox{Apricot}{\isaFS{{\isacharsemicolon}}} $\text{\isaFS{idt}}^{\text{\color{GreenYellow}0}}$ \foclcolorbox{Apricot}{\isaFS{{\isacharequal}}} $\text{\isaFS{any}}^{\text{\color{GreenYellow}0}}$ \foclcolorbox{Apricot}{\isaFS{{\isacharbar}}} $\text{\isaFS{any}}^{\text{\color{GreenYellow}0}}$ \foclcolorbox{Apricot}{\isaFS{{\isacharparenright}}} & {{\color{Gray} \isaFS{OclIterateSet}}}%
\\

%

&\footnotesize\inlineocl"_ ->exists( _ | _ )"
& \hide{\color{Gray}($\text{\isaFS{logic}}^{\text{\color{GreenYellow}1000}}$)}\simpleArgs{$\text{\isaFS{logic}}^{\text{\color{GreenYellow}0}}$} \foclcolorbox{Apricot}{\isaFS{{\isacharminus}{\isachargreater}exists\isactrlsub S\isactrlsub e\isactrlsub t{\isacharparenleft}}} \fbox{$\text{\isaFS{id}}$} \foclcolorbox{Apricot}{\isaFS{{\isacharbar}}} \simpleArgs{$\text{\isaFS{logic}}^{\text{\color{GreenYellow}0}}$} \foclcolorbox{Apricot}{\isaFS{{\isacharparenright}}} & {{\color{Gray} \isaFS{OclExistSet}}}%
\\

%

&\footnotesize\inlineocl"_ ->forAll( _ | _ )"
& \hide{\color{Gray}($\text{\isaFS{logic}}^{\text{\color{GreenYellow}1000}}$)}\simpleArgs{$\text{\isaFS{logic}}^{\text{\color{GreenYellow}0}}$} \foclcolorbox{Apricot}{\isaFS{{\isacharminus}{\isachargreater}forAll\isactrlsub S\isactrlsub e\isactrlsub t{\isacharparenleft}}} \fbox{$\text{\isaFS{id}}$} \foclcolorbox{Apricot}{\isaFS{{\isacharbar}}} \simpleArgs{$\text{\isaFS{logic}}^{\text{\color{GreenYellow}0}}$} \foclcolorbox{Apricot}{\isaFS{{\isacharparenright}}} & {{\color{Gray} \isaFS{OclForallSet}}}%
\\


  %
&\footnotesize\inlineocl"_ ->asSequence()"
& \hide{\color{Gray}($\text{\isaFS{logic}}^{\text{\color{GreenYellow}1000}}$)}\simpleArgs{$\text{\isaFS{logic}}^{\text{\color{GreenYellow}0}}$} \foclcolorbox{Apricot}{\isaFS{{\isacharminus}{\isachargreater}asSequence\isactrlsub S\isactrlsub e\isactrlsub t{\isacharparenleft}{\isacharparenright}}} & {{ \isaFS{UML{\isacharunderscore}Library{\isachardot}OclAsSeq\isactrlsub S\isactrlsub e\isactrlsub t}}\hideT{\text{\space\color{Black}\isaFS{const}}}}%
\\
%
&\footnotesize\inlineocl"_ ->asBag()"
& \hide{\color{Gray}($\text{\isaFS{logic}}^{\text{\color{GreenYellow}1000}}$)}\simpleArgs{$\text{\isaFS{logic}}^{\text{\color{GreenYellow}0}}$} \foclcolorbox{Apricot}{\isaFS{{\isacharminus}{\isachargreater}asBag\isactrlsub S\isactrlsub e\isactrlsub t{\isacharparenleft}{\isacharparenright}}} & {{ \isaFS{UML{\isacharunderscore}Library{\isachardot}OclAsBag\isactrlsub S\isactrlsub e\isactrlsub t}}\hideT{\text{\space\color{Black}\isaFS{const}}}}%
\\
%
&\footnotesize\inlineocl"_ ->asPair()"
& \hide{\color{Gray}($\text{\isaFS{logic}}^{\text{\color{GreenYellow}1000}}$)}\simpleArgs{$\text{\isaFS{logic}}^{\text{\color{GreenYellow}0}}$} \foclcolorbox{Apricot}{\isaFS{{\isacharminus}{\isachargreater}asPair\isactrlsub S\isactrlsub e\isactrlsub t{\isacharparenleft}{\isacharparenright}}} & {{ \isaFS{UML{\isacharunderscore}Library{\isachardot}OclAsPair\isactrlsub S\isactrlsub e\isactrlsub t}}\hideT{\text{\space\color{Black}\isaFS{const}}}}%
\\

  
&\footnotesize\inlineocl"_ ->sum()"
& \hide{\color{Gray}($\text{\isaFS{logic}}^{\text{\color{GreenYellow}1000}}$)}\simpleArgs{$\text{\isaFS{logic}}^{\text{\color{GreenYellow}0}}$} \foclcolorbox{Apricot}{\isaFS{{\isacharminus}{\isachargreater}sum\isactrlsub S\isactrlsub e\isactrlsub t{\isacharparenleft}{\isacharparenright}}} & {{ \isaFS{UML{\isacharunderscore}Set{\isachardot}OclSum}}\hideT{\text{\space\color{Black}\isaFS{const}}}}%
\\

%


%


%


%

&\footnotesize\inlineocl"_ ->excludesAll( _ )"
& \hide{\color{Gray}($\text{\isaFS{logic}}^{\text{\color{GreenYellow}1000}}$)}\simpleArgs{$\text{\isaFS{logic}}^{\text{\color{GreenYellow}0}}$} \foclcolorbox{Apricot}{\isaFS{{\isacharminus}{\isachargreater}excludesAll\isactrlsub S\isactrlsub e\isactrlsub t{\isacharparenleft}}} \simpleArgs{$\text{\isaFS{logic}}^{\text{\color{GreenYellow}0}}$} \foclcolorbox{Apricot}{\isaFS{{\isacharparenright}}} & {{ \isaFS{UML{\isacharunderscore}Set{\isachardot}OclExcludesAll}}\hideT{\text{\space\color{Black}\isaFS{const}}}}%
\\

%

&\footnotesize\inlineocl"_ ->includesAll( _ )"
& \hide{\color{Gray}($\text{\isaFS{logic}}^{\text{\color{GreenYellow}1000}}$)}\simpleArgs{$\text{\isaFS{logic}}^{\text{\color{GreenYellow}0}}$} \foclcolorbox{Apricot}{\isaFS{{\isacharminus}{\isachargreater}includesAll\isactrlsub S\isactrlsub e\isactrlsub t{\isacharparenleft}}} \simpleArgs{$\text{\isaFS{logic}}^{\text{\color{GreenYellow}0}}$} \foclcolorbox{Apricot}{\isaFS{{\isacharparenright}}} & {{ \isaFS{UML{\isacharunderscore}Set{\isachardot}OclIncludesAll}}\hideT{\text{\space\color{Black}\isaFS{const}}}}%
\\

%


%

&\footnotesize\inlineocl"_ ->any()"
& \hide{\color{Gray}($\text{\isaFS{logic}}^{\text{\color{GreenYellow}1000}}$)}\simpleArgs{$\text{\isaFS{logic}}^{\text{\color{GreenYellow}0}}$} \foclcolorbox{Apricot}{\isaFS{{\isacharminus}{\isachargreater}any\isactrlsub S\isactrlsub e\isactrlsub t{\isacharparenleft}{\isacharparenright}}} & {{ \isaFS{UML{\isacharunderscore}Set{\isachardot}OclANY}}\hideT{\text{\space\color{Black}\isaFS{const}}}}%
\\

%

&\footnotesize\inlineocl"_ ->notEmpty()"
& \hide{\color{Gray}($\text{\isaFS{logic}}^{\text{\color{GreenYellow}1000}}$)}\simpleArgs{$\text{\isaFS{logic}}^{\text{\color{GreenYellow}0}}$} \foclcolorbox{Apricot}{\isaFS{{\isacharminus}{\isachargreater}notEmpty\isactrlsub S\isactrlsub e\isactrlsub t{\isacharparenleft}{\isacharparenright}}} & {{ \isaFS{UML{\isacharunderscore}Set{\isachardot}OclNotEmpty}}\hideT{\text{\space\color{Black}\isaFS{const}}}}%
\\

%

&\footnotesize\inlineocl"_ ->isEmpty()"
& \hide{\color{Gray}($\text{\isaFS{logic}}^{\text{\color{GreenYellow}1000}}$)}\simpleArgs{$\text{\isaFS{logic}}^{\text{\color{GreenYellow}0}}$} \foclcolorbox{Apricot}{\isaFS{{\isacharminus}{\isachargreater}isEmpty\isactrlsub S\isactrlsub e\isactrlsub t{\isacharparenleft}{\isacharparenright}}} & {{ \isaFS{UML{\isacharunderscore}Set{\isachardot}OclIsEmpty}}\hideT{\text{\space\color{Black}\isaFS{const}}}}%
\\

%

&\footnotesize\inlineocl"_ ->size()"
& \hide{\color{Gray}($\text{\isaFS{logic}}^{\text{\color{GreenYellow}1000}}$)}\simpleArgs{$\text{\isaFS{logic}}^{\text{\color{GreenYellow}0}}$} \foclcolorbox{Apricot}{\isaFS{{\isacharminus}{\isachargreater}size\isactrlsub S\isactrlsub e\isactrlsub t{\isacharparenleft}{\isacharparenright}}} & {{ \isaFS{UML{\isacharunderscore}Set{\isachardot}OclSize}}\hideT{\text{\space\color{Black}\isaFS{const}}}}%
\\

%

&\footnotesize\inlineocl"_ ->excludes( _ )"
& \hide{\color{Gray}($\text{\isaFS{logic}}^{\text{\color{GreenYellow}1000}}$)}\simpleArgs{$\text{\isaFS{logic}}^{\text{\color{GreenYellow}0}}$} \foclcolorbox{Apricot}{\isaFS{{\isacharminus}{\isachargreater}excludes\isactrlsub S\isactrlsub e\isactrlsub t{\isacharparenleft}}} \simpleArgs{$\text{\isaFS{logic}}^{\text{\color{GreenYellow}0}}$} \foclcolorbox{Apricot}{\isaFS{{\isacharparenright}}} & {{ \isaFS{UML{\isacharunderscore}Set{\isachardot}OclExcludes}}\hideT{\text{\space\color{Black}\isaFS{const}}}}%
\\

%

&\footnotesize\inlineocl"_ ->includes( _ )"
& \hide{\color{Gray}($\text{\isaFS{logic}}^{\text{\color{GreenYellow}1000}}$)}\simpleArgs{$\text{\isaFS{logic}}^{\text{\color{GreenYellow}0}}$} \foclcolorbox{Apricot}{\isaFS{{\isacharminus}{\isachargreater}includes\isactrlsub S\isactrlsub e\isactrlsub t{\isacharparenleft}}} \simpleArgs{$\text{\isaFS{logic}}^{\text{\color{GreenYellow}0}}$} \foclcolorbox{Apricot}{\isaFS{{\isacharparenright}}} & {{ \isaFS{UML{\isacharunderscore}Set{\isachardot}OclIncludes}}\hideT{\text{\space\color{Black}\isaFS{const}}}}%
\\

%

%

\cmidrule{1-4}
  %%%%%%%%%%%%%%%%%%%%%%%%%%%%%%%%%%%%%%%%%%%%%%%%%%%%%%%%%%%%%%%%%%%%%%% 
  %%%% 11.7.2 Sequence
  %%%%%%%%%%%%%%%%%%%%%%%%%%%%%%%%%%%%%%%%%%%%%%%%%%%%%%%%%%%%%%%%%%%%%%%
\multirow{15}{*}{\rotatebox{90}{Sequence and Iterators on Sequence}}

&\footnotesize\inlineocl"Sequence ( _ )"
& \hide{\color{Gray}($\text{\isaFS{type}}^{\text{\color{GreenYellow}1000}}$)} \foclcolorbox{Apricot}{\isaFS{Sequence{\isacharparenleft}}} $\text{\isaFS{type}}^{\text{\color{GreenYellow}0}}$ \foclcolorbox{Apricot}{\isaFS{{\isacharparenright}}} & {{ \isaFS{UML{\isacharunderscore}Types{\isachardot}Sequence\isactrlsub b\isactrlsub a\isactrlsub s\isactrlsub e}}\text{\space\color{Black}\isaFS{type}}}%
  \\
&\footnotesize\inlineocl"Sequence{}"
       & \hide{\color{Gray}($\text{\isaFS{logic}}^{\text{\color{GreenYellow}1000}}$)} \foclcolorbox{Apricot}{\isaFS{Sequence{\isacharbraceleft}{\isacharbraceright}}} & {{ \isaFS{UML{\isacharunderscore}Sequence{\isachardot}mtSequence}}\hideT{\text{\space\color{Black}\isaFS{const}}}}%
\\

%
&\footnotesize\inlineocl"Sequence{ _ }"
& \hide{\color{Gray}($\text{\isaFS{logic}}^{\text{\color{GreenYellow}1000}}$)} \foclcolorbox{Apricot}{\isaFS{Sequence{\isacharbraceleft}}} $\text{\isaFS{args}}^{\text{\color{GreenYellow}0}}$ \foclcolorbox{Apricot}{\isaFS{{\isacharbraceright}}} & {{\color{Gray} \isaFS{OclFinsequence}}}%
\\

&\footnotesize\inlineocl"_ ->any()"
& \hide{\color{Gray}($\text{\isaFS{logic}}^{\text{\color{GreenYellow}1000}}$)}\simpleArgs{$\text{\isaFS{logic}}^{\text{\color{GreenYellow}0}}$} \foclcolorbox{Apricot}{\isaFS{{\isacharminus}{\isachargreater}any\isactrlsub S\isactrlsub e\isactrlsub q{\isacharparenleft}{\isacharparenright}}} & {{ \isaFS{UML{\isacharunderscore}Sequence{\isachardot}OclANY}}\hideT{\text{\space\color{Black}\isaFS{const}}}}%
\\

%

&\footnotesize\inlineocl"_ ->notEmpty()"
& \hide{\color{Gray}($\text{\isaFS{logic}}^{\text{\color{GreenYellow}1000}}$)}\simpleArgs{$\text{\isaFS{logic}}^{\text{\color{GreenYellow}0}}$} \foclcolorbox{Apricot}{\isaFS{{\isacharminus}{\isachargreater}notEmpty\isactrlsub S\isactrlsub e\isactrlsub q{\isacharparenleft}{\isacharparenright}}} & {{ \isaFS{UML{\isacharunderscore}Sequence{\isachardot}OclNotEmpty}}\hideT{\text{\space\color{Black}\isaFS{const}}}}%
\\

%

&\footnotesize\inlineocl"_ ->isEmpty()"
& \hide{\color{Gray}($\text{\isaFS{logic}}^{\text{\color{GreenYellow}1000}}$)}\simpleArgs{$\text{\isaFS{logic}}^{\text{\color{GreenYellow}0}}$} \foclcolorbox{Apricot}{\isaFS{{\isacharminus}{\isachargreater}isEmpty\isactrlsub S\isactrlsub e\isactrlsub q{\isacharparenleft}{\isacharparenright}}} & {{ \isaFS{UML{\isacharunderscore}Sequence{\isachardot}OclIsEmpty}}\hideT{\text{\space\color{Black}\isaFS{const}}}}%
\\

%

&\footnotesize\inlineocl"_ ->size()"
& \hide{\color{Gray}($\text{\isaFS{logic}}^{\text{\color{GreenYellow}1000}}$)}\simpleArgs{$\text{\isaFS{logic}}^{\text{\color{GreenYellow}0}}$} \foclcolorbox{Apricot}{\isaFS{{\isacharminus}{\isachargreater}size\isactrlsub S\isactrlsub e\isactrlsub q{\isacharparenleft}{\isacharparenright}}} & {{ \isaFS{UML{\isacharunderscore}Sequence{\isachardot}OclSize}}\hideT{\text{\space\color{Black}\isaFS{const}}}}%
\\

%

&\footnotesize\inlineocl"_ ->select( _ | _ )"
& \hide{\color{Gray}($\text{\isaFS{logic}}^{\text{\color{GreenYellow}1000}}$)}\simpleArgs{$\text{\isaFS{logic}}^{\text{\color{GreenYellow}0}}$} \foclcolorbox{Apricot}{\isaFS{{\isacharminus}{\isachargreater}select\isactrlsub S\isactrlsub e\isactrlsub q{\isacharparenleft}}} \fbox{$\text{\isaFS{id}}$} \foclcolorbox{Apricot}{\isaFS{{\isacharbar}}} \simpleArgs{$\text{\isaFS{logic}}^{\text{\color{GreenYellow}0}}$} \foclcolorbox{Apricot}{\isaFS{{\isacharparenright}}} & {{\color{Gray} \isaFS{OclSelectSeq}}}%
\\

%
&\footnotesize\inlineocl"_ ->collect( _ | _ )"
& \hide{\color{Gray}($\text{\isaFS{logic}}^{\text{\color{GreenYellow}1000}}$)}\simpleArgs{$\text{\isaFS{logic}}^{\text{\color{GreenYellow}0}}$} \foclcolorbox{Apricot}{\isaFS{{\isacharminus}{\isachargreater}collect\isactrlsub S\isactrlsub e\isactrlsub q{\isacharparenleft}}} \fbox{$\text{\isaFS{id}}$} \foclcolorbox{Apricot}{\isaFS{{\isacharbar}}} \simpleArgs{$\text{\isaFS{logic}}^{\text{\color{GreenYellow}0}}$} \foclcolorbox{Apricot}{\isaFS{{\isacharparenright}}} & {{\color{Gray} \isaFS{OclCollectSeq}}}%
\\

%
&\footnotesize\inlineocl"_ ->exists( _ | _ )"
& \hide{\color{Gray}($\text{\isaFS{logic}}^{\text{\color{GreenYellow}1000}}$)}\simpleArgs{$\text{\isaFS{logic}}^{\text{\color{GreenYellow}0}}$} \foclcolorbox{Apricot}{\isaFS{{\isacharminus}{\isachargreater}exists\isactrlsub S\isactrlsub e\isactrlsub q{\isacharparenleft}}} \fbox{$\text{\isaFS{id}}$} \foclcolorbox{Apricot}{\isaFS{{\isacharbar}}} \simpleArgs{$\text{\isaFS{logic}}^{\text{\color{GreenYellow}0}}$} \foclcolorbox{Apricot}{\isaFS{{\isacharparenright}}} & {{\color{Gray} \isaFS{OclExistSeq}}}%
\\

%
&\footnotesize\inlineocl"_ ->forAll( _ | _ )"
& \hide{\color{Gray}($\text{\isaFS{logic}}^{\text{\color{GreenYellow}1000}}$)}\simpleArgs{$\text{\isaFS{logic}}^{\text{\color{GreenYellow}0}}$} \foclcolorbox{Apricot}{\isaFS{{\isacharminus}{\isachargreater}forAll\isactrlsub S\isactrlsub e\isactrlsub q{\isacharparenleft}}} \fbox{$\text{\isaFS{id}}$} \foclcolorbox{Apricot}{\isaFS{{\isacharbar}}} \simpleArgs{$\text{\isaFS{logic}}^{\text{\color{GreenYellow}0}}$} \foclcolorbox{Apricot}{\isaFS{{\isacharparenright}}} & {{\color{Gray} \isaFS{OclForallSeq}}}%
\\

%
&\footnotesize\inlineocl"_ ->iterate( _ ; _ : _ = _ | _ )"
& \hide{\color{Gray}($\text{\isaFS{logic}}^{\text{\color{GreenYellow}1000}}$)}\simpleArgs{$\text{\isaFS{logic}}^{\text{\color{GreenYellow}0}}$} \foclcolorbox{Apricot}{\isaFS{{\isacharminus}{\isachargreater}iterate\isactrlsub S\isactrlsub e\isactrlsub q{\isacharparenleft}}} $\text{\isaFS{idt}}^{\text{\color{GreenYellow}0}}$ \foclcolorbox{Apricot}{\isaFS{{\isacharsemicolon}}} $\text{\isaFS{idt}}^{\text{\color{GreenYellow}0}}$ \foclcolorbox{Apricot}{\isaFS{{\isacharequal}}} $\text{\isaFS{any}}^{\text{\color{GreenYellow}0}}$ \foclcolorbox{Apricot}{\isaFS{{\isacharbar}}} $\text{\isaFS{any}}^{\text{\color{GreenYellow}0}}$ \foclcolorbox{Apricot}{\isaFS{{\isacharparenright}}} & {{\color{Gray} \isaFS{OclIterateSeq}}}%
\\

%
&\footnotesize\inlineocl"_ ->last()"
& \hide{\color{Gray}($\text{\isaFS{logic}}^{\text{\color{GreenYellow}1000}}$)}\simpleArgs{$\text{\isaFS{logic}}^{\text{\color{GreenYellow}0}}$} \foclcolorbox{Apricot}{\isaFS{{\isacharminus}{\isachargreater}last\isactrlsub S\isactrlsub e\isactrlsub q{\isacharparenleft}}} \simpleArgs{$\text{\isaFS{logic}}^{\text{\color{GreenYellow}0}}$} \foclcolorbox{Apricot}{\isaFS{{\isacharparenright}}} & {{ \isaFS{UML{\isacharunderscore}Sequence{\isachardot}OclLast}}\hideT{\text{\space\color{Black}\isaFS{const}}}}%
\\

%

&\footnotesize\inlineocl"_ ->first()"
& \hide{\color{Gray}($\text{\isaFS{logic}}^{\text{\color{GreenYellow}1000}}$)}\simpleArgs{$\text{\isaFS{logic}}^{\text{\color{GreenYellow}0}}$} \foclcolorbox{Apricot}{\isaFS{{\isacharminus}{\isachargreater}first\isactrlsub S\isactrlsub e\isactrlsub q{\isacharparenleft}}} \simpleArgs{$\text{\isaFS{logic}}^{\text{\color{GreenYellow}0}}$} \foclcolorbox{Apricot}{\isaFS{{\isacharparenright}}} & {{ \isaFS{UML{\isacharunderscore}Sequence{\isachardot}OclFirst}}\hideT{\text{\space\color{Black}\isaFS{const}}}}%
\\

%

&\footnotesize\inlineocl"_ ->at( _ )"
& \hide{\color{Gray}($\text{\isaFS{logic}}^{\text{\color{GreenYellow}1000}}$)}\simpleArgs{$\text{\isaFS{logic}}^{\text{\color{GreenYellow}0}}$} \foclcolorbox{Apricot}{\isaFS{{\isacharminus}{\isachargreater}at\isactrlsub S\isactrlsub e\isactrlsub q{\isacharparenleft}}} \simpleArgs{$\text{\isaFS{logic}}^{\text{\color{GreenYellow}0}}$} \foclcolorbox{Apricot}{\isaFS{{\isacharparenright}}} & {{ \isaFS{UML{\isacharunderscore}Sequence{\isachardot}OclAt}}\hideT{\text{\space\color{Black}\isaFS{const}}}}%
\\

%
&\footnotesize\inlineocl"_ ->union( _ )"
& \hide{\color{Gray}($\text{\isaFS{logic}}^{\text{\color{GreenYellow}1000}}$)}\simpleArgs{$\text{\isaFS{logic}}^{\text{\color{GreenYellow}0}}$} \foclcolorbox{Apricot}{\isaFS{{\isacharminus}{\isachargreater}union\isactrlsub S\isactrlsub e\isactrlsub q{\isacharparenleft}}} \simpleArgs{$\text{\isaFS{logic}}^{\text{\color{GreenYellow}0}}$} \foclcolorbox{Apricot}{\isaFS{{\isacharparenright}}} & {{ \isaFS{UML{\isacharunderscore}Sequence{\isachardot}OclUnion}}\hideT{\text{\space\color{Black}\isaFS{const}}}}%
\\

%

&\footnotesize\inlineocl"_ ->append( _ )"
& \hide{\color{Gray}($\text{\isaFS{logic}}^{\text{\color{GreenYellow}1000}}$)}\simpleArgs{$\text{\isaFS{logic}}^{\text{\color{GreenYellow}0}}$} \foclcolorbox{Apricot}{\isaFS{{\isacharminus}{\isachargreater}append\isactrlsub S\isactrlsub e\isactrlsub q{\isacharparenleft}}} \simpleArgs{$\text{\isaFS{logic}}^{\text{\color{GreenYellow}0}}$} \foclcolorbox{Apricot}{\isaFS{{\isacharparenright}}} & {{ \isaFS{UML{\isacharunderscore}Sequence{\isachardot}OclAppend}}\hideT{\text{\space\color{Black}\isaFS{const}}}}%
\\

%
&\footnotesize\inlineocl"_ ->excluding( _ )"
& \hide{\color{Gray}($\text{\isaFS{logic}}^{\text{\color{GreenYellow}1000}}$)}\simpleArgs{$\text{\isaFS{logic}}^{\text{\color{GreenYellow}0}}$} \foclcolorbox{Apricot}{\isaFS{{\isacharminus}{\isachargreater}excluding\isactrlsub S\isactrlsub e\isactrlsub q{\isacharparenleft}}} \simpleArgs{$\text{\isaFS{logic}}^{\text{\color{GreenYellow}0}}$} \foclcolorbox{Apricot}{\isaFS{{\isacharparenright}}} & {{ \isaFS{UML{\isacharunderscore}Sequence{\isachardot}OclExcluding}}\hideT{\text{\space\color{Black}\isaFS{const}}}}%
\\

%

&\footnotesize\inlineocl"_ ->including( _ )"
& \hide{\color{Gray}($\text{\isaFS{logic}}^{\text{\color{GreenYellow}1000}}$)}\simpleArgs{$\text{\isaFS{logic}}^{\text{\color{GreenYellow}0}}$} \foclcolorbox{Apricot}{\isaFS{{\isacharminus}{\isachargreater}including\isactrlsub S\isactrlsub e\isactrlsub q{\isacharparenleft}}} \simpleArgs{$\text{\isaFS{logic}}^{\text{\color{GreenYellow}0}}$} \foclcolorbox{Apricot}{\isaFS{{\isacharparenright}}} & {{ \isaFS{UML{\isacharunderscore}Sequence{\isachardot}OclIncluding}}\hideT{\text{\space\color{Black}\isaFS{const}}}}%
\\

%

&\footnotesize\inlineocl"_ ->prepend( _ )"
& \hide{\color{Gray}($\text{\isaFS{logic}}^{\text{\color{GreenYellow}1000}}$)}\simpleArgs{$\text{\isaFS{logic}}^{\text{\color{GreenYellow}0}}$} \foclcolorbox{Apricot}{\isaFS{{\isacharminus}{\isachargreater}prepend\isactrlsub S\isactrlsub e\isactrlsub q{\isacharparenleft}}} \simpleArgs{$\text{\isaFS{logic}}^{\text{\color{GreenYellow}0}}$} \foclcolorbox{Apricot}{\isaFS{{\isacharparenright}}} & {{ \isaFS{UML{\isacharunderscore}Sequence{\isachardot}OclPrepend}}\hideT{\text{\spae\color{Black}\isaFS{const}}}}%
\\

  %
&\footnotesize\inlineocl"_ ->asSet()"
& \hide{\color{Gray}($\text{\isaFS{logic}}^{\text{\color{GreenYellow}1000}}$)}\simpleArgs{$\text{\isaFS{logic}}^{\text{\color{GreenYellow}0}}$} \foclcolorbox{Apricot}{\isaFS{{\isacharminus}{\isachargreater}asSet\isactrlsub S\isactrlsub e\isactrlsub q{\isacharparenleft}{\isacharparenright}}} & {{ \isaFS{UML{\isacharunderscore}Library{\isachardot}OclAsSet\isactrlsub S\isactrlsub e\isactrlsub q}}\hideT{\text{\space\color{Black}\isaFS{const}}}}%
\\


%
&\footnotesize\inlineocl"_ ->asBag()"
& \hide{\color{Gray}($\text{\isaFS{logic}}^{\text{\color{GreenYellow}1000}}$)}\simpleArgs{$\text{\isaFS{logic}}^{\text{\color{GreenYellow}0}}$} \foclcolorbox{Apricot}{\isaFS{{\isacharminus}{\isachargreater}asBag\isactrlsub S\isactrlsub e\isactrlsub q{\isacharparenleft}{\isacharparenright}}} & {{ \isaFS{UML{\isacharunderscore}Library{\isachardot}OclAsBag\isactrlsub S\isactrlsub e\isactrlsub q}}\hideT{\text{\space\color{Black}\isaFS{const}}}}%
\\


%

&\footnotesize\inlineocl"_ ->asPair()"
& \hide{\color{Gray}($\text{\isaFS{logic}}^{\text{\color{GreenYellow}1000}}$)}\simpleArgs{$\text{\isaFS{logic}}^{\text{\color{GreenYellow}0}}$} \foclcolorbox{Apricot}{\isaFS{{\isacharminus}{\isachargreater}asPair\isactrlsub S\isactrlsub e\isactrlsub q{\isacharparenleft}{\isacharparenright}}} & {{ \isaFS{UML{\isacharunderscore}Library{\isachardot}OclAsPair\isactrlsub S\isactrlsub e\isactrlsub q}}\hideT{\text{\space\color{Black}\isaFS{const}}}}%
\\


  
%
%

\cmidrule{1-4}
  %%%%%%%%%%%%%%%%%%%%%%%%%%%%%%%%%%%%%%%%%%%%%%%%%%%%%%%%%%%%%%%%%%%%%%% 
  %%%% 11.7.3 Bag
  %%%%%%%%%%%%%%%%%%%%%%%%%%%%%%%%%%%%%%%%%%%%%%%%%%%%%%%%%%%%%%%%%%%%%%%
\multirow{15}{*}{\rotatebox{90}{Bag and Iterators on Bag}}
%

&\footnotesize\inlineocl"Bag ( _ )"
& \hide{\color{Gray}($\text{\isaFS{type}}^{\text{\color{GreenYellow}1000}}$)} \foclcolorbox{Apricot}{\isaFS{Bag{\isacharparenleft}}} $\text{\isaFS{type}}^{\text{\color{GreenYellow}0}}$ \foclcolorbox{Apricot}{\isaFS{{\isacharparenright}}} & {{ \isaFS{UML{\isacharunderscore}Types{\isachardot}Bag\isactrlsub b\isactrlsub a\isactrlsub s\isactrlsub e}}\text{\space\color{Black}\isaFS{type}}}%
\\

%
%
&\footnotesize\inlineocl"Bag{}"
& \hide{\color{Gray}($\text{\isaFS{logic}}^{\text{\color{GreenYellow}1000}}$)} \foclcolorbox{Apricot}{\isaFS{Bag{\isacharbraceleft}{\isacharbraceright}}} & {{ \isaFS{UML{\isacharunderscore}Bag{\isachardot}mtBag}}\hideT{\text{\space\color{Black}\isaFS{const}}}}%
\\

%
&\footnotesize\inlineocl"Bag{ _ }"
& \hide{\color{Gray}($\text{\isaFS{logic}}^{\text{\color{GreenYellow}1000}}$)} \foclcolorbox{Apricot}{\isaFS{Bag{\isacharbraceleft}}} $\text{\isaFS{args}}^{\text{\color{GreenYellow}0}}$ \foclcolorbox{Apricot}{\isaFS{{\isacharbraceright}}} & {{\color{Gray} \isaFS{OclFinbag}}}%
\\

  
&\footnotesize\inlineocl"_ ->sum()"
& \hide{\color{Gray}($\text{\isaFS{logic}}^{\text{\color{GreenYellow}1000}}$)}\simpleArgs{$\text{\isaFS{logic}}^{\text{\color{GreenYellow}0}}$} \foclcolorbox{Apricot}{\isaFS{{\isacharminus}{\isachargreater}sum\isactrlsub B\isactrlsub a\isactrlsub g{\isacharparenleft}{\isacharparenright}}} & {{ \isaFS{UML{\isacharunderscore}Bag{\isachardot}OclSum}}\hideT{\text{\space\color{Black}\isaFS{const}}}}%
\\

%

&\footnotesize\inlineocl"_ ->count( _ )"
& \hide{\color{Gray}($\text{\isaFS{logic}}^{\text{\color{GreenYellow}1000}}$)}\simpleArgs{$\text{\isaFS{logic}}^{\text{\color{GreenYellow}0}}$} \foclcolorbox{Apricot}{\isaFS{{\isacharminus}{\isachargreater}count\isactrlsub B\isactrlsub a\isactrlsub g{\isacharparenleft}}} \simpleArgs{$\text{\isaFS{logic}}^{\text{\color{GreenYellow}0}}$} \foclcolorbox{Apricot}{\isaFS{{\isacharparenright}}} & {{ \isaFS{UML{\isacharunderscore}Bag{\isachardot}OclCount}}\hideT{\text{\space\color{Black}\isaFS{const}}}}%
\\

%

&\footnotesize\inlineocl"_ ->intersection( _ )"
& \hide{\color{Gray}($\text{\isaFS{logic}}^{\text{\color{GreenYellow}1000}}$)}\simpleArgs{$\text{\isaFS{logic}}^{\text{\color{GreenYellow}0}}$} \foclcolorbox{Apricot}{\isaFS{{\isacharminus}{\isachargreater}intersection\isactrlsub B\isactrlsub a\isactrlsub g{\isacharparenleft}}} \simpleArgs{$\text{\isaFS{logic}}^{\text{\color{GreenYellow}0}}$} \foclcolorbox{Apricot}{\isaFS{{\isacharparenright}}} & {{ \isaFS{UML{\isacharunderscore}Bag{\isachardot}OclIntersection}}\hideT{\text{\space\color{Black}\isaFS{const}}}}%
\\

  
%

&\footnotesize\inlineocl"_ ->union( _ )"
& \hide{\color{Gray}($\text{\isaFS{logic}}^{\text{\color{GreenYellow}1000}}$)}\simpleArgs{$\text{\isaFS{logic}}^{\text{\color{GreenYellow}0}}$} \foclcolorbox{Apricot}{\isaFS{{\isacharminus}{\isachargreater}union\isactrlsub B\isactrlsub a\isactrlsub g{\isacharparenleft}}} \simpleArgs{$\text{\isaFS{logic}}^{\text{\color{GreenYellow}0}}$} \foclcolorbox{Apricot}{\isaFS{{\isacharparenright}}} & {{ \isaFS{UML{\isacharunderscore}Bag{\isachardot}OclUnion}}\hideT{\text{\space\color{Black}\isaFS{const}}}}%
\\

%

&\footnotesize\inlineocl"_ ->excludesAll( _ )"
& \hide{\color{Gray}($\text{\isaFS{logic}}^{\text{\color{GreenYellow}1000}}$)}\simpleArgs{$\text{\isaFS{logic}}^{\text{\color{GreenYellow}0}}$} \foclcolorbox{Apricot}{\isaFS{{\isacharminus}{\isachargreater}excludesAll\isactrlsub B\isactrlsub a\isactrlsub g{\isacharparenleft}}} \simpleArgs{$\text{\isaFS{logic}}^{\text{\color{GreenYellow}0}}$} \foclcolorbox{Apricot}{\isaFS{{\isacharparenright}}} & {{ \isaFS{UML{\isacharunderscore}Bag{\isachardot}OclExcludesAll}}\hideT{\text{\space\color{Black}\isaFS{const}}}}%
\\

%

&\footnotesize\inlineocl"_ ->includesAll( _ )"
& \hide{\color{Gray}($\text{\isaFS{logic}}^{\text{\color{GreenYellow}1000}}$)}\simpleArgs{$\text{\isaFS{logic}}^{\text{\color{GreenYellow}0}}$} \foclcolorbox{Apricot}{\isaFS{{\isacharminus}{\isachargreater}includesAll\isactrlsub B\isactrlsub a\isactrlsub g{\isacharparenleft}}} \simpleArgs{$\text{\isaFS{logic}}^{\text{\color{GreenYellow}0}}$} \foclcolorbox{Apricot}{\isaFS{{\isacharparenright}}} & {{ \isaFS{UML{\isacharunderscore}Bag{\isachardot}OclIncludesAll}}\hideT{\text{\space\color{Black}\isaFS{const}}}}%
\\

%

&\footnotesize\inlineocl"_ ->reject( _ | _ )"
& \hide{\color{Gray}($\text{\isaFS{logic}}^{\text{\color{GreenYellow}1000}}$)}\simpleArgs{$\text{\isaFS{logic}}^{\text{\color{GreenYellow}0}}$} \foclcolorbox{Apricot}{\isaFS{{\isacharminus}{\isachargreater}reject\isactrlsub B\isactrlsub a\isactrlsub g{\isacharparenleft}}} \fbox{$\text{\isaFS{id}}$} \foclcolorbox{Apricot}{\isaFS{{\isacharbar}}} \simpleArgs{$\text{\isaFS{logic}}^{\text{\color{GreenYellow}0}}$} \foclcolorbox{Apricot}{\isaFS{{\isacharparenright}}} & {{\color{Gray} \isaFS{OclRejectBag}}}%
\\

%

&\footnotesize\inlineocl"_ ->select( _ | _ )"
& \hide{\color{Gray}($\text{\isaFS{logic}}^{\text{\color{GreenYellow}1000}}$)}\simpleArgs{$\text{\isaFS{logic}}^{\text{\color{GreenYellow}0}}$} \foclcolorbox{Apricot}{\isaFS{{\isacharminus}{\isachargreater}select\isactrlsub B\isactrlsub a\isactrlsub g{\isacharparenleft}}} \fbox{$\text{\isaFS{id}}$} \foclcolorbox{Apricot}{\isaFS{{\isacharbar}}} \simpleArgs{$\text{\isaFS{logic}}^{\text{\color{GreenYellow}0}}$} \foclcolorbox{Apricot}{\isaFS{{\isacharparenright}}} & {{\color{Gray} \isaFS{OclSelectBag}}}%
\\

%

&\footnotesize\inlineocl"_ ->iterate( _ ; _ = _ | _ )"
& \hide{\color{Gray}($\text{\isaFS{logic}}^{\text{\color{GreenYellow}1000}}$)}\simpleArgs{$\text{\isaFS{logic}}^{\text{\color{GreenYellow}0}}$} \foclcolorbox{Apricot}{\isaFS{{\isacharminus}{\isachargreater}iterate\isactrlsub B\isactrlsub a\isactrlsub g{\isacharparenleft}}} $\text{\isaFS{idt}}^{\text{\color{GreenYellow}0}}$ \foclcolorbox{Apricot}{\isaFS{{\isacharsemicolon}}} $\text{\isaFS{idt}}^{\text{\color{GreenYellow}0}}$ \foclcolorbox{Apricot}{\isaFS{{\isacharequal}}} $\text{\isaFS{any}}^{\text{\color{GreenYellow}0}}$ \foclcolorbox{Apricot}{\isaFS{{\isacharbar}}} $\text{\isaFS{any}}^{\text{\color{GreenYellow}0}}$ \foclcolorbox{Apricot}{\isaFS{{\isacharparenright}}} & {{\color{Gray} \isaFS{OclIterateBag}}}%
\\

%

&\footnotesize\inlineocl"_ ->exists( _ | _ )"
& \hide{\color{Gray}($\text{\isaFS{logic}}^{\text{\color{GreenYellow}1000}}$)}\simpleArgs{$\text{\isaFS{logic}}^{\text{\color{GreenYellow}0}}$} \foclcolorbox{Apricot}{\isaFS{{\isacharminus}{\isachargreater}exists\isactrlsub B\isactrlsub a\isactrlsub g{\isacharparenleft}}} \fbox{$\text{\isaFS{id}}$} \foclcolorbox{Apricot}{\isaFS{{\isacharbar}}} \simpleArgs{$\text{\isaFS{logic}}^{\text{\color{GreenYellow}0}}$} \foclcolorbox{Apricot}{\isaFS{{\isacharparenright}}} & {{\color{Gray} \isaFS{OclExistBag}}}%
\\

%

&\footnotesize\inlineocl"_ ->forAll( _ | _ )"
& \hide{\color{Gray}($\text{\isaFS{logic}}^{\text{\color{GreenYellow}1000}}$)}\simpleArgs{$\text{\isaFS{logic}}^{\text{\color{GreenYellow}0}}$} \foclcolorbox{Apricot}{\isaFS{{\isacharminus}{\isachargreater}forAll\isactrlsub B\isactrlsub a\isactrlsub g{\isacharparenleft}}} \fbox{$\text{\isaFS{id}}$} \foclcolorbox{Apricot}{\isaFS{{\isacharbar}}} \simpleArgs{$\text{\isaFS{logic}}^{\text{\color{GreenYellow}0}}$} \foclcolorbox{Apricot}{\isaFS{{\isacharparenright}}} & {{\color{Gray} \isaFS{OclForallBag}}}%
\\

%

&\footnotesize\inlineocl"_ ->any()"
& \hide{\color{Gray}($\text{\isaFS{logic}}^{\text{\color{GreenYellow}1000}}$)}\simpleArgs{$\text{\isaFS{logic}}^{\text{\color{GreenYellow}0}}$} \foclcolorbox{Apricot}{\isaFS{{\isacharminus}{\isachargreater}any\isactrlsub B\isactrlsub a\isactrlsub g{\isacharparenleft}{\isacharparenright}}} & {{ \isaFS{UML{\isacharunderscore}Bag{\isachardot}OclANY}}\hideT{\text{\space\color{Black}\isaFS{const}}}}%
\\

%

&\footnotesize\inlineocl"_ ->notEmpty()"
& \hide{\color{Gray}($\text{\isaFS{logic}}^{\text{\color{GreenYellow}1000}}$)}\simpleArgs{$\text{\isaFS{logic}}^{\text{\color{GreenYellow}0}}$} \foclcolorbox{Apricot}{\isaFS{{\isacharminus}{\isachargreater}notEmpty\isactrlsub B\isactrlsub a\isactrlsub g{\isacharparenleft}{\isacharparenright}}} & {{ \isaFS{UML{\isacharunderscore}Bag{\isachardot}OclNotEmpty}}\hideT{\text{\space\color{Black}\isaFS{const}}}}%
\\

%

&\footnotesize\inlineocl"_ ->isEmpty()"
& \hide{\color{Gray}($\text{\isaFS{logic}}^{\text{\color{GreenYellow}1000}}$)}\simpleArgs{$\text{\isaFS{logic}}^{\text{\color{GreenYellow}0}}$} \foclcolorbox{Apricot}{\isaFS{{\isacharminus}{\isachargreater}isEmpty\isactrlsub B\isactrlsub a\isactrlsub g{\isacharparenleft}{\isacharparenright}}} & {{ \isaFS{UML{\isacharunderscore}Bag{\isachardot}OclIsEmpty}}\hideT{\text{\space\color{Black}\isaFS{const}}}}%
\\

%

&\footnotesize\inlineocl"_ ->size()"
& \hide{\color{Gray}($\text{\isaFS{logic}}^{\text{\color{GreenYellow}1000}}$)}\simpleArgs{$\text{\isaFS{logic}}^{\text{\color{GreenYellow}0}}$} \foclcolorbox{Apricot}{\isaFS{{\isacharminus}{\isachargreater}size\isactrlsub B\isactrlsub a\isactrlsub g{\isacharparenleft}{\isacharparenright}}} & {{ \isaFS{UML{\isacharunderscore}Bag{\isachardot}OclSize}}\hideT{\text{\space\color{Black}\isaFS{const}}}}%
\\

%

&\footnotesize\inlineocl"_ ->excludes( _ )"
& \hide{\color{Gray}($\text{\isaFS{logic}}^{\text{\color{GreenYellow}1000}}$)}\simpleArgs{$\text{\isaFS{logic}}^{\text{\color{GreenYellow}0}}$} \foclcolorbox{Apricot}{\isaFS{{\isacharminus}{\isachargreater}excludes\isactrlsub B\isactrlsub a\isactrlsub g{\isacharparenleft}}} \simpleArgs{$\text{\isaFS{logic}}^{\text{\color{GreenYellow}0}}$} \foclcolorbox{Apricot}{\isaFS{{\isacharparenright}}} & {{ \isaFS{UML{\isacharunderscore}Bag{\isachardot}OclExcludes}}\hideT{\text{\space\color{Black}\isaFS{const}}}}%
\\

%

&\footnotesize\inlineocl"_ ->includes( _ )"
& \hide{\color{Gray}($\text{\isaFS{logic}}^{\text{\color{GreenYellow}1000}}$)}\simpleArgs{$\text{\isaFS{logic}}^{\text{\color{GreenYellow}0}}$} \foclcolorbox{Apricot}{\isaFS{{\isacharminus}{\isachargreater}includes\isactrlsub B\isactrlsub a\isactrlsub g{\isacharparenleft}}} \simpleArgs{$\text{\isaFS{logic}}^{\text{\color{GreenYellow}0}}$} \foclcolorbox{Apricot}{\isaFS{{\isacharparenright}}} & {{ \isaFS{UML{\isacharunderscore}Bag{\isachardot}OclIncludes}}\hideT{\text{\space\color{Black}\isaFS{const}}}}%
\\

%

&\footnotesize\inlineocl"_ ->excluding( _ )"
& \hide{\color{Gray}($\text{\isaFS{logic}}^{\text{\color{GreenYellow}1000}}$)}\simpleArgs{$\text{\isaFS{logic}}^{\text{\color{GreenYellow}0}}$} \foclcolorbox{Apricot}{\isaFS{{\isacharminus}{\isachargreater}excluding\isactrlsub B\isactrlsub a\isactrlsub g{\isacharparenleft}}} \simpleArgs{$\text{\isaFS{logic}}^{\text{\color{GreenYellow}0}}$} \foclcolorbox{Apricot}{\isaFS{{\isacharparenright}}} & {{ \isaFS{UML{\isacharunderscore}Bag{\isachardot}OclExcluding}}\hideT{\text{\space\color{Black}\isaFS{const}}}}%
\\

%

&\footnotesize\inlineocl"_ ->including( _ )"
& \hide{\color{Gray}($\text{\isaFS{logic}}^{\text{\color{GreenYellow}1000}}$)}\simpleArgs{$\text{\isaFS{logic}}^{\text{\color{GreenYellow}0}}$} \foclcolorbox{Apricot}{\isaFS{{\isacharminus}{\isachargreater}including\isactrlsub B\isactrlsub a\isactrlsub g{\isacharparenleft}}} \simpleArgs{$\text{\isaFS{logic}}^{\text{\color{GreenYellow}0}}$} \foclcolorbox{Apricot}{\isaFS{{\isacharparenright}}} & {{ \isaFS{UML{\isacharunderscore}Bag{\isachardot}OclIncluding}}\hideT{\text{\space\color{Black}\isaFS{const}}}}%
\\

  %
&\footnotesize\inlineocl"_ ->asSet()"
& \hide{\color{Gray}($\text{\isaFS{logic}}^{\text{\color{GreenYellow}1000}}$)}\simpleArgs{$\text{\isaFS{logic}}^{\text{\color{GreenYellow}0}}$} \foclcolorbox{Apricot}{\isaFS{{\isacharminus}{\isachargreater}asSet\isactrlsub B\isactrlsub a\isactrlsub g{\isacharparenleft}{\isacharparenright}}} & {{ \isaFS{UML{\isacharunderscore}Library{\isachardot}OclAsSet\isactrlsub B\isactrlsub a\isactrlsub g}}\hideT{\text{\space\color{Black}\isaFS{const}}}}%
\\
  %
&\footnotesize\inlineocl"_ ->asSeq()"
& \hide{\color{Gray}($\text{\isaFS{logic}}^{\text{\color{GreenYellow}1000}}$)}\simpleArgs{$\text{\isaFS{logic}}^{\text{\color{GreenYellow}0}}$} \foclcolorbox{Apricot}{\isaFS{{\isacharminus}{\isachargreater}asSeq\isactrlsub B\isactrlsub a\isactrlsub g{\isacharparenleft}{\isacharparenright}}} & {{ \isaFS{UML{\isacharunderscore}Library{\isachardot}OclAsSeq\isactrlsub B\isactrlsub a\isactrlsub g}}\hideT{\text{\space\color{Black}\isaFS{const}}}}%
\\
  %
&\footnotesize\inlineocl"_ ->asPair()"
& \hide{\color{Gray}($\text{\isaFS{logic}}^{\text{\color{GreenYellow}1000}}$)}\simpleArgs{$\text{\isaFS{logic}}^{\text{\color{GreenYellow}0}}$} \foclcolorbox{Apricot}{\isaFS{{\isacharminus}{\isachargreater}asPair\isactrlsub B\isactrlsub a\isactrlsub g{\isacharparenleft}{\isacharparenright}}} & {{ \isaFS{UML{\isacharunderscore}Library{\isachardot}OclAsPair\isactrlsub B\isactrlsub a\isactrlsub g}}\hideT{\text{\space\color{Black}\isaFS{const}}}}%
\\

\cmidrule{1-4}
  %%%%%%%%%%%%%%%%%%%%%%%%%%%%%%%%%%%%%%%%%%%%%%%%%%%%%%%%%%%%%%%%%%%%%%% 
  %%%% Pair
  %%%%%%%%%%%%%%%%%%%%%%%%%%%%%%%%%%%%%%%%%%%%%%%%%%%%%%%%%%%%%%%%%%%%%%%
\multirow{3}{*}{\rotatebox{90}{Pair}}

&\footnotesize\inlineocl""
& \hide{\color{Gray}($\text{\isaFS{type}}^{\text{\color{GreenYellow}1000}}$)} \foclcolorbox{Apricot}{\isaFS{Pair{\isacharparenleft}}} $\text{\isaFS{type}}^{\text{\color{GreenYellow}0}}$ \foclcolorbox{Apricot}{\isaFS{{\isacharcomma}}} $\text{\isaFS{type}}^{\text{\color{GreenYellow}0}}$ \foclcolorbox{Apricot}{\isaFS{{\isacharparenright}}} & {{ \isaFS{UML{\isacharunderscore}Types{\isachardot}Pair\isactrlsub b\isactrlsub a\isactrlsub s\isactrlsub e}}\text{\space\color{Black}\isaFS{type}}}%
\\

  %
&
& \hide{\color{Gray}($\text{\isaFS{logic}}^{\text{\color{GreenYellow}1000}}$)} \foclcolorbox{Apricot}{\isaFS{Pair{\isacharbraceleft}}}\simpleArgs{$\text{\isaFS{logic}}^{\text{\color{GreenYellow}0}}$} \foclcolorbox{Apricot}{\isaFS{{\isacharcomma}}} \simpleArgs{$\text{\isaFS{logic}}^{\text{\color{GreenYellow}0}}$} \foclcolorbox{Apricot}{\isaFS{{\isacharbraceright}}} & {{ \isaFS{UML{\isacharunderscore}Pair{\isachardot}OclPair}}\hideT{\text{\space\color{Black}\isaFS{const}}}}%
\\

%
&
& \hide{\color{Gray}($\text{\isaFS{logic}}^{\text{\color{GreenYellow}1000}}$)}\simpleArgs{$\text{\isaFS{logic}}^{\text{\color{GreenYellow}0}}$} \foclcolorbox{Apricot}{\isaFS{{\isachardot}Second{\isacharparenleft}{\isacharparenright}}} & {{ \isaFS{UML{\isacharunderscore}Pair{\isachardot}OclSecond}}\hideT{\text{\space\color{Black}\isaFS{const}}}}%
\\

%
&
& \hide{\color{Gray}($\text{\isaFS{logic}}^{\text{\color{GreenYellow}1000}}$)}\simpleArgs{$\text{\isaFS{logic}}^{\text{\color{GreenYellow}0}}$} \foclcolorbox{Apricot}{\isaFS{{\isachardot}First{\isacharparenleft}{\isacharparenright}}} & {{ \isaFS{UML{\isacharunderscore}Pair{\isachardot}OclFirst}}\hideT{\text{\space\color{Black}\isaFS{const}}}}%
\\

%
%
&\footnotesize\inlineocl"_ ->asSequence()"
& \hide{\color{Gray}($\text{\isaFS{logic}}^{\text{\color{GreenYellow}1000}}$)}\simpleArgs{$\text{\isaFS{logic}}^{\text{\color{GreenYellow}0}}$} \foclcolorbox{Apricot}{\isaFS{{\isacharminus}{\isachargreater}asSequence\isactrlsub P\isactrlsub a\isactrlsub i\isactrlsub r{\isacharparenleft}{\isacharparenright}}} & {{ \isaFS{UML{\isacharunderscore}Library{\isachardot}OclAsSeq\isactrlsub P\isactrlsub a\isactrlsub i\isactrlsub r}}\hideT{\text{\space\color{Black}\isaFS{const}}}}%
\\


%
&\footnotesize\inlineocl"_ ->asSet()"
& \hide{\color{Gray}($\text{\isaFS{logic}}^{\text{\color{GreenYellow}1000}}$)}\simpleArgs{$\text{\isaFS{logic}}^{\text{\color{GreenYellow}0}}$} \foclcolorbox{Apricot}{\isaFS{{\isacharminus}{\isachargreater}asSet\isactrlsub P\isactrlsub a\isactrlsub i\isactrlsub r{\isacharparenleft}{\isacharparenright}}} & {{ \isaFS{UML{\isacharunderscore}Library{\isachardot}OclAsSet\isactrlsub P\isactrlsub a\isactrlsub i\isactrlsub r}}\hideT{\text{\space\color{Black}\isaFS{const}}}}%
\\


  \cmidrule{1-4}
  %%%%%%%%%%%%%%%%%%%%%%%%%%%%%%%%%%%%%%%%%%%%%%%%%%%%%%%%%%%%%%%%%%%%%%% 
  %%%% Pair
  %%%%%%%%%%%%%%%%%%%%%%%%%%%%%%%%%%%%%%%%%%%%%%%%%%%%%%%%%%%%%%%%%%%%%%%
\multirow{3}{*}{\rotatebox{90}{State Access}}

&\footnotesize\inlineocl"_ .allInstances()"
& \hide{\color{Gray}($\text{\isaFS{logic}}^{\text{\color{GreenYellow}1000}}$)}\simpleArgs{$\text{\isaFS{logic}}^{\text{\color{GreenYellow}0}}$} \foclcolorbox{Apricot}{\isaFS{{\isachardot}allInstances{\isacharparenleft}{\isacharparenright}}} & {{ \isaFS{UML{\isacharunderscore}State{\isachardot}OclAllInstances{\isacharunderscore}at{\isacharunderscore}post}}\hideT{\text{\space\color{Black}\isaFS{const}}}}%
\\

%
&
& \hide{\color{Gray}($\text{\isaFS{logic}}^{\text{\color{GreenYellow}1000}}$)}\simpleArgs{$\text{\isaFS{logic}}^{\text{\color{GreenYellow}0}}$} \foclcolorbox{Apricot}{\isaFS{{\isachardot}allInstances{\isacharat}pre{\isacharparenleft}{\isacharparenright}}} & {{ \isaFS{UML{\isacharunderscore}State{\isachardot}OclAllInstances{\isacharunderscore}at{\isacharunderscore}pre}}\hideT{\text{\space\color{Black}\isaFS{const}}}}%
\\

%


%

&
& \hide{\color{Gray}($\text{\isaFS{logic}}^{\text{\color{GreenYellow}1000}}$)}\simpleArgs{$\text{\isaFS{logic}}^{\text{\color{GreenYellow}0}}$} \foclcolorbox{Apricot}{\isaFS{{\isachardot}oclIsDeleted{\isacharparenleft}{\isacharparenright}}} & {{ \isaFS{UML{\isacharunderscore}State{\isachardot}OclIsDeleted}}\hideT{\text{\space\color{Black}\isaFS{const}}}}%
\\

%

&
& \hide{\color{Gray}($\text{\isaFS{logic}}^{\text{\color{GreenYellow}1000}}$)}\simpleArgs{$\text{\isaFS{logic}}^{\text{\color{GreenYellow}0}}$} \foclcolorbox{Apricot}{\isaFS{{\isachardot}oclIsMaintained{\isacharparenleft}{\isacharparenright}}} & {{ \isaFS{UML{\isacharunderscore}State{\isachardot}OclIsMaintained}}\hideT{\text{\space\color{Black}\isaFS{const}}}}%
\\

%

&
& \hide{\color{Gray}($\text{\isaFS{logic}}^{\text{\color{GreenYellow}1000}}$)}\simpleArgs{$\text{\isaFS{logic}}^{\text{\color{GreenYellow}0}}$} \foclcolorbox{Apricot}{\isaFS{{\isachardot}oclIsAbsent{\isacharparenleft}{\isacharparenright}}} & {{ \isaFS{UML{\isacharunderscore}State{\isachardot}OclIsAbsent}}\hideT{\text{\space\color{Black}\isaFS{const}}}}%
\\

%
&
& \hide{\color{Gray}($\text{\isaFS{logic}}^{\text{\color{GreenYellow}1000}}$)}\simpleArgs{$\text{\isaFS{logic}}^{\text{\color{GreenYellow}0}}$} \foclcolorbox{Apricot}{\isaFS{{\isacharminus}{\isachargreater}oclIsModifiedOnly{\isacharparenleft}{\isacharparenright}}} & {{ \isaFS{UML{\isacharunderscore}State{\isachardot}OclIsModifiedOnly}}\hideT{\text{\space\color{Black}\isaFS{const}}}}%
\\

%

&\footnotesize\inlineocl"_ @pre _"
& \hide{\color{Gray}($\text{\isaFS{logic}}^{\text{\color{GreenYellow}1000}}$)}\simpleArgs{$\text{\isaFS{logic}}^{\text{\color{GreenYellow}0}}$} \foclcolorbox{Apricot}{\isaFS{{\isacharat}pre}} \simpleArgs{$\text{\isaFS{logic}}^{\text{\color{GreenYellow}0}}$} & {{ \isaFS{UML{\isacharunderscore}State{\isachardot}OclSelf{\isacharunderscore}at{\isacharunderscore}pre}}\hideT{\text{\space\color{Black}\isaFS{const}}}}%
\\

%
&
& \hide{\color{Gray}($\text{\isaFS{logic}}^{\text{\color{GreenYellow}1000}}$)}\simpleArgs{$\text{\isaFS{logic}}^{\text{\color{GreenYellow}0}}$} \foclcolorbox{Apricot}{\isaFS{{\isacharat}post}} \simpleArgs{$\text{\isaFS{logic}}^{\text{\color{GreenYellow}0}}$} & {{ \isaFS{UML{\isacharunderscore}State{\isachardot}OclSelf{\isacharunderscore}at{\isacharunderscore}post}}\hideT{\text{\space\color{Black}\isaFS{const}}}}%
\\



  \cmidrule{1-4}
  
%%%%  
%%%%
%%%%  Other Stuff
%%%%

%  \multirow{7}{*}{\rotatebox{90}{Unsorted}}

%
% &
% & \hide{\color{Gray}($\text{\isaFS{logic}}^{\text{\color{GreenYellow}1000}}$)} \foclcolorbox{Apricot}{\isaFS{{\isasymlceil}}}\simpleArgs{$\text{\isaFS{logic}}^{\text{\color{GreenYellow}0}}$} \foclcolorbox{Apricot}{\isaFS{{\isasymrceil}}} & {{ \isaFS{UML{\isacharunderscore}Types{\isachardot}drop}}\hideT{\text{\space\color{Black}\isaFS{const}}}}%
% \\
% %
% &
% & \hide{\color{Gray}($\text{\isaFS{logic}}^{\text{\color{GreenYellow}1000}}$)} \foclcolorbox{Apricot}{\isaFS{I{\isasymlbrakk}}} $\text{\isaFS{any}}^{\text{\color{GreenYellow}0}}$ \foclcolorbox{Apricot}{\isaFS{{\isasymrbrakk}}} & {{ \isaFS{UML{\isacharunderscore}Types{\isachardot}Sem}}\hideT{\text{\space\color{Black}\isaFS{const}}}}%
% \\


% %
% &
% & \hide{\color{Gray}($\text{\isaFS{logic}}^{\text{\color{GreenYellow}1000}}$)} \foclcolorbox{Apricot}{\isaFS{{\isasymbottom}}} & {{ \isaFS{UML{\isacharunderscore}Types{\isachardot}bot{\isacharunderscore}class{\isachardot}bot}}\hideT{\text{\space\color{Black}\isaFS{const}}}}%
% \\

% %

  
% %
% &
% & \hide{\color{Gray}($\text{\isaFS{logic}}^{\text{\color{GreenYellow}1000}}$)} \foclcolorbox{Apricot}{\isaFS{{\isasymbottom}}} & {{ \isaFS{Option{\isachardot}option{\isachardot}None}}\hideT{\text{\space\color{Black}\isaFS{const}}}}%
% \\




% %


% %


% %

% %
% % & \hide{\color{Gray}($\text{\isaFS{logic}}^{\text{\color{GreenYellow}1000}}$)} $\text{\isaFS{cartouche{\isacharunderscore}position}}^{\text{\color{GreenYellow}0}}$ & {{\color{Gray} \isaFS{cartouche{\isacharunderscore}oclstring}}}%
% % \\

% %
% &
% & \hide{\color{Gray}($\text{\isaFS{logic}}^{\text{\color{GreenYellow}1000}}$)} \foclcolorbox{Apricot}{\isaFS{{\isacharunderscore}{\isacharprime}}} & {{\color{Gray} \isaFS{ocl{\isacharunderscore}denotation}}}%
% \\


% %
% %

% %
% &
% & \hide{\color{Gray}($\text{\isaFS{type}}^{\text{\color{GreenYellow}1000}}$)} \foclcolorbox{Apricot}{\isaFS{{\isasymlangle}}} $\text{\isaFS{type}}^{\text{\color{GreenYellow}0}}$ \foclcolorbox{Apricot}{\isaFS{{\isasymrangle}\isactrlsub {\isasymbottom}}} & {{ \isaFS{Option{\isachardot}option}}\text{\space\color{Black}\isaFS{type}}}%
% \\

%


  
%




  
%%%%%%%%%%%%%%%%%%%%%%%%%%%%%%%%%%%%%%%%%%%%%%%%%%%%%%

  
%



  
  

  
\end{longtable}
}

%%% Local Variables:
%%% fill-column:80
%%% x-symbol-8bits:nil
%%% mode: latex
%%% TeX-master: "syntax_main"
%%% End:


\isatagannexa
  \part{Table of Contents}
  \clearpage {\small \tableofcontents }
\endisatagannexa
\end{document}

%%% Local Variables:
%%% mode: latex
%%% TeX-master: t
%%% End:

%  LocalWords:  implementors denotational OCL UML
