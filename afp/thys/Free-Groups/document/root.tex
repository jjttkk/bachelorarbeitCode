\documentclass[11pt,a4paper]{article}
\usepackage[T1]{fontenc}
\usepackage{isabelle,isabellesym}
\usepackage{amssymb}
\usepackage[english]{babel}
\usepackage[only,bigsqcap]{stmaryrd}
\usepackage{textcomp}

% this should be the last package used
\usepackage{pdfsetup}

% urls in roman style, theory text in math-similar italics
\urlstyle{rm}
\isabellestyle{it}


\begin{document}

\title{Free Groups}
\author{Joachim Breitner}
\maketitle

\begin{abstract}
  Free Groups are, in a sense, the most generic kind of group. They are defined
  over a set of generators with no additional relations in between them. They
  play an important role in the definition of group presentations and in
  other fields.

  This theory provides the definition of Free Group as the set of fully
  canceled words in the generators. The universal property is proven, as well as some
  isomorphisms results about Free Groups.
\end{abstract}

\tableofcontents

% include generated text of all theories
\input{session}

\end{document}
