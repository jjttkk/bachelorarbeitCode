\documentclass[11pt,a4paper]{article}
\usepackage[T1]{fontenc}
\usepackage{isabelle,isabellesym}
\usepackage{amsmath}

% further packages required for unusual symbols (see also
% isabellesym.sty), use only when needed

%\usepackage{amssymb}
  %for \<leadsto>, \<box>, \<diamond>, \<sqsupset>, \<mho>, \<Join>,
  %\<lhd>, \<lesssim>, \<greatersim>, \<lessapprox>, \<greaterapprox>,
  %\<triangleq>, \<yen>, \<lozenge>

%\usepackage{eurosym}
  %for \<euro>

%\usepackage[only,bigsqcap,bigparallel,fatsemi,interleave,sslash]{stmaryrd}
  %for \<Sqinter>, \<Parallel>, \<Zsemi>, \<Parallel>, \<sslash>

%\usepackage{eufrak}
  %for \<AA> ... \<ZZ>, \<aa> ... \<zz> (also included in amssymb)

%\usepackage{textcomp}
  %for \<onequarter>, \<onehalf>, \<threequarters>, \<degree>, \<cent>,
  %\<currency>

% this should be the last package used
\usepackage{pdfsetup}

% urls in roman style, theory text in math-similar italics
\urlstyle{rm}
\isabellestyle{it}

% for uniform font size
%\renewcommand{\isastyle}{\isastyleminor}


\begin{document}

\title{Interpolation Polynomials (in HOL-Algebra)}
\author{Emin Karayel}
\maketitle
\begin{abstract}
A well known result from algebra is that, on any field, there is exactly one polynomial of degree
less than $n$ interpolating $n$ points~\cite[\textsection 7]{shoup2009computational}.

This entry contains a formalization of the above result, as well as the following generalization
in the case of finite fields $F$: There are $\lvert F\rvert^{m-n}$ polynomials of degree less
than $m \geq n$ interpolating the same $n$ points, where $\lvert F \rvert$ denotes the size of the
domain of the field. To establish the result the entry also includes a formalization of 
Lagrange interpolation, which might be of independent interest.

The formalized results are defined on the algebraic structures from HOL-Algebra, which are
distinct from the type-class based structures defined in HOL. Note that there is an existing 
formalization for polynomial interpolation and, in particular, Lagrange interpolation by Thiemann and
Yamada~\cite{Polynomial_Interpolation-AFP} on the type-class based structures in HOL.
\end{abstract}

\tableofcontents

% sane default for proof documents
\parindent 0pt\parskip 0.5ex

% generated text of all theories
\input{session}

% optional bibliography
\bibliographystyle{abbrv}
\bibliography{root}

\end{document}

%%% Local Variables:
%%% mode: latex
%%% TeX-master: t
%%% End:
