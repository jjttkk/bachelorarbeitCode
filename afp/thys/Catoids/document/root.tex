\documentclass[11pt,a4paper]{article}
\usepackage[T1]{fontenc}
\usepackage{isabelle,isabellesym}

% further packages required for unusual symbols (see also
% isabellesym.sty), use only when needed

%\usepackage{amssymb}
  %for \<leadsto>, \<box>, \<diamond>, \<sqsupset>, \<mho>, \<Join>,
  %\<lhd>, \<lesssim>, \<greatersim>, \<lessapprox>, \<greaterapprox>,
  %\<triangleq>, \<yen>, \<lozenge>

%\usepackage{eurosym}
  %for \<euro>

\usepackage[only,bigsqcap,bigparallel,fatsemi,interleave,sslash]{stmaryrd}
  %for \<Sqinter>, \<Parallel>, \<Zsemi>, \<Parallel>, \<sslash>

%\usepackage{eufrak}
  %for \<AA> ... \<ZZ>, \<aa> ... \<zz> (also included in amssymb)

%\usepackage{textcomp}
  %for \<onequarter>, \<onehalf>, \<threequarters>, \<degree>, \<cent>,
  %\<currency>

% this should be the last package used
\usepackage{pdfsetup}

% urls in roman style, theory text in math-similar italics
\urlstyle{rm}
\isabellestyle{it}

% for uniform font size
%\renewcommand{\isastyle}{\isastyleminor}


\begin{document}

\title{Catoids, Categories, Groupoids}
\author{Georg Struth}
\maketitle

\begin{abstract}
  This AFP entry formalises catoids, which are generalisations of
  single-set categories, and groupoids. More specifically, in catoids,
  the partial composition of arrows in a category is generalised to a
  multioperation, which sends pairs of elements to sets of elements,
  and the definedness condition of arrow composition -- two arrows can
  be composed if and only the target of the first matches the source
  of the second -- is relaxed. Beyond a library of basic laws for
  catoids, single-set categories and groupoids, I formalise the facts
  that every catoid can be lifted to a modal powerset quantale, that
  every groupoid can be lifted to a Dedekind quantale and to power set
  relation algebras, a special case of a famous result of Jónsson and
  Tarski. Finally, I show that single-set categories are equivalent to
  a standard axiomatisation of categories based on a set of objects
  and a set of arrows, and compare catoids with related structures
  such as multimonoid and relational monoids (monoids in the monoidal
  category Rel).
\end{abstract}

\tableofcontents

% sane default for proof documents
\parindent 0pt\parskip 0.5ex

\section{Introductory Remarks}

These Isabelle theories formalise results on catoids
from~\cite{CranchDS20,CalkFJSZ21,FahrenbergJSZ23} and groupoids
from~\cite{CalkMPS23}. Catoids generalise single-set categories, as
they can be found in Chapter XII of Mac Lane's
book~\cite{MacLane98}. One particular result, namely that catoids can
be lifted to (power set) relation algebras, is due to Jónsson and
Tarski~\cite{JonssonT52}.

A wide-ranging formalisation of category theory
based on single-set categories formalised as locales can already be
found in the AFP~\cite{Stark16}. The present type-class-based alternative
might lend itself to a similar programme.

The multioperation $X\times X\to \mathcal{P} X$ in the definition of
catoids is obviously isomorphic to a ternary relation
$X\to X\to X\to 2$. Simple mathematical components for relational
monoids, which are isomorphic (as categories with suitable morphisms)
to catoids, can already be found in the AFP~\cite{DongolGHS17}. At
this stage, I do not integrate the two components. They use different
formalisations of quantales with Isabelle, which remain to be
consolidated.

Catoids and groupoids admit many models. Those of catoids range from
shuffle monoids and generalised effect algebras to base algebras of
incidence and matrix algebras~\cite{FahrenbergJSZ23}, whereas
groupoids are so ubiquitous in mathematics that some mathematicians
have argued for interchanging their names with groups,
see~\cite{Brown87} for a brief history, which goes decades beyond that
of category theory.

The mathematical components in this AFP entry are also stepping stones
towards the formalisation of $(\omega,p)$-catoids, strict
$(\omega,p)$-categories and $(\omega,p)$-quantales. Components for
these structures will feature in a separate AFP entry. They contribute
to a larger programme on the formalisation of higher rewriting
techniques with proof assistants.

 \vspace{\baselineskip}

 I am grateful for a fellowship at the Collegium de Lyon, Institute
 for Advanced Study, where this formalisation work has been done.

% generated text of all theories
\input{session}

% optional bibliography
\bibliographystyle{abbrv}
\bibliography{root}




\end{document}

%%% Local Variables:
%%% mode: latex
%%% TeX-master: t
%%% End:
