\documentclass[11pt,a4paper]{article}
\usepackage[T1]{fontenc}
\usepackage{isabelle,isabellesym}

% this should be the last package used
%\usepackage{url}
%\usepackage{hyperref}
\usepackage{pdfsetup}

% urls in roman style, theory text in math-similar italics
\urlstyle{rm}
\isabellestyle{it}


\begin{document}

\title{Renaming-Enriched Sets (Rensets) and Renaming-Based Recursion}
\author{Andrei Popescu}
\maketitle

\begin{abstract}
  I formalize the notion of \emph{renaming-enriched sets} (\emph{rensets} sor short) and renaming-based recursion introduced in my \href{https://link.springer.com/book/10.1007/978-3-031-10769-6}{IJCAR 2022} 
  paper \href{https://link.springer.com/chapter/10.1007/978-3-031-10769-6_36}{``Rensets and Renaming-Based Recursion for Syntax with Bindings''} \cite{DBLP:conf/cade/Popescu22}.  
  Rensets are an algebraic axiomatization of renaming (variable-for-variable substitution). 
  The formalization includes a connection with nominal sets \cite{DBLP:conf/lics/GabbayP99,pitts_2013}, showing that any renset naturally gives rise to a nominal set. 
   It also includes examples of deploying the renaming-based recursor:  semantic interpretation, counting functions for free and bound occurrences, unary and parallel substitution, etc. Finally, it includes a variation of rensets that axiomatize term-for-variable substitution, called \emph{substitutive sets}, which yields a  corresponding recursion principle. 
\end{abstract}

\tableofcontents

% include generated text of all theories
\input{session}

\bibliographystyle{abbrv}
\bibliography{root}

\end{document}
