\documentclass[11pt,a4paper]{article}
\usepackage[T1]{fontenc}
\usepackage{isabelle,isabellesym}

% this should be the last package used
\usepackage{pdfsetup}

% urls in roman style, theory text in math-similar italics
\urlstyle{rm}
\isabellestyle{it}

% for uniform font size
% \renewcommand{\isastyle}{\isastyleminor}

\begin{document}

\title{A Naive Prover for First-Order Logic}
\author{Asta Halkjær From}
\maketitle

\begin{abstract}
	The AFP entry Abstract Completeness by Blanchette, Popescu and Traytel~\cite{Abstract-Completeness-AFP} formalizes the core of Beth/Hintikka-style completeness proofs for first-order logic and can be used to formalize executable sequent calculus provers.
	In the Journal of Automated Reasoning~\cite{BlanchettePT17}, the authors instantiate the framework with a sequent calculus for first-order logic and prove its completeness.
	Their use of an infinite set of proof rules indexed by formulas yields very direct arguments.
	A fair stream of these rules controls the prover, making its definition remarkably simple.
	The AFP entry, however, only contains a toy example for propositional logic.
	The AFP entry A Sequent Calculus Prover for First-Order Logic with Functions by From and Jacobsen~\cite{FOL-Seq-Calc2-AFP} also uses the framework, but uses a finite set of generic rules resulting in a more sophisticated prover with more complicated proofs.

	This entry contains an executable sequent calculus prover for first-order logic with functions in the style presented by Blanchette et al.
	The prover can be exported to Haskell and this entry includes formalized proofs of its soundness and completeness.
	The proofs are simpler than those for the prover by From and Jacobsen~\cite{FOL-Seq-Calc2-AFP} but the performance of the prover is significantly worse.

	The included theory \isa{Fair-Stream} first proves that the sequence of natural numbers 0, 0, 1, 0, 1, 2, etc.\ is fair.
	It then proves that mapping any surjective function across the sequence preserves fairness.
	This method of obtaining a fair stream of rules is similar to the one given by Blanchette et al.~\cite{BlanchettePT17}.
	The concrete functions from natural numbers to terms, formulas and rules are defined using the \isa{Nat-Bijection} theory in the HOL-Library.
\end{abstract}

\newpage

\tableofcontents

\newpage

% sane default for proof documents
\parindent 0pt\parskip 0.5ex

% generated text of all theories
\input{session}

% optional bibliography
\bibliographystyle{abbrv}
\bibliography{root}

\end{document}

%%% Local Variables:
%%% mode: latex
%%% TeX-master: t
%%% End:
