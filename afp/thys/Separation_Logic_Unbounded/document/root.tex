\documentclass[11pt,a4paper]{article}
\usepackage[T1]{fontenc}
\usepackage{isabelle,isabellesym}

% further packages required for unusual symbols (see also
% isabellesym.sty), use only when needed

%\usepackage{amssymb}
  %for \<leadsto>, \<box>, \<diamond>, \<sqsupset>, \<mho>, \<Join>,
  %\<lhd>, \<lesssim>, \<greatersim>, \<lessapprox>, \<greaterapprox>,
  %\<triangleq>, \<yen>, \<lozenge>

%\usepackage{eurosym}
  %for \<euro>

%\usepackage[only,bigsqcap,bigparallel,fatsemi,interleave,sslash]{stmaryrd}
  %for \<Sqinter>, \<Parallel>, \<Zsemi>, \<Parallel>, \<sslash>

%\usepackage{eufrak}
  %for \<AA> ... \<ZZ>, \<aa> ... \<zz> (also included in amssymb)

%\usepackage{textcomp}
  %for \<onequarter>, \<onehalf>, \<threequarters>, \<degree>, \<cent>,
  %\<currency>

% this should be the last package used
\usepackage{pdfsetup}

% urls in roman style, theory text in math-similar italics
\urlstyle{rm}
\isabellestyle{it}

% for uniform font size
%\renewcommand{\isastyle}{\isastyleminor}

\begin{document}

\title{Unbounded Separation Logic}
\author{Thibault Dardinier\\
Department of Computer Science, ETH Zurich, Switzerland}

\maketitle

\begin{abstract}
	Many separation logics~\cite{Reynolds02a} support fractional permissions~\cite{Boyland03,BornatCOP05} to distinguish between read and write access to a heap location, for instance, to allow concurrent reads while enforcing exclusive writes. Fractional permissions extend to composite assertions such as (co)inductive predicates and magic wands by allowing those to be multiplied~\cite{LeHobor18,Brotherston20,Wands22} by a fraction. Typical separation logic proofs require that this multiplication has three key properties: it needs to distribute over assertions, it should permit fractions to be factored out from assertions, and two fractions of the same assertion should be combinable into one larger fraction.

	Existing formal semantics incorporating fractional assertions into a separation logic define multiplication semantically (via models), resulting in a semantics in which distributivity and combinability do not hold for key resource assertions such as magic wands, and fractions cannot be factored out from a separating conjunction. By contrast, existing automatic separation logic verifiers~\cite{LeinoMuellerSmans09,JacobsSPVPP11,MuellerSchwerhoffSummers16,vercors2017} define multiplication syntactically, resulting in a different semantics for which it is unknown whether distributivity and combinability hold for all assertions.

In this entry, we present and formalize an \emph{unbounded} version of separation logic~\cite{UnboundedSL},
a novel semantics for separation logic assertions that allows states to hold more than a full permission to a heap location during the evaluation of an assertion.
By reimposing upper bounds on the permissions held per location at statement boundaries, we retain key properties of separation logic, in particular, we prove that the frame rule still holds.
We also prove that our assertion semantics unifies semantic and syntactic multiplication and thereby reconciles the discrepancy between separation logic theory and tools and enjoys distributivity, factorisability, and combinability.
\end{abstract}

\tableofcontents

% sane default for proof documents
\parindent 0pt\parskip 0.5ex

% generated text of all theories
\input{session}

% optional bibliography
\bibliographystyle{abbrv}
\bibliography{root}

\end{document}

%%% Local Variables:
%%% mode: latex
%%% TeX-master: t
%%% End:
