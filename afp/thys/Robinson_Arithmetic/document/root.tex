\documentclass[10pt,a4paper]{article}
\usepackage[T1]{fontenc}
\usepackage{isabelle,isabellesym}
\usepackage{a4wide}
\usepackage[english]{babel}
\usepackage{eufrak}
\usepackage{amssymb}

% this should be the last package used
\usepackage{pdfsetup}

% urls in roman style, theory text in math-similar italics
\urlstyle{rm}
\isabellestyle{literal}


\begin{document}

\title{Robinson Arithmetic}
\author{Andrei Popescu \and Dmitriy Traytel}

\maketitle

\begin{abstract} We instantiate our syntax-independent logic infrastructure developed in
\href{https://www.isa-afp.org/entries/Syntax_Independent_Logic.html}{a separate AFP entry} to the FOL theory
of Robinson arithmetic (also known as Q). The latter was formalised using Nominal Isabelle by adapting \href{https://www.isa-afp.org/entries/Incompleteness.html}{Larry
Paulson’s formalization of the Hereditarily Finite Set theory}.
\end{abstract}

\tableofcontents

% sane default for proof documents
\parindent 0pt\parskip 0.5ex

% generated text of all theories
\input{session}

\end{document}

%%% Local Variables:
%%% mode: latex
%%% TeX-master: t
%%% End:
