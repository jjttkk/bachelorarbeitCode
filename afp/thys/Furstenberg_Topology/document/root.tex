\documentclass[11pt,a4paper]{article}
\usepackage[T1]{fontenc}
\usepackage{isabelle,isabellesym}
\usepackage{amsfonts, amsmath, amssymb}

% this should be the last package used
\usepackage{pdfsetup}

% urls in roman style, theory text in math-similar italics
\urlstyle{rm}
\isabellestyle{it}

\begin{document}

\title{Furstenberg's Topology And\\ His Proof of the Infinitude of Primes}
\author{Manuel Eberl}
\maketitle

\begin{abstract}
This article gives a formal version of Furstenberg's topological proof of the infinitude of
primes. He defines a topology on the integers based on arithmetic progressions (or, equivalently, residue classes).
Using some fairly obvious properties of this topology, the infinitude of primes is then easily obtained.

Apart from this, this topology is also fairly `nice' in general: it is second countable, metrizable, and perfect. All of these (well-known)
facts are formally proven, including an explicit metric for the topology given by Zulfeqarr.
\end{abstract}

\tableofcontents
\newpage
\parindent 0pt\parskip 0.5ex

\input{session}

\bibliographystyle{abbrv}
\bibliography{root}

\end{document}

%%% Local Variables:
%%% mode: latex
%%% TeX-master: t
%%% End:
