\documentclass[11pt,a4paper]{article}
\usepackage[T1]{fontenc}
\usepackage{isabelle,isabellesym}

% further packages required for unusual symbols (see also
% isabellesym.sty), use only when needed

\usepackage{amssymb}
  %for \<leadsto>, \<box>, \<diamond>, \<sqsupset>, \<mho>, \<Join>,
  %\<lhd>, \<lesssim>, \<greatersim>, \<lessapprox>, \<greaterapprox>,
  %\<triangleq>, \<yen>, \<lozenge>
\usepackage{amsmath}
\usepackage{mathrsfs}
\usepackage{mathpartir}

%\usepackage{eurosym}
  %for \<euro>

\usepackage[only,bigsqcap]{stmaryrd}
%for \<Sqinter>

%% For Japanese languages in reference.
\usepackage{luatexja}

%\usepackage{eufrak}
  %for \<AA> ... \<ZZ>, \<aa> ... \<zz> (also included in amssymb)

%\usepackage{textcomp}
  %for \<onequarter>, \<onehalf>, \<threequarters>, \<degree>, \<cent>,
  %\<currency>

% this should be the last package used
\usepackage{pdfsetup}

% urls in roman style, theory text in math-similar italics
\urlstyle{rm}
\isabellestyle{it}


% for uniform font size
%\renewcommand{\isastyle}{\isastyleminor}


\begin{document}

\title{Standard Borel Spaces}
\author{Michikazu Hirata}
\maketitle
\begin{abstract}
  This entry includes a formalization of standard Borel spaces
  and (a variant of) the Borel isomorphism theorem.
  A separable complete metrizable topological space is called a polish space
  and a measurable space generated from a polish space is called a standard Borel space.
  We formalize the notion of standard Borel spaces by establishing
  set-based metric spaces,
  and then prove (a variant of) the Borel isomorphism theorem.
  The theorem states that a standard Borel spaces is either a countable discrete space
  or isomorphic to $\mathbb{R}$.
\end{abstract}

\tableofcontents

% sane default for proof documents
\parindent 0pt\parskip 0.5ex

% generated text of all theories
\input{session}

% optional bibliography
\bibliographystyle{abbrv}
\bibliography{root}

\end{document}

%%% Local Variables:
%%% mode: latex
%%% TeX-master: t
%%% End:
