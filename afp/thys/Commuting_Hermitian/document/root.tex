\documentclass[11pt,a4paper]{article}
\usepackage[T1]{fontenc}
\usepackage{isabelle,isabellesym}
\newcommand{\mnset}[1]{\left\{#1\right\}}\newcommand{\set}[1]{\left\{#1\right\}}
% further packages required for unusual symbols (see also
% isabellesym.sty), use only when needed

%\usepackage{amssymb}
  %for \<leadsto>, \<box>, \<diamond>, \<sqsupset>, \<mho>, \<Join>,
  %\<lhd>, \<lesssim>, \<greatersim>, \<lessapprox>, \<greaterapprox>,
  %\<triangleq>, \<yen>, \<lozenge>

%\usepackage{eurosym}
  %for \<euro>

%\usepackage[only,bigsqcap,bigparallel,fatsemi,interleave,sslash]{stmaryrd}
  %for \<Sqinter>, \<Parallel>, \<Zsemi>, \<Parallel>, \<sslash>

%\usepackage{eufrak}
  %for \<AA> ... \<ZZ>, \<aa> ... \<zz> (also included in amssymb)

%\usepackage{textcomp}
  %for \<onequarter>, \<onehalf>, \<threequarters>, \<degree>, \<cent>,
  %\<currency>

% this should be the last package used
\usepackage{pdfsetup}

% urls in roman style, theory text in math-similar italics
\urlstyle{rm}
\isabellestyle{it}

% for uniform font size
%\renewcommand{\isastyle}{\isastyleminor}


\begin{document}

\title{Simultaneous diagonalization of pairwise commuting Hermitian matrices}
\author{Mnacho Echenim}
\maketitle

\begin{abstract}
	A Hermitian matrix is a square complex matrix $A$ that is equal to its conjugate transpose $A^\dagger$. The (finite-dimensional) spectral theorem states that for any such matrix $A$, we have the equality $A = U\cdot B\cdot U^\dagger$, where $U$ is a unitary matrix and $B$ is a diagonal matrix containing only real elements. We formalize the generalization of this result, which states that if $\mnset{A_1, \ldots, A_n}$ are Hermitian and pairwise commuting matrices, then there exists a unitary matrix $U$ such that $A_i = U\cdot B_i \cdot U^\dagger$, for $i = 1,\ldots, n$, and each $B_i$ is diagonal and contains only real elements. Sets of pairwise commuting Hermitian matrices are called \emph{Complete Sets of Commuting Observables} in Quantum Mechanics, where they represent physical quantities that can be simultaneously measured to uniquely distinguish quantum states.
\end{abstract}

\tableofcontents

\paragraph{Acknowledgments} This work was partially supported by Agence Nationale de la Recherche, through \emph{Plan France 2030 (ref. ANR-22-PETQ-0007)}.

% sane default for proof documents
\parindent 0pt\parskip 0.5ex

% generated text of all theories
\input{session}

% optional bibliography
%\bibliographystyle{abbrv}
%\bibliography{root}

\end{document}

%%% Local Variables:
%%% mode: latex
%%% TeX-master: t
%%% End:
