\documentclass[11pt,a4paper]{article}
\usepackage[T1]{fontenc}
\usepackage{isabelle,isabellesym}
\usepackage{tikz}
\usetikzlibrary{arrows}
\usepackage{amssymb}

% Bibliography
\usepackage[authoryear,sort]{natbib}
\bibpunct();A{},

% this should be the last package used
\usepackage{pdfsetup}

% urls in roman style, theory text in math-similar italics
\urlstyle{rm}
\isabellestyle{it}


\begin{document}

\title{The Stern-Brocot Tree}
\author{Peter Gammie \and Andreas Lochbihler}
\maketitle

\begin{abstract}
  The Stern-Brocot tree contains all rational numbers exactly once and in their lowest terms.
  We formalise the Stern-Brocot tree as a coinductive tree using recursive and iterative specifications,
  which we have proven equivalent, and show that it indeed contains all the numbers as stated.
  Following Hinze, we prove that the Stern-Brocot tree can be linearised looplessly into Stern's
  diatonic sequence (also known as Dijkstra's fusc function) and that it is a permutation of the
  Bird tree.

  The reasoning stays at an abstract level by appealing to the uniqueness of solutions of
  guarded recursive equations and lifting algebraic laws point-wise to trees and streams using
  applicative functors.
\end{abstract}

\tableofcontents

% sane default for proof documents
\parindent 0pt\parskip 0.5ex

\paragraph{Acknowledgements}
Thanks to Dave Cock for a fruitful discussion about unique fixed points.

% generated text of all theories
\input{session}

\bibliographystyle{plainnat}
\bibliography{root}
%\addcontentsline{toc}{section}{References}

\end{document}
