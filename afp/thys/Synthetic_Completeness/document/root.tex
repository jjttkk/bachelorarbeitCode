\documentclass[11pt,a4paper,oneside]{book}
\usepackage[T1]{fontenc}
\usepackage{isabelle,isabellesym}
\usepackage{fullpage}
\usepackage{amssymb}
% this should be the last package used
\usepackage{pdfsetup}

% urls in roman style, theory text in math-similar italics
\urlstyle{rm}
\isabellestyle{it}

% for uniform font size
%\renewcommand{\isastyle}{\isastyleminor}

\begin{document}

\title{Synthetic Completeness}
\author{Asta Halkjær From}
\maketitle

\setcounter{page}{2}

\chapter*{Abstract}
\addcontentsline{toc}{chapter}{Abstract}

In this work, I provide an abstract framework for proving the completeness of a logical calculus using the synthetic method.
The synthetic method is based on maximal consistent saturated sets (MCSs).
A set of formulas is consistent (with respect to the calculus) when we cannot derive a contradiction from it.
It is maximally consistent when it contains every formula that is consistent with it.
For logics where it is relevant, it is saturated when it contains a witness for every existential formula.
To prove completeness using these maximal consistent saturated sets, we prove a truth lemma: every formula in an MCS has a satisfying model.
Here, Hintikka sets provide a useful stepping stone.
These can be seen as characterizations of the MCSs based on simple subformula conditions rather than via the calculus.
We then prove that every Hintikka set gives rise to a satisfying model and that MCSs are Hintikka sets.
Now, assume a valid formula cannot be derived.
Then its negation must be consistent and therefore satisfiable.
This contradicts validity and the original formula must be derivable.

To start, I build maximal consistent saturated sets for any logic that satisfies a small set of assumptions.
I do this using a transfinite version of Lindenbaum's lemma, which allows me to support languages of any cardinality.
I then prove useful abstract results about derivations and refutations as they relate to MCSs.
Finally, I show how Hintikka sets can be derived from the logic's semantics, outlining one way to prove the required truth lemma.

To demonstrate the versatility of the framework, I instantiate it with five different examples.
The formalization contains soundness and completeness results for:
a propositional tableau calculus,
a propositional sequent calculus,
an axiomatic system for modal logic,
a labelled natural deduction system for hybrid logic and
a natural deduction system for first-order logic.
The tableau example uses custom Hintikka sets based on the calculus, but the other four examples derive them from the semantics in the style of the framework.
The hybrid and first-order logic examples rely on saturated MCSs.
This places requirements on the cardinalities of their languages to ensure that there are enough witnesses available.
In both cases, the type of witnesses must be infinite and have cardinality at least that of the type of propositional/predicate symbols.

\clearpage
\phantomsection
\addcontentsline{toc}{chapter}{Contents}
\tableofcontents

% sane default for proof documents
\parindent 0pt\parskip 0.5ex

% generated text of all theories
\input{session}

\clearpage
\phantomsection
\addcontentsline{toc}{chapter}{Bibliography}
\bibliographystyle{abbrv}
\bibliography{root}

\nocite{*}

\end{document}

%%% Local Variables:
%%% mode: latex
%%% TeX-master: t
%%% End:
