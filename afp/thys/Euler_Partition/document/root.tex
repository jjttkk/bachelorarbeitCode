\documentclass[11pt,a4paper]{article}
\usepackage[T1]{fontenc}
\usepackage{isabelle,isabellesym}

% this should be the last package used
\usepackage{pdfsetup}

% urls in roman style, theory text in math-similar italics
\urlstyle{rm}
\isabellestyle{it}


\begin{document}

\title{Euler's Partition Theorem}
\author{Lukas Bulwahn}
\maketitle

\begin{abstract}
Euler's Partition Theorem states that the number of partitions with only
distinct parts is equal to the number of partitions with only odd parts.
The combinatorial proof follows John Harrison's pre-existing HOL Light
formalization~\cite{Harrison}.
To understand the rough idea of the proof, I read the lecture notes of
the MIT course 18.312 on Algebraic Combinatorics~\cite{Musiker-2009}
by Gregg Musiker.
This theorem is the 45th theorem of the Top 100 Theorems list.

\end{abstract}

\tableofcontents

% sane default for proof documents
\parindent 0pt\parskip 0.5ex

% generated text of all theories
\input{session}

\bibliographystyle{abbrv}
\bibliography{root}

\end{document}

%%% Local Variables:
%%% mode: latex
%%% TeX-master: t
%%% End:
