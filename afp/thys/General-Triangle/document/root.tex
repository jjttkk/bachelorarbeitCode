\documentclass[11pt,a4paper,notitlepage]{scrartcl}
\usepackage[T1]{fontenc}
\usepackage{isabelle,isabellesym}
\usepackage{textcomp}

% this should be the last package used
\usepackage{pdfsetup}

% urls in roman style, theory text in math-similar italics
\urlstyle{rm}
\isabellestyle{it}

\usepackage{tikz}
\usetikzlibrary{calc}

\title{The General Triangle Is Unique}
\author{Joachim Breitner}
\date{April 1, 2011}
\begin{document}

\titlehead{
\begin{center}
\begin{tikzpicture}
\draw (0,0) node (a) {} -- (6,0) node (b) {}
 -- (intersection cs:
  first line ={(a) -- (45:1cm)},
  second line ={(b) -- ($(b) + (120:1cm)$) }) node (c) {} -- cycle;
\draw (a) +(8mm,0) arc (0:45:8mm);  
\draw (b) +(-8mm,0) arc (180:120:8mm);  
\draw ($(c) + (225:8mm)$) arc (225:300:8mm);
\node at ($(a) + (20:5mm)$) {$a$};
\node at ($(b) + (150:5mm)$) {$b$};
\node at ($(c) + (265:5mm)$) {$c$};
\end{tikzpicture}
\end{center}
}

\maketitle

\begin{abstract}
  Some acute-angled triangles are special, e.g.\ right-angled or isosceles
  triangles. Some are not of this kind, but, without measuring angles, look as
  if they are. In that sense, there is exactly one general triangle. This
  well-known fact\cite{Tergan} is proven here formally.
\end{abstract}

%\tableofcontents

% include generated text of all theories
\input{session}

\bibliographystyle{abbrv}
\bibliography{root}

\end{document}
