\documentclass[11pt,a4paper]{article}
\usepackage[T1]{fontenc}
\usepackage{isabelle,isabellesym}

\usepackage{url}

% this should be the last package used
\usepackage{pdfsetup}

% urls in roman style, theory text in math-similar italics
\urlstyle{rm}
\isabellestyle{it}


\begin{document}

\title{Derangements}
\author{Lukas Bulwahn}
\maketitle

\begin{abstract}
The Derangements Formula describes the number of fixpoint-free permutations
as closed-form formula. This theorem is the 88th theorem of the Top 100
Theorems list.
\end{abstract}
\tableofcontents

% sane default for proof documents
\parindent 0pt\parskip 0.5ex

% generated text of all theories
\input{session}

% optional bibliography
\nocite{Harrison,wikipedia:derangement,wikipedia:fixpunktfreie-permutation,wikipedia:rencontres-numbers}


\bibliographystyle{abbrv}
\bibliography{root}

\end{document}

%%% Local Variables:
%%% mode: latex
%%% TeX-master: t
%%% End:
