\documentclass[11pt,a4paper]{book}
\usepackage[T1]{fontenc}
\usepackage{isabelle,isabellesym}
\usepackage{amssymb}
\usepackage[english]{babel}
\usepackage[only,bigsqcap]{stmaryrd}

% this should be the last package used
\usepackage{pdfsetup}

% urls in roman style, theory text in math-similar italics
\urlstyle{rm}
\isabellestyle{it}

% Tweaks
\newcounter{TTStweak_tag}
\setcounter{TTStweak_tag}{0}
\newcommand{\setTTS}{\setcounter{TTStweak_tag}{1}}
\newcommand{\resetTTS}{\setcounter{TTStweak_tag}{0}}
\newcommand{\insertTTS}{\ifnum\value{TTStweak_tag}=1 \ \ \ \fi}

\renewcommand{\isakeyword}[1]{\resetTTS\emph{\bf\def\isachardot{.}\def\isacharunderscore{\isacharunderscorekeyword}\def\isacharbraceleft{\{}\def\isacharbraceright{\}}#1}}
\renewcommand{\isachardoublequoteopen}{\insertTTS}
\renewcommand{\isachardoublequoteclose}{\setTTS}
\renewcommand{\isanewline}{\mbox{}\par\mbox{}\resetTTS}

\renewcommand{\isamarkupcmt}[1]{\hangindent5ex{\isastylecmt --- #1}}

\newcommand{\isaheader}[1]{#1}
\newcommand{\isachapter}[1]{\chapter{#1}}
\newcommand{\isasection}[1]{\section{#1}}

\renewcommand{\isamarkupchapter}[1]{\chapter{#1}}
\renewcommand{\isamarkupsection}[1]{\subsection{#1}}
\renewcommand{\isamarkupsubsection}[1]{\subsubsection{#1}}
\renewcommand{\isamarkupsubsubsection}[1]{\paragraph{#1}}

\makeatletter
\newenvironment{abstract}{%
  \small
  \begin{center}%
    {\bfseries \abstractname\vspace{-.5em}\vspace{\z@}}%
  \end{center}%
  \quotation}{\endquotation}
\makeatother

% Macros for interface and implementation documentation

\newcommand{\docIntf}[2]{

{\bf Interface} #1 (theory #2)\\}

\newcommand{\docAbstype}[1]{\hspace*{2em}Abstract type: #1\\}

\newcommand{\docDesc}[1]{#1}

\newenvironment{docImpls}{

{\bf Implementations:}
\begin{description}%
}{\end{description}}

\newcommand{\docImpl}[1]{\item[#1]}

\newcommand{\docType}[1]{(Type: #1)}

\newcommand{\docAbbrv}[1]{(Abbrv: #1)}


\begin{document}

\title{Isabelle Collections Framework}
\author{By Peter Lammich and Andreas Lochbihler}
\maketitle

\begin{abstract}
  This development provides an efficient, extensible, machine checked collections framework for use
  in Isabelle/HOL. The library adopts the concepts of interface, implementation and generic algorithm from
  object-oriented programming and implements them in Isabelle/HOL.

  The framework features the use of data refinement techniques to refine an abstract specification (using high-level concepts like sets) to a more concrete implementation (using collection datastructures, like red-black-trees). The code-generator of Isabelle/HOL can be used to generate efficient code in all supported target languages, i.e. Haskell, SML, and OCaml.
\end{abstract}

\clearpage

\tableofcontents

\clearpage

% sane default for proof documents
\parindent 0pt\parskip 0.5ex

\section{Introduction}

This paper presents two models for a hotel key card system and the
verification of their safety (in Isabelle/HOL~\cite{LNCS2283}). The
models are based on Section~6.2, \emph{Hotel Room Locking}, and
Appendix~E in the book by Daniel Jackson~\cite{Jackson06}. Jackson
employs his Alloy system to check that there are no small
counterexamples to safety. We confirm his conjecture of safety by a
formal proof.

Most hotels operate a digital key card system. Upon check-in, you
receive a card with your own key on it (typically a pseudorandom
number). The lock for each room reads your card and opens the door if
the key is correct. The system is decentralized,
i.e.\ each lock is a stand-alone, battery-powered device without
connection to the computer at reception or to any other device. So
how does the lock know that your key is correct? There are a number of
similar systems and we discuss the one described in Appendix~E
of~\cite{Jackson06}. Here each card carries two keys: the old key of
the previous occupant of the room ($key_1$), and your own new key
($key_2$). The lock always holds one key, its ``current'' key. When
you enter your room for the first time, the lock notices that its
current key is $key_1$ on your card and recodes itself, i.e.\ it replaces
its own current key with $key_2$ on your card. When you enter the next
time, the lock finds that its current key is equal to your $key_2$ and
opens the door without recoding itself. Your card is never modified by
the lock. Eventually, a new guest with a new key enters the room,
recodes the lock, and you cannot enter anymore.

After a short introduction of the notation we discuss two very
different specifications, a state based and a trace based one, and
prove their safety and their equivalence. The complete formalization
is available online in the \emph{Archive of Formal Proofs} at
\url{isa-afp.org}.


% generated text of all theories
\input{session}

\section{Conclusion}\label{sec:concl}
  We have presented a verification of two variants of Gabow's algorithm: Computation of the strongly connected components of
  a graph, and emptiness check of a generalized B\"uchi automaton. We have extracted efficient code with a performance comparable to a
  reference implementation in Java.
  
  We have modularized the formalization in two directions: First, we share most of the proofs between the two variants of the algorithm. Second,
  we use a stepwise refinement approach to separate the algorithmic ideas and the correctness proof from implementation details.
  Sharing of the proofs reduced the overall effort of developing both algorithms. Using a stepwise refinement approach allowed us to
  formalize an efficient implementation, without making the correctness proof complex and unmanageable by cluttering it with implementation details.

  Our development approach is independent of Gabow's algorithm, and can be re-used for the verification of other algorithms.

  \paragraph{Current and Future Work} 
  An important direction of future work is to fine-tune the implementation of 
  the emptiness check algorithm for speed, as speed of the checking algorithm
  directely influences the performance of the modelchecker.


% optional bibliography
\bibliographystyle{abbrv}
\bibliography{root}

\end{document}

%%% Local Variables:
%%% mode: latex
%%% TeX-master: t
%%% End:
