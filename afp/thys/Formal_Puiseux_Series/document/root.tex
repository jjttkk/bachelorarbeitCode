\documentclass[11pt,a4paper]{article}
\usepackage[T1]{fontenc}
\usepackage{isabelle,isabellesym}
\usepackage{amsfonts,amsmath,amssymb}
\usepackage{pgfplots}

% this should be the last package used
\usepackage{pdfsetup}

% urls in roman style, theory text in math-similar italics
\urlstyle{rm}
\isabellestyle{it}

\begin{document}

\title{Formal Puiseux Series}
\author{Manuel Eberl}
\maketitle

\begin{abstract}
Formal Puiseux series are generalisations of formal power series and formal Laurent series that also allow for fractional exponents. They have the following general form:
\[\sum_{i=N}^\infty a_{i/d} X^{i/d}\]
where $N$ is an integer and $d$ is a positive integer.

This entry defines these series including their basic algebraic properties. Furthermore, it proves the Newton--Puiseux Theorem, namely that the Puiseux series over an algebraically closed field of characteristic 0 are also algebraically closed.
\end{abstract}

\newpage
\tableofcontents
\newpage
\parindent 0pt\parskip 0.5ex

\input{session}

\nocite{corless96}
\bibliographystyle{abbrv}
\bibliography{root}

\end{document}

%%% Local Variables:
%%% mode: latex
%%% TeX-master: t
%%% End:
