\documentclass[11pt,a4paper]{article}
\usepackage[T1]{fontenc}
\usepackage{isabelle,isabellesym}
\usepackage[english]{babel}
\usepackage{amssymb}
\usepackage[only,bigsqcap]{stmaryrd}

% this should be the last package used
\usepackage{pdfsetup}

% urls in roman style, theory text in math-similar italics
\urlstyle{rm}
\isabellestyle{it}

\newcommand{\isaheader}[1]{#1}
\newcommand{\isachapter}[1]{\chapter{#1}}
\newcommand{\isasection}[1]{\section{#1}}

% General
\newcommand{\ie}{i.\,e.\ }
\newcommand{\eg}{e.\,g.\ }
\newcommand{\wrt}{w.\,r.\,t.\ }
\newcommand{\cf}{cf.\ }

\begin{document}

\title{Promela Formalization}
\author{By Ren\'{e} Neumann}
\maketitle

\begin{abstract}
    We present an executable formalization of the language Promela, the description language for models of the model checker SPIN. This formalization is part of the work for a completely verified
    model checker (CAVA), but also serves as a useful (and executable!) description of the semantics of the language itself, something that is currently missing.
    The formalization uses three steps: It takes an abstract syntax tree generated from an SML parser, removes syntactic sugar and enriches it with type information. This further gets translated into a transition system, on which the semantic engine (read: successor function) operates.
\end{abstract}

\clearpage

\tableofcontents

\clearpage

\section{Introduction}

This paper presents two models for a hotel key card system and the
verification of their safety (in Isabelle/HOL~\cite{LNCS2283}). The
models are based on Section~6.2, \emph{Hotel Room Locking}, and
Appendix~E in the book by Daniel Jackson~\cite{Jackson06}. Jackson
employs his Alloy system to check that there are no small
counterexamples to safety. We confirm his conjecture of safety by a
formal proof.

Most hotels operate a digital key card system. Upon check-in, you
receive a card with your own key on it (typically a pseudorandom
number). The lock for each room reads your card and opens the door if
the key is correct. The system is decentralized,
i.e.\ each lock is a stand-alone, battery-powered device without
connection to the computer at reception or to any other device. So
how does the lock know that your key is correct? There are a number of
similar systems and we discuss the one described in Appendix~E
of~\cite{Jackson06}. Here each card carries two keys: the old key of
the previous occupant of the room ($key_1$), and your own new key
($key_2$). The lock always holds one key, its ``current'' key. When
you enter your room for the first time, the lock notices that its
current key is $key_1$ on your card and recodes itself, i.e.\ it replaces
its own current key with $key_2$ on your card. When you enter the next
time, the lock finds that its current key is equal to your $key_2$ and
opens the door without recoding itself. Your card is never modified by
the lock. Eventually, a new guest with a new key enters the room,
recodes the lock, and you cannot enter anymore.

After a short introduction of the notation we discuss two very
different specifications, a state based and a trace based one, and
prove their safety and their equivalence. The complete formalization
is available online in the \emph{Archive of Formal Proofs} at
\url{isa-afp.org}.



% sane default for proof documents
\parindent 0pt\parskip 0.5ex

% generated text of all theories
\input{session}

% optional bibliography
\bibliographystyle{abbrv}
\bibliography{root}

\end{document}

%%% Local Variables:
%%% mode: latex
%%% TeX-master: t
%%% End:
