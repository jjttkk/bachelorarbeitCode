\documentclass[11pt,a4paper]{article}
\usepackage[T1]{fontenc}
\usepackage{isabelle,isabellesym}
\usepackage{amssymb}

% this should be the last package used
\usepackage{pdfsetup}

% urls in roman style, theory text in math-similar italics
\urlstyle{rm}
\isabellestyle{it}

\begin{document}

\title{Hermite Normal Form}
\author{By Jose Divas\'on and Jes\'us Aransay\thanks{This research has been funded 
  by the research grant FPI-UR-12 of the Universidad de La Rioja and by the project MTM2014-54151-P from Ministerio de Econom\'ia y Competitividad
(Gobierno de Espa\~na).}}
\maketitle

\begin{abstract}
The Hermite Normal Form is a canonical matrix analogue of Reduced Echelon Form, but involving matrices over more general rings.
In this work we formalise an algorithm to compute the Hermite Normal Form of a matrix by means of elementary row operations, 
taking advantage of the Echelon Form AFP entry. We have proven the correctness of such an algorithm and refined it to immutable arrays.
Furthermore, we have also formalised the uniqueness of the Hermite Normal Form of a matrix.
Code can be exported and some examples of execution involving $\mathbb{Z}$-matrices and $\mathbb{K}[x]$-matrices are presented as well.
\end{abstract}

\tableofcontents

% sane default for proof documents
\parindent 0pt\parskip 0.5ex

% generated text of all theories
\input{session}

\end{document}

%%% Local Variables:
%%% mode: latex
%%% TeX-master: t
%%% End:
