\documentclass[11pt,a4paper]{article}
\usepackage[T1]{fontenc}
\usepackage{isabelle,isabellesym}
% further packages required for unusual symbols (see also
% isabellesym.sty), use only when needed

\usepackage{amssymb}
  %for \<leadsto>, \<box>, \<diamond>, \<sqsupset>, \<mho>, \<Join>,
  %\<lhd>, \<lesssim>, \<greatersim>, \<lessapprox>, \<greaterapprox>,
  %\<triangleq>, \<yen>, \<lozenge>
\usepackage{amsmath}

%\usepackage{eurosym}
  %for \<euro>

%\usepackage[only,bigsqcap]{stmaryrd}
  %for \<Sqinter>

%\usepackage{eufrak}
  %for \<AA> ... \<ZZ>, \<aa> ... \<zz> (also included in amssymb)

%\usepackage{textcomp}
  %for \<onequarter>, \<onehalf>, \<threequarters>, \<degree>, \<cent>,
  %\<currency>

% this should be the last package used
\usepackage{pdfsetup}

% urls in roman style, theory text in math-similar italics
\urlstyle{rm}
\isabellestyle{it}

% for uniform font size
%\renewcommand{\isastyle}{\isastyleminor}


\begin{document}

\title{Isabelle Marries Dirac: a Library for Quantum Computation and Quantum Information}
\author{Anthony Bordg, Hanna Lachnitt and Yijun He}
\maketitle

\tableofcontents

\begin{abstract}
This work is an effort to formalise some quantum algorithms and results in quantum information theory. Formal methods being critical for the safety and security of algorithms and protocols, we foresee their widespread use for quantum computing in the future. We have developed a large library for quantum computing in Isabelle based on a matrix representation for quantum circuits, successfully formalising the no-cloning theorem, quantum teleportation, Deutsch's algorithm, the Deutsch-Jozsa algorithm and the quantum Prisoner's Dilemma.
\end{abstract}


% sane default for proof documents
\parindent 0pt\parskip 0.5ex

% generated text of all theories
\input{session}

\section{Acknowledgements}

The work has been jointly supported by the Cambridge Mathematics Placements (CMP) 
Programme and the ERC Advanced Grant ALEXANDRIA (Project GA 742178).

% optional bibliography
\nocite{*}
\bibliographystyle{abbrv}
\bibliography{root}

\end{document}

%%% Local Variables:
%%% mode: latex
%%% TeX-master: t
%%% End:
