\documentclass[11pt,a4paper]{article}
\usepackage[T1]{fontenc}
\usepackage{isabelle,isabellesym}

% further packages required for unusual symbols (see also
% isabellesym.sty), use only when needed

\usepackage{amssymb}
  %for \<leadsto>, \<box>, \<diamond>, \<sqsupset>, \<mho>, \<Join>,
  %\<lhd>, \<lesssim>, \<greatersim>, \<lessapprox>, \<greaterapprox>,
  %\<triangleq>, \<yen>, \<lozenge>

%\usepackage{eurosym}
  %for \<euro>

%\usepackage[only,bigsqcap]{stmaryrd}
  %for \<Sqinter>

%\usepackage{eufrak}
  %for \<AA> ... \<ZZ>, \<aa> ... \<zz> (also included in amssymb)

%\usepackage{textcomp}
  %for \<onequarter>, \<onehalf>, \<threequarters>, \<degree>, \<cent>,
  %\<currency>

% this should be the last package used
\usepackage{pdfsetup}

% urls in roman style, theory text in math-similar italics
\urlstyle{rm}
\isabellestyle{it}

% for uniform font size
%\renewcommand{\isastyle}{\isastyleminor}


\begin{document}

\title{Verified Metatheory and Type Inference for a Name-Carrying Simply-Typed \(\lambda\)-Calculus}
\author{Michael Rawson}
\maketitle

\begin{abstract}
I formalise a Church-style simply-typed \(\lambda\)-calculus, extended with pairs, a unit value, and projection functions, and show some metatheory of the calculus, such as the subject reduction property.
Particular attention is paid to the treatment of names in the calculus.
A nominal style of binding is used, but I use a manual approach over Nominal Isabelle in order to extract an executable type inference algorithm.
More information can be found in my \href{http://www.openthesis.org/documents/Verified-Metatheory-Type-Inference-Simply-603182.html}{undergraduate dissertation}.
\end{abstract}
\tableofcontents

% sane default for proof documents
\parindent 0pt\parskip 0.5ex

% generated text of all theories
\input{session}

% optional bibliography
%\bibliographystyle{abbrv}
%\bibliography{root}

\end{document}

%%% Local Variables:
%%% mode: latex
%%% TeX-master: t
%%% End:
