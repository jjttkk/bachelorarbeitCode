\documentclass[11pt,a4paper]{article}
\usepackage[T1]{fontenc}
\usepackage{isabelle,isabellesym}
\usepackage[nottoc,numbib]{tocbibind}
\usepackage{textcomp}

% this should be the last package used
\usepackage{pdfsetup}

% urls in roman style, theory text in math-similar italics
\urlstyle{rm}
\isabellestyle{it}


\renewcommand{\refname}{Bibliography}

\begin{document}

\title{Representations of Finite Groups}
\author{Jeremy Sylvestre \\ University of Alberta, Augustana Campus \\ \href{mailto:jsylvest@ualberta.ca}{\url{jeremy.sylvestre@ualberta.ca}}}

\maketitle

\begin{abstract}
We provide a formal framework for the theory of representations of finite groups, as modules over the group ring. Along the way, we develop the general theory of groups (relying on the \textit{group{\_}add} class for the basics), modules, and vector spaces, to the extent required for theory of group representations. We then provide formal proofs of several important introductory theorems in the subject, including Maschke's theorem, Schur's lemma, and Frobenius reciprocity. We also prove that every irreducible representation is isomorphic to a submodule of the group ring, leading to the fact that for a finite group there are only finitely many isomorphism classes of irreducible representations. In all of this, no restriction is made on the characteristic of the ring or field of scalars until the definition of a group representation, and then the only restriction made is that the characteristic must not divide the order of the group.
\end{abstract}

\tableofcontents

% sane default for proof documents
\parindent 0pt\parskip 0.5ex

\vspace*{32pt}
\textit{Note:} A number of the proofs in this theory were modelled on or inspired by proofs in the books listed in the bibliography.
\vspace*{32pt}

% generated text of all theories
\input{session}

% optional bibliography
\nocite{*}
\bibliographystyle{abbrv}
\bibliography{root}

\end{document}

%%% Local Variables:
%%% mode: latex
%%% TeX-master: t
%%% End:
