\documentclass[11pt,a4paper]{article}
\usepackage{isabelle,isabellesym}
\usepackage{authblk}

% further packages required for unusual symbols (see also
% isabellesym.sty), use only when needed

%\usepackage{amssymb}
  %for \<leadsto>, \<box>, \<diamond>, \<sqsupset>, \<mho>, \<Join>,
  %\<lhd>, \<lesssim>, \<greatersim>, \<lessapprox>, \<greaterapprox>,
  %\<triangleq>, \<yen>, \<lozenge>

%\usepackage{eurosym}
  %for \<euro>

%\usepackage[only,bigsqcap]{stmaryrd}
  %for \<Sqinter>

%\usepackage{eufrak}
  %for \<AA> ... \<ZZ>, \<aa> ... \<zz> (also included in amssymb)

%\usepackage{textcomp}
  %for \<onequarter>, \<onehalf>, \<threequarters>, \<degree>, \<cent>,
  %\<currency>

% this should be the last package used
\usepackage{pdfsetup}

% urls in roman style, theory text in math-similar italics
\urlstyle{rm}
\isabellestyle{it}

% for uniform font size
%\renewcommand{\isastyle}{\isastyleminor}


\begin{document}

\title{Digit Expansions}

\author{Jonas Bayer, Marco David, Abhik Pal and Benedikt Stock}

\maketitle

%\footnotetext[*]{Text}

\begin{abstract}
We formalize how a natural number $a$ can be expanded as
\[ a = \sum_{k=0}^l a_k b^k \]
for some base $b$ and prove properties about functions that operate on such expansions. This includes the formalization of concepts such as digit shifts and carries. For a base that is a power of $2$ we formalize the binary AND, binary orthogonality and binary masking of two natural numbers. This library on digit expansions builds the basis for the formalization of the DPRM theorem.
\end{abstract}


\tableofcontents

\newpage

% sane default for proof documents
\parindent 0pt\parskip 0.5ex

\newpage

% generated text of all theories
\input{session}

% optional bibliography
\bibliographystyle{abbrv}
\bibliography{root}

\end{document}

%%% Local Variables:
%%% mode: latex
%%% TeX-master: t
%%% End:
