\documentclass[11pt,a4paper]{article}
\usepackage[T1]{fontenc}
\usepackage{isabelle,isabellesym}

% further packages required for unusual symbols (see also
% isabellesym.sty), use only when needed

%\usepackage{amssymb}
  %for \<leadsto>, \<box>, \<diamond>, \<sqsupset>, \<mho>, \<Join>,
  %\<lhd>, \<lesssim>, \<greatersim>, \<lessapprox>, \<greaterapprox>,
  %\<triangleq>, \<yen>, \<lozenge>

%\usepackage{eurosym}
  %for \<euro>

%\usepackage[only,bigsqcap]{stmaryrd}
  %for \<Sqinter>

%\usepackage{eufrak}
  %for \<AA> ... \<ZZ>, \<aa> ... \<zz> (also included in amssymb)

%\usepackage{textcomp}
  %for \<onequarter>, \<onehalf>, \<threequarters>, \<degree>, \<cent>,
  %\<currency>

% this should be the last package used
\usepackage{pdfsetup}

% urls in roman style, theory text in math-similar italics
\urlstyle{rm}
\isabellestyle{it}

% for uniform font size
%\renewcommand{\isastyle}{\isastyleminor}


\begin{document}

\title{A verified algorithm for computing the Smith normal form of a matrix}
\author{Jose Divas\'on}
\maketitle

\begin{abstract}
This work presents a formal proof in Isabelle/HOL of an algorithm
to transform a matrix into its Smith normal form, a canonical
matrix form, in a general setting: the algorithm is parameterized by
operations to prove its existence over elementary divisor rings, while execution
is guaranteed over Euclidean domains. We also provide a formal proof
on some results about the generality of this algorithm as well as the
uniqueness of the Smith normal form.

Since Isabelle/HOL does not feature dependent types, the development is carried out switching conveniently between two different
existing libraries: the Hermite normal form (based on HOL Analysis) and the Jordan normal form AFP entries. This permits to reuse results from both developments and it is done by means of the lifting and transfer package together with the use of local type definitions.
\end{abstract}


\tableofcontents

% sane default for proof documents
\parindent 0pt\parskip 0.5ex

% generated text of all theories
\input{session}

% optional bibliography
%\bibliographystyle{abbrv}
%\bibliography{root}

\end{document}

%%% Local Variables:
%%% mode: latex
%%% TeX-master: t
%%% End:
