\documentclass[11pt,a4paper]{article}
\usepackage[T1]{fontenc}
\usepackage{isabelle,isabellesym}
\usepackage{amssymb}
\usepackage[english]{babel}

% this should be the last package used
\usepackage{pdfsetup}

% urls in roman style, theory text in math-similar italics
\urlstyle{rm}
\isabellestyle{it}


\begin{document}

\title{Relation Algebra}
\author{Alasdair Armstrong, Simon Foster, Georg Struth, Tjark Weber}
\maketitle

\begin{abstract}
Tarski's algebra of binary relations is formalised along the lines of
the standard textbooks of Maddux and Schmidt and Str\"ohlein. This
includes relation-algebraic concepts such as subidentities, vectors
and a domain operation as well as various notions associated to
functions. Relation algebras are also expanded by a reflexive
transitive closure operation, and they are linked with Kleene algebras
and models of binary relations and Boolean matrices.
\end{abstract}

\tableofcontents

% sane default for proof documents
\parindent 0pt\parskip 0.5ex

\section{Introductory Remarks}

These theory files are only sparsely commented. Background information
can be found in Tarski's original article~\cite{tarski41} and in the
books by Maddux~\cite{maddux06} and Schmidt and
Str{\"o}hlein~\cite{schmidt87}. We briefly discuss proof automation
and the formalisation of direct products in~\cite{armstrong14}.

% generated text of all theories
\input{session}

% optional bibliography
\bibliographystyle{abbrv}
\bibliography{root}

\end{document}

%%% Local Variables:
%%% mode: latex
%%% TeX-master: t
%%% End:
