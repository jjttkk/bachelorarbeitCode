\documentclass{article}
\usepackage[T1]{fontenc}
\usepackage{isabelle,isabellesym}
\usepackage{amsmath}

% further packages required for unusual symbols (see also
% isabellesym.sty), use only when needed

%\usepackage{amssymb}
  %for \<leadsto>, \<box>, \<diamond>, \<sqsupset>, \<mho>, \<Join>,
  %\<lhd>, \<lesssim>, \<greatersim>, \<lessapprox>, \<greaterapprox>,
  %\<triangleq>, \<yen>, \<lozenge>

%\usepackage{eurosym}
  %for \<euro>

%\usepackage[only,bigsqcap]{stmaryrd}
  %for \<Sqinter>

%\usepackage{eufrak}
  %for \<AA> ... \<ZZ>, \<aa> ... \<zz> (also included in amssymb)

%\usepackage{textcomp}
  %for \<onequarter>, \<onehalf>, \<threequarters>, \<degree>, \<cent>,
  %\<currency>

% this should be the last package used
\usepackage{pdfsetup}

\isadroptag{theory}
\isafoldtag{proof}

% urls in roman style, theory text in math-similar italics
\urlstyle{rm}
\isabellestyle{it}

% for uniform font size
%\renewcommand{\isastyle}{\isastyleminor}

\begin{document}

\title{A Case Study in Basic Algebra}
\author{Clemens Ballarin}
\date{}

\maketitle

\begin{abstract}
  The focus of this case study is re-use in abstract algebra.  It contains
  locale-based formalisations of selected parts of set, group and ring
  theory from Jacobson's \emph{Basic Algebra} leading to the
  respective fundamental homomorphism theorems.  The study is not
  intended as a library base for abstract algebra.  It rather explores
  an approach towards abstract algebra in Isabelle.
\end{abstract}

% sane default for proof documents
\parindent 0pt\parskip 0.5ex

% for proof reading
\pagestyle{plain}
% generated text of all theories
\input{session}

% optional bibliography
\bibliographystyle{abbrv}
\bibliography{root}

\end{document}
